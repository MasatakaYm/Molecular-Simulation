\documentclass[a4paper, 10.5pt, oneside, openany, uplatex]{jsarticle}

\author{山内 仁喬}
% 余白の設定.
% 参考文献:Latex2e 美文書作成入門, 14.3ページレイアウトの変更

% 行長の変更
\setlength{\textwidth}{40zw}           %全角40文字分

% 行間を制御するコマンド
\renewcommand{\baselinestretch}{0.9}

% 左マージンを変更
\setlength{\oddsidemargin}{25truemm}   % 左余白
\addtolength{\oddsidemargin}{-1truein} % 左位置デフォルトから-1inch

% 上マージンを変更
\setlength{\topmargin}{15truemm}       % 上余白
\addtolength{\topmargin}{-1truein}     % 上位置デフォルトから-1inch

% 本文領域の縦横の長さ変更
\setlength{\textheight}{242truemm}     % テキスト高さ: 297-(25+30)=242mm
\setlength{\textwidth}{160truemm}      % テキスト幅:  210-(25+25)=160mm
\setlength{\fullwidth}{\textwidth}     % ページ全体の幅


% 図・表の個数などの設定.
%% 図・表を入りやすさを制御するパラメーター
\setcounter{topnumber}{4}
\setcounter{bottomnumber}{4}
\setcounter{totalnumber}{4}
\setcounter{dbltopnumber}{3}
\setcounter{tocdepth}{1} % 項レベルまで目次に反映させるコマンド.
\renewcommand{\topfraction}{.95}
\renewcommand{\bottomfraction}{.90}
\renewcommand{\textfraction}{.05}
\renewcommand{\floatpagefraction}{.95}

% 使用するパッケージを記述.
\usepackage{amsmath} % 複雑な数式を使うときに便利
\usepackage{dcolumn}
\usepackage{color}
\usepackage{tabularx, dcolumn}
\usepackage{bm} % 数式環境内で太字を使うときに便利.
\usepackage{subcaption}  % 関連した複数の図を並べる時に使う
\usepackage[dvipdfmx]{graphicx} % 画像を挿入したり,テキストや図の拡大縮小・回転を行う.
\usepackage{verbatim} % 入力どおりの出力を行う.
\usepackage{makeidx} % 索引を作成できる.
\usepackage{dcolumn} % 表の数値を小数点で桁を揃える.
\usepackage{lscape} % 図表を90度横に倒して配置する.
\usepackage{setspace} % 行間調整.

\def\mbf#1{\mbox{\boldmath ${#1}$}}

% \newcolumntype{d}{D{+}{\,\pm\,}{4,5}}
% \newcolumntype{i}{D{+}{\,\pm\,}{-1}}
% \newcolumntype{.}{D{.}{.}{6,3}}

\input{../include/begin}

\title{自己相関関数}
\maketitle

\section{自己相関関数の定義}
自己相関関数$C(\tau)$は

\begin{equation}
    C(\tau)
    =
    \frac{\langle x(t) x(t + \tau) \rangle - \langle x(t) \rangle \langle x(t + \tau) \rangle}{\langle x(t) x(t) \rangle - \langle x(t) \rangle^{2}}
\end{equation}
と計算される.

\section{解析的に自己相関関数が計算できる関数の例}
ここでは解析的に自己相関関数が計算できる例として, 正弦波と余弦波の自己相関関数を紹介する.
このような例はプログラムを実装したときの確認として役に立つ.

\subsection{正弦波の自己相関関数}
正弦波を
\begin{equation}
    x(t) = a \sin(\omega t + \phi)
\end{equation}
とおく.
時刻$t+\tau$での正弦波は
\begin{align}
    x(t+\tau)
    &= a \sin(\omega (t + \tau) + \phi) \\
    &= a \sin(\omega t + \phi) \cos(\omega \tau)
     + a \cos(\omega t + \phi) \sin(\omega \tau)
\end{align}
であるので, その積は
\begin{align}
    x(t) x(t + \tau)
    &=
    a^{2} \sin^{2}(\omega t + \phi) \cos(\omega \tau) +
    a^{2} \sin(\omega t + \phi) \cos(\omega t + \phi) \sin(\omega \tau) \\
    &=
    \frac{a^{2}}{2} [1 - \cos(2\omega t + 2\phi)] \cos(\omega \tau) +
    \frac{a^{2}}{2} [\sin(2\omega t + 2\phi)] \sin(\omega \tau) \\
    &=
    \frac{a^{2}}{2}
    [ \cos(\omega \tau)
    + \sin(2\omega t + 2\phi)\sin(\omega \tau)
    - \cos(2\omega t + 2\phi)\cos(\omega \tau)] \\
    &=
    \frac{a^{2}}{2}
    [\cos(\omega \tau) - \cos (2\omega t + 2\phi + \omega \tau)]
\end{align}
と計算される.
続いて自己相関関数の計算に必要な平均を求めていく.
\begin{align}
    \langle x(t) \rangle = \langle a \sin(\omega t + \phi) \rangle = 0
\end{align}

\begin{align}
    \langle x(t+\tau) \rangle
    &=
    \langle
        a \sin(\omega t + \phi) \cos(\omega \tau) +
        a \cos(\omega t + \phi) \sin(\omega \tau)
    \rangle \\
    &=
    a [\cos(\omega \tau) \langle \sin(\omega t + \phi) \rangle +
       \sin(\omega \tau) \langle \cos(\omega t + \phi) \rangle ]\\
    &=
    0
\end{align}

\begin{align}
    \langle x(t) x(t+\tau) \rangle
    &=
    \frac{a^{2}}{2}
    \langle
        \cos(\omega \tau) - \cos(2\omega t + 2\phi + \omega \tau)
    \rangle \\
    &=
    \frac{a^{2}}{2} \cos(\omega \tau)
\end{align}

\begin{align}
    \langle x(t) x(t) \rangle
    &=
    \langle
        a^{2} \sin^{2} (\omega t + \phi)
    \rangle \\
    &=
    a^{2} \langle \sin^{2}(\omega t + \phi) \rangle
\end{align}
以上を用いると, 自己相関関数は
\begin{align}
    C(\tau)
    &=
    \frac{\langle x(t) x(t + \tau) \rangle - \langle x(t) \rangle \langle x(t + \tau) \rangle}{\langle x(t) x(t) \rangle - \langle x(t) \rangle^{2}} \\
    &=
    \frac{\frac{a^{2}}{2} \cos(\omega \tau)}{a^{2} \langle \sin^{2}(\omega t + \phi)
    \rangle} \\
    &=
    \frac{1}{2}
    \frac{\cos(\omega \tau)}{\langle \frac{1-\cos(2\omega t + 2\phi)}{2}\rangle} \\
    &=
    \frac{1}{2}
    \frac{\cos(\omega \tau)}{\frac{1}{2} [1 - \langle \cos(2\omega t + 2\phi) \rangle]} \\
    &=
    \cos(\omega \tau)
\end{align}
と計算される.
まとめると, 正弦波の自己相関関数は余弦波となる.


\subsection{余弦波の自己相関関数}
余弦波を
\begin{equation}
    x(t) = a \cos(\omega t + \phi)
\end{equation}
とおく.
時刻$t+\tau$での正弦波は
\begin{align}
    x(t+\tau)
    &= a \cos(\omega (t + \tau) + \phi) \\
    &= a \cos(\omega t + \phi) \cos(\omega \tau)
     + a \sin(\omega t + \phi) \sin(\omega \tau)
\end{align}
であるので, その積は
\begin{align}
    x(t) x(t + \tau)
    &=
    a^{2} \cos^{2}(\omega t + \phi) \cos(\omega \tau) -
    a^{2} \sin(\omega t + \phi) \cos(\omega t + \phi) \sin(\omega \tau) \\
    &=
    \frac{a^{2}}{2} [1 + \cos(2\omega t + 2\phi)] \cos(\omega \tau) -
    \frac{a^{2}}{2} [\sin(2\omega t + 2\phi)] \sin(\omega \tau) \\
    &=
    \frac{a^{2}}{2}
    [ \cos(\omega \tau)
    + \cos(2\omega t + 2\phi)\cos(\omega \tau)
    - \sin(2\omega t + 2\phi)\sin(\omega \tau)] \\
    &=
    \frac{a^{2}}{2}
    [\cos(\omega \tau) + \cos (2\omega t + 2\phi + \omega \tau)]
\end{align}
と計算される.
続いて自己相関関数の計算に必要な平均を求めていく.
\begin{align}
    \langle x(t) \rangle = \langle a \cos(\omega t + \phi) \rangle = 0
\end{align}

\begin{align}
    \langle x(t+\tau) \rangle
    &=
    \langle
        a \cos(\omega t + \phi) \cos(\omega \tau) -
        a \sin(\omega t + \phi) \sin(\omega \tau)
    \rangle \\
    &=
    a [\cos(\omega \tau) \langle \cos(\omega t + \phi) \rangle -
       \sin(\omega \tau) \langle \sin(\omega t + \phi) \rangle ]\\
    &=
    0
\end{align}

\begin{align}
    \langle x(t) x(t+\tau) \rangle
    &=
    \frac{a^{2}}{2} \cos(\omega \tau)
\end{align}

\begin{align}
    \langle x(t) x(t) \rangle
    &=
    \langle
        a^{2} \cos^{2} (\omega t + \phi)
    \rangle \\
    &=
    a^{2} \langle \frac{1}{2} [1 + \cos(2\omega t + 2\phi)] \rangle \\
    &=
    \frac{a^{2}}{2}
\end{align}
以上を用いると, 自己相関関数は
\begin{align}
    C(\tau)
    &=
    \frac{\langle x(t) x(t + \tau) \rangle - \langle x(t) \rangle \langle x(t + \tau) \rangle}{\langle x(t) x(t) \rangle - \langle x(t) \rangle^{2}} \\
    &=
    \frac{\frac{a^{2}}{2} \cos(\omega \tau)}{\frac{a^{2}}{2}} \\
    &=
    \cos(\omega \tau)
\end{align}
と計算される.
まとめると, 余弦波の自己相関関数は余弦波となる.

% \bibliographystyle{junsrt}
% \bibliography{}
\input{../include/end}
