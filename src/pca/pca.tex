\documentclass[a4paper, 10.5pt, oneside, openany, uplatex]{jsarticle}

\author{山内 仁喬}
% 余白の設定.
% 参考文献:Latex2e 美文書作成入門, 14.3ページレイアウトの変更

% 行長の変更
\setlength{\textwidth}{40zw}           %全角40文字分

% 行間を制御するコマンド
\renewcommand{\baselinestretch}{0.9}

% 左マージンを変更
\setlength{\oddsidemargin}{25truemm}   % 左余白
\addtolength{\oddsidemargin}{-1truein} % 左位置デフォルトから-1inch

% 上マージンを変更
\setlength{\topmargin}{15truemm}       % 上余白
\addtolength{\topmargin}{-1truein}     % 上位置デフォルトから-1inch

% 本文領域の縦横の長さ変更
\setlength{\textheight}{242truemm}     % テキスト高さ: 297-(25+30)=242mm
\setlength{\textwidth}{160truemm}      % テキスト幅:  210-(25+25)=160mm
\setlength{\fullwidth}{\textwidth}     % ページ全体の幅


% 図・表の個数などの設定.
%% 図・表を入りやすさを制御するパラメーター
\setcounter{topnumber}{4}
\setcounter{bottomnumber}{4}
\setcounter{totalnumber}{4}
\setcounter{dbltopnumber}{3}
\setcounter{tocdepth}{1} % 項レベルまで目次に反映させるコマンド.
\renewcommand{\topfraction}{.95}
\renewcommand{\bottomfraction}{.90}
\renewcommand{\textfraction}{.05}
\renewcommand{\floatpagefraction}{.95}

% 使用するパッケージを記述.
\usepackage{amsmath} % 複雑な数式を使うときに便利
\usepackage{dcolumn}
\usepackage{color}
\usepackage{tabularx, dcolumn}
\usepackage{bm} % 数式環境内で太字を使うときに便利.
\usepackage{subcaption}  % 関連した複数の図を並べる時に使う
\usepackage[dvipdfmx]{graphicx} % 画像を挿入したり,テキストや図の拡大縮小・回転を行う.
\usepackage{verbatim} % 入力どおりの出力を行う.
\usepackage{makeidx} % 索引を作成できる.
\usepackage{dcolumn} % 表の数値を小数点で桁を揃える.
\usepackage{lscape} % 図表を90度横に倒して配置する.
\usepackage{setspace} % 行間調整.

\def\mbf#1{\mbox{\boldmath ${#1}$}}

% \newcolumntype{d}{D{+}{\,\pm\,}{4,5}}
% \newcolumntype{i}{D{+}{\,\pm\,}{-1}}
% \newcolumntype{.}{D{.}{.}{6,3}}

\begin{document}


\title{主成分解析 (PCA: Principal Component Analysis)}
\maketitle

主成分解析\cite{1981Karplus}は, 多次元空間の中で生体分子がどの
ような挙動を示しているのか解析する際に用いられる.
蛋白質の揺らぎは異方性が高いため, 物理量とも関連付けられることができる
\cite{1991Kitao, 1999Kitao, 2001Kitao}.
また, シミュレーションで得られた構造のクラスタリングをする際にも主成分
解析は有用である.

\section{主成分解析の基礎}
主成分解析では, 原子座標$\mathbf{Q}$を原子座標の線形結合として表される独立な集団座標 (=主成分座標) $\mathbf{\Xi} = \{\xi_{ak}\}$ への線形変換を考える.
原子座標から主成分座標への変換行列を$\mathbf{A}^{\mathrm{t}}$とすると
\begin{align}
  \mathbf{\Xi} &= \mathbf{A}^{\mathrm{t}} \mathbf{Q}
\end{align}
とかける.
以降の便利のため, $k$番目の構造における座標ベクトルを$\mathbf{Q}_{k}$, $\mathbf{\Xi}_{k}$と書くことにする.
基底が正規直交であるとすると,
\begin{align}
  \mathbf{A}^{\mathrm{t}} \mathbf{A} =
  \mathbf{A} \mathbf{A}^{\mathrm{t}} =
  \mathbf{I}
\end{align}
が成立する.
ここで$\mathbf{I}$は単位行列である.
主成分分析では, 新たに得られた軸に置いてデータの分散が最大になるように線形変換を施す.
変換後の分散は
\begin{align}
  \Tilde{\sigma}^{2} (\mathbf{A}) &=
  \sum_{k=1}^{K} w_{k}
  (\mathbf{\Xi}_{k} - \langle \mathbf{\Xi} \rangle)^{\mathrm{t}}
  (\mathbf{\Xi}_{k} - \langle \mathbf{\Xi} \rangle)
  \notag
  \\
  &=
  \sum_{k=1}^{K} w_{k}
  \left\{
         \mathbf{A}^{\mathrm{t}}
         (\mathbf{Q}_{k} - \langle \mathbf{Q} \rangle)
  \right\}^{\mathrm{t}}
  \left\{
         \mathbf{A}^{\mathrm{t}}
         (\mathbf{Q}_{k} - \langle \mathbf{Q} \rangle)
  \right\}
  \notag
  \\
  &=
  \sum_{k=1}^{K} w_{k}
  \mathrm{Tr}
  \left[
        \mathbf{A}^{\mathrm{t}}
        (\mathbf{Q}_{k} - \langle \mathbf{Q} \rangle)
        (\mathbf{Q}_{k} - \langle \mathbf{Q} \rangle)^{\mathrm{t}}
        \mathbf{A}
  \right]
  \notag
  \\
  &=
  \mathrm{Tr}
  \left[
        \mathbf{A}^{\mathrm{t}}
        \sum_{k=1}^{K} w_{k}
        (\mathbf{Q}_{k} - \langle \mathbf{Q} \rangle)
        (\mathbf{Q}_{k} - \langle \mathbf{Q} \rangle)^{\mathrm{t}}
        \mathbf{A}
  \right]
  \notag
  \\
  &=
  \mathrm{Tr}
  \left[
        \mathbf{A}^{\mathrm{t}}
        \mathbf{\Sigma}
        \mathbf{A}
  \right]
  \label{eq:variance}
\end{align}
と計算される.
ここで, データの平均を計算するためにデータの重み$w_{k}$を導入した.
また, $\mathbf{\Sigma}$は分散共分散行列である.
また, 途中の式変形において
$\mathbf{x}^{\mathrm{t}}\mathbf{x} = \mathrm{Tr}(\mathbf{x}\mathbf{x}^{\mathrm{t}})$が成り立つことを使用た.
$\mathbf{A}^{\mathrm{t}}\mathbf{A}$の制約のもとで, 式(\ref{eq:variance})を最大にする$\mathbf{A}$を求める問題に帰着する:
 \begin{equation}
  J(\mathbf{A})
  \equiv
  \mathrm{Tr}
  [\mathbf{A}^{\mathrm{t}} \mathbf{\Sigma} \mathbf{A} ]
  -
  [ (\mathbf{A}^{\mathrm{t}}\mathbf{A} - \mathrm{I})\mathbf{\Sigma}  ]
 \end{equation}
つまり, $\Sigma$は対角行列である.
$\mathbf{A}$で偏微分すると
\begin{align}
  \frac{\partial J(\mathbf{A})}{\partial \mathbf{A}} =
  2 \mathbf{\Sigma}\mathbf{A} - 2 \mathbf{A} \mathbf{\Sigma} = 0
\end{align}
ここで, (n, m)行列$\mathbf{X}$, $n$次元正方行列$\mathbf{Y}$について
\begin{align}
  \frac{\partial}{\partial \mathbf{X}}
  \mathrm{Tr}
  \{ \mathbf{X}^{\mathrm{t}} \mathbf{Y} \mathrm{X} \} =
  (\mathbf{Y} + \mathbf{Y}^{\mathrm{t}}) \mathbf{X}
\end{align}
を用いた. ゆえに,
\begin{align}
  \mathbf{\Sigma} \mathbf{A} = \mathbf{A} \mathbf{\Sigma}
  \notag
  \\
  \mathbf{A}^{\mathrm{t}}\mathbf{\Sigma} \mathbf{A} = \mathbf{\Sigma}
\end{align}
となり, $\mathbf{A}^{\mathrm{t}}$は$\mathbf{\Sigma}$を対角化する行列となる.


\section{主成分解析のタンパク質への応用}
$N$個の原子からなる系を考える.
ここでは各原子の座標$x, y, z$を, より一般的な形である$q$で表すことにする.
実際に主成分解析を行う場合, 全原子を扱わずに主鎖の原子座標やC$_{\alpha}$原子の座標のみに適用することが多い.
その他の内部座標(例えば二面角や各アミノ酸残基の重心座標)を用いてもよい.
分子シミュレーションによって, $K$個の構造を得たとする.
各瞬間座標を, それぞれの平均座標からの変位に変換して, 質量の平方根で重み付けをする.
瞬間構造の座標を行方向, 時系列を列方向にまとめると, $3N \times K$行列
\begin{align}
 \mathbf{Q} =
 \left\{ Q_{ik} \right\} =
 \left\{
         \sqrt{m_{i}} \left(q_{ik} - \langle q_{i} \rangle\right)
 \right\}
\end{align}
を得る.
ここで, $\langle \mathbf{q}_{i} \rangle$は座標$\mathbf{q}_{ik}$の平均構造である.
$\langle \mathbf{q}_{i} \rangle$は以下の手順で求められる:
まず仮の$\mathbf{q}_{ik}^{\prime}$
\footnote{自由度の数が同じであれば, どんな座標値でもいい. 例えば, 初期構造$\mathbf{q}_{i1}$などを用いる.}
を定める. 続いて, $\mathbf{q}_{ik}^{\prime}$に対して$K$個の瞬間構造を全てbest fitさせて平均構造を求める.
さらに求めた平均構造に対して瞬間構造全てをbest fitし平均構造を計算する.
この操作を繰り返すことで平均構造を収束させる.
通常は数回のイテレーションで平均構造に収束するはずである.

主成分解析では, 原子座標$\mathbf{Q}$を原子座標の線形結合として表される独立な集団座標 (=主成分座標) $\mathbf{\Sigma} = \{\sigma_{ak}\}$ への線形変換を考える.
原子座標から主成分座標への変換行列を $\mathbf{P}^{\mathrm{t}}$ とすると
\begin{align}
  \mathbf{\Sigma} &= \mathbf{P}^{\mathrm{t}} \mathbf{Q} %~, \\
  %\mathbf{Q}      &= \mathbf{P} \mathbf{\Sigma}
\end{align}
とかける.
ここで, 分散共分散行列$\mathbf{R}$を与える:
\begin{align}
  \mathbf{R} &=
  \langle \mathbf{Q}^{\mathrm{t}} \mathbf{Q} \rangle
  \notag
  \\
  &=
  \left\{
  \sqrt{m_{i} m_{j}}
        \left\langle
                    \left( q_{ik} - \langle q_{i} \rangle \right)
                    \left( q_{jk} - \langle q_{j} \rangle \right)
        \right\rangle
  \right\}
  \notag
  \\
  &=
  \left\{
        \sqrt{m_{i} m_{j}}
        \left(
               \left\langle q_{ik} q_{jk} \right\rangle
             - \left\langle q_{i} \right\rangle
               \left\langle q_{j} \right\rangle
        \right)
  \right\} .
  \notag
\end{align}
この行列は明らかに対称行列である.
各構造に対して重み$w_{k}$が課せられているとすると, 平均は
\begin{align}
  \left\langle q_{i} \right\rangle
  &=
  \sum_{k=1}^{K} w_{k} q_{ik}
  \notag
  \\
  \left\langle q_{i} q_{j} \right\rangle
  &=
  \sum_{k=1}^{K} w_{k} (q_{ik} q_{jk})
  \notag
\end{align}
と計算される. ただし重み$w_{k}$は規格化されているものとする.

原子座標から主成分座標への変換行列$\mathbf{P}^{\mathrm{t}}$は, 原子座標の分散共分散行列$\mathbf{R}$の標準固有値問題 (対角化)から得られる.
\begin{align}
  \mathbf{R}\mathbf{P} =  \mathbf{P} \mathbf{\Lambda} % \\
 % \mathbf{P}\mathbf{P}^{\mathrm{t}} = \mathbf{P}^{\mathrm{t}} \mathbf{P} = \mathbf{I}
\end{align}
ここで$\mathbf{\Lambda}$は対角行列である. その成分 $\lambda_{a} (a = 1,2, \cdots, 3N)$ は固有値であり, 主成分座標方向の分散を表す:
 $\lambda_{a} = \langle \sigma_{ak} ^{2} \rangle$.
事前に原子座標に対してbest fitをして並進と回転の自由度を取り除いている場合, 6つの固有値がゼロとなる.
行列$\mathbf{P}$の各列は, 分散共分散行列$\mathbf{R}$の固有ベクトル$\mathbf{u}_{a}$である:
\begin{align}
  \mathbf{P} =
  \left[
        \begin{array}{cccc}
         \mathbf{u}_{1} &
         \mathbf{u}_{2} &
         \cdots &
         \mathbf{u}_{3N} \\
         \end{array}
  \right]~.
\end{align}
固有ベクトル$\mathbf{u}_{a}$はa番目の主成分軸に対応する.
また, 固有ベクトルは規格化されているものとする. つまり, 変換行列は
$\mathbf{P}\mathbf{P}^{\mathrm{t}} = \mathbf{P}^{\mathrm{t}} \mathbf{P} = \mathbf{I}$ を満たす.
原子座標$\mathbf{Q}$は平均構造からの変位であることから, 座標軸の原点は分布の平均値になっている.

主成分座標$\mathbf{\Sigma} = \{\sigma_{ak}\}$を得るには, $\mathbf{Q}$を主成分座標に射影すればよい:
\begin{align}
   \mathbf{\Sigma}
 = \{\sigma_{ak}\}
&= \mathbf{P}^{\mathrm{t}} \mathbf{Q}.
\end{align}
$\{\sigma_{ak}\}$の$a$行目の列ベクトル $(k = 1, 2, \cdots, K)$は
第$a$主成分座標の時系列$\sigma_{a}(t_{k})$であり, 具体的に次のように計算される:
\begin{align}
  \mathbf{\sigma}_{a} (t_{k}) &=
   \left[
         \begin{array}{cccc}
           u_{1}^{(1)}  & u_{1}^{(2)}  & \cdots & u_{1}^{(3N)}  \\
           u_{2}^{(1)}  & u_{2}^{(2)}  & \cdots & u_{2}^{(3N)}  \\
             \vdots     &   \vdots     & \ddots &  \vdots       \\
           u_{3N}^{(1)} & u_{3N}^{(2)} & \cdots & u_{3N}^{(3N)} \\
         \end{array}
  \right]
  \left[
        \begin{array}{c}
          Q_{1} (t_{k}) \\
          Q_{2} (t_{k}) \\
          \vdots \\
          Q_{3N} (t_{k})
        \end{array}
  \right]
\\
&=
  \left[
        \begin{array}{c}
          u_{1}^{(1)}  Q_{1} (t_{k}) + u_{1}^{(2)}  Q_{2} (t_{k}) +
          \cdots u_{1}^{(3N)}  Q_{3N} (t_{k})
          \\
          u_{2}^{(1)}  Q_{1} (t_{k}) + u_{2}^{(2)}  Q_{2} (t_{k}) +
          \cdots u_{2}^{(3N)}  Q_{3N} (t_{k})
          \\
          \vdots
          \\
          u_{3N}^{(1)} Q_{1} (t_{k}) + u_{3N}^{(2)} Q_{2} (t_{k}) +
          \cdots u_{3N}^{(3N)} Q_{3N} (t_{k})
        \end{array}
   \right] .
\end{align}

$\lambda_{a}$は$\sigma_{ak}$の$k$に関する分散であることから次の式が成立する:
\begin{align}
 \lambda_{a} =
 \langle \sigma_{ak}^{2} \rangle =
 \sum_{i=1}^{3N} v_{ik}^{2} \langle q_{ak}^{2} \rangle.
\end{align}
また, 分散の総和への主成分$a$の寄与率は
\begin{align}
  \frac{\lambda_{a}}{\sum_{a^{\prime}=1}^{3N-6} \lambda_{a^{\prime}}}
\end{align}
から計算することができる.

$\mathbf{\Sigma}$を標準偏差である$\lambda^{-\frac{1}{2}}$でスケールして
\begin{align}
  \mathbf{U}^{\mathrm{t}} =
  \lambda^{-\frac{1}{2}} \mathbf{\Sigma} =
  \lambda^{-\frac{1}{2}} \mathbf{P}^{\mathrm{t}} \mathbf{Q}
\end{align}
とすると, $\mathbf{U}^{\mathrm{t}}\mathbf{U} = \mathbf{I}$である.
$K \times 3N$行列$\mathbf{Q}^{\mathrm{t}}$の特異値分解
\begin{align}
  \mathbf{Q}^{\mathrm{t}} = \mathbf{U} \lambda^{\frac{1}{2}} \mathbf{P}^{\mathrm{t}}
\end{align}
からも変換行列$\mathbf{P}^{\mathrm{t}}$を得ることができる.

エネルギー面が完全に多次元のパラボラとなっている時は,
得られる変換行列は基準振動解析の値と一致する.
この時, 固有値$\lambda_{a}$は固有振動数$\omega_{a}$と, 次の関係を持つ\cite{1991Kitao, 1999Kitao, 2001Kitao}:
\begin{align}
  \lambda_{a} = \frac{k_{B}T}{\omega_{a}^{2}}.
\end{align}

\section{PCAの実行例}

簡単な例題を通して, PCAがどのように実行されるかを確認する. 
このような例は, プログラムを実装したときのデバッグに役に立つ. 

\subsection{PCAを解析的実行した例1}
次のデータの集合$\{\mathbf{x}_{i}\}$を考える:
\begin{align}
  \mathbf{x}_{1} = (1,2),~~
  \mathbf{x}_{2} = (2,4),~~
  \mathbf{x}_{3} = (3,6)
\end{align}
あるいは, データの平均がゼロになるようにシフトしたデータの集合$\{\mathbf{x}_{i}^{\prime}\}$
\begin{align}
  \mathbf{x}_{1}^{\prime} = (-1,-2),~~
  \mathbf{x}_{2}^{\prime} = (0,0),~~
  \mathbf{x}_{3}^{\prime} = (1,2)
\end{align}
を考えても得られる主成分軸は変わらないが, 今回は元の集合$\{\mathbf{x}_{i}\}$を考えていく. 
\footnote{実際は, シフトしたデータの集合$\{\mathbf{x}_{i}^{\prime}\}$を用いた方が, 分散共分散行列の$\langle x \rangle$や$\langle y \rangle$項が0となるため, 計算が楽である. }
分散共分散行列は, 
\begin{align}
  \begin{bmatrix}
    \langle x^{2} \rangle - \langle x \rangle^{2} &
    \langle xy    \rangle - \langle x \rangle \langle y \rangle \\
    \langle xy    \rangle - \langle x \rangle \langle y \rangle &
    \langle y^{2} \rangle - \langle y \rangle^{2}
  \end{bmatrix}
  &=
  \begin{bmatrix}
    \frac{1+4+9}{3} - \left(\frac{1+2+3}{3}\right)^{2} &
    \frac{2+9+18}{3} - \left(\frac{1+2+3}{3}\right)\left(\frac{2+4+6}{3}\right) \\
    \frac{2+9+18}{3} - \left(\frac{1+2+3}{3}\right)\left(\frac{2+4+6}{3}\right) &
    \frac{4+16+36}{3} - \left(\frac{2+4+6}{3}\right)^{2}
  \end{bmatrix}
  \notag \\ &=
  \begin{bmatrix}
    0.666\ldots & 1.333\ldots \\
    1.333\ldots & 2.666\ldots
  \end{bmatrix}
  \notag
\end{align}
と計算される. 
分散共分散行列の固有値と固有ベクトルを求めると
\begin{alignat}{4}
  \lambda_{1} &= 3.333\ldots, ~~~~&& \mathbf{u}_{1} &&=  ( 0.45, 0.89)^{t} \\
  \lambda_{2} &= 0.000\ldots, ~~~~&& \mathbf{u}_{2} &&=  (-0.89, 0.45)^{t}
\end{alignat}
を得る. 
最後に, 元のデータを主成分軸上に射影する. 
通常平均値が主成分軸の原点になるようにするので, シフトしたデータの集合$\{\mathbf{x}_{i}^{\prime}\}$を主成分軸上に射影して, 新たな座標の集合$\mathbf{y}$を得る. 
\begin{align}
  \mathbf{y} &=
  \begin{bmatrix}
     0.45 & 0.89 \\
    -0.89 & 0.45
  \end{bmatrix}
  \begin{bmatrix}
    -1 & 0 & 1 \\
    -2 & 0 & 2
 \end{bmatrix}
 \\
 &\simeq
 \begin{bmatrix}
  -2.23 & 0 & 2.23 \\
   0    & 0 & 0
\end{bmatrix}
\end{align}

\subsection{PCAを解析的実行した例2}
次のデータの集合$\{\mathbf{x}_{i}\}$を考える:
\begin{alignat}{3}
  \mathbf{x}_{1} &= (1,1),~~ &
  \mathbf{x}_{2} &= (3,3),~~ &
  \mathbf{x}_{3} &= (5,5),~~ \notag \\
  \mathbf{x}_{4} &= (2,4),~~ &
  \mathbf{x}_{5} &= (4,2)    &\notag
  %\mathbf{x}_{\mathrm{ave}} &= (3,~3)
\end{alignat}
なお, データ点の平均値は$\mathbf{x}_{\mathrm{ave}}= (3,~3)$である. 
データの平均がゼロになるようにシフトしたデータの集合$\{\mathbf{x}_{i}^{\prime}\}$は
\begin{alignat}{3}
  \mathbf{x}_{1}^{\prime} &= (-2,-2),~~&
  \mathbf{x}_{2}^{\prime} &= (0,0),~~  &
  \mathbf{x}_{3}^{\prime} &= (2,2),~~  \notag \\
  \mathbf{x}_{4}^{\prime} &= (-1,1),~~ &
  \mathbf{x}_{5}^{\prime} &= (1,-1),   &\notag
\end{alignat}
である. 
分散共分散行列は, 
\begin{align}
  \begin{bmatrix}
    \langle x^{2} \rangle - \langle x \rangle^{2} &
    \langle xy    \rangle - \langle x \rangle \langle y \rangle \\
    \langle xy    \rangle - \langle x \rangle \langle y \rangle &
    \langle y^{2} \rangle - \langle y \rangle^{2}
  \end{bmatrix}
  &=
  \begin{bmatrix}
    \frac{4+0+4+1+1}{5} & \frac{4+0+4-1-1}{5} \\
    \frac{4+0+4-1-1}{5} & \frac{4+0+4+1+1}{5} \\
  \end{bmatrix}
  \notag \\ &=
  \begin{bmatrix}
    2   & 1.2 \\
    1.2 & 2
  \end{bmatrix}
  \notag
\end{align}
分散共分散行列の固有値と固有ベクトルを求めると
\begin{alignat}{4}
  \lambda_{1} &= 3.2 ~~~~&& \mathbf{u}_{1} &&= ( 0.71, 0.71)^{t} \\
  \lambda_{2} &= 0.8 ~~~~&& \mathbf{u}_{2} &&= (-0.71, 0.71)^{t}
\end{alignat}
を得る. 
最後に, 元のデータを主成分軸上に射影する. 
通常平均値が主成分軸の原点になるようにするので, シフトしたデータの集合$\{\mathbf{x}_{i}^{\prime}\}$を主成分軸上に射影して, 新たな座標の集合$\mathbf{y}$を得る. 
\begin{align}
  \mathbf{y} &=
  \begin{bmatrix}
     0.71 & 0.71 \\
    -0.71 & 0.71
  \end{bmatrix}
  \begin{bmatrix}
    -2 & 0 & 2 & -1 &  1 \\
    -2 & 0 & 2 &  1 & -1
 \end{bmatrix}
 \\
 &\simeq
 \begin{bmatrix}
  -2.84 & 0 & 2.84 & 0    &  0 \\
   0    & 0 & 0    & 1.42 & -1.42
\end{bmatrix}
\end{align}


\bibliographystyle{junsrt}
\bibliography{pca}
\end{document}

