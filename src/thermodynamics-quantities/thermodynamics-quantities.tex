\documentclass[a4paper, 10.5pt, oneside, openany, uplatex]{jsarticle}

\author{山内 仁喬}
% 余白の設定.
% 参考文献:Latex2e 美文書作成入門, 14.3ページレイアウトの変更

% 行長の変更
\setlength{\textwidth}{40zw}           %全角40文字分

% 行間を制御するコマンド
\renewcommand{\baselinestretch}{0.9}

% 左マージンを変更
\setlength{\oddsidemargin}{25truemm}   % 左余白
\addtolength{\oddsidemargin}{-1truein} % 左位置デフォルトから-1inch

% 上マージンを変更
\setlength{\topmargin}{15truemm}       % 上余白
\addtolength{\topmargin}{-1truein}     % 上位置デフォルトから-1inch

% 本文領域の縦横の長さ変更
\setlength{\textheight}{242truemm}     % テキスト高さ: 297-(25+30)=242mm
\setlength{\textwidth}{160truemm}      % テキスト幅:  210-(25+25)=160mm
\setlength{\fullwidth}{\textwidth}     % ページ全体の幅


% 図・表の個数などの設定.
%% 図・表を入りやすさを制御するパラメーター
\setcounter{topnumber}{4}
\setcounter{bottomnumber}{4}
\setcounter{totalnumber}{4}
\setcounter{dbltopnumber}{3}
\setcounter{tocdepth}{1} % 項レベルまで目次に反映させるコマンド.
\renewcommand{\topfraction}{.95}
\renewcommand{\bottomfraction}{.90}
\renewcommand{\textfraction}{.05}
\renewcommand{\floatpagefraction}{.95}

% 使用するパッケージを記述.
\usepackage{amsmath} % 複雑な数式を使うときに便利
\usepackage{dcolumn}
\usepackage{color}
\usepackage{tabularx, dcolumn}
\usepackage{bm} % 数式環境内で太字を使うときに便利.
\usepackage{subcaption}  % 関連した複数の図を並べる時に使う
\usepackage[dvipdfmx]{graphicx} % 画像を挿入したり,テキストや図の拡大縮小・回転を行う.
\usepackage{verbatim} % 入力どおりの出力を行う.
\usepackage{makeidx} % 索引を作成できる.
\usepackage{dcolumn} % 表の数値を小数点で桁を揃える.
\usepackage{lscape} % 図表を90度横に倒して配置する.
\usepackage{setspace} % 行間調整.

\def\mbf#1{\mbox{\boldmath ${#1}$}}

% \newcolumntype{d}{D{+}{\,\pm\,}{4,5}}
% \newcolumntype{i}{D{+}{\,\pm\,}{-1}}
% \newcolumntype{.}{D{.}{.}{6,3}}

\input{../include/begin}

\title{熱力学量の算出}
\maketitle

\section{熱力学量}
シミュレーションから状態Aと状態B間の熱力学量差を計算する方法を説明する. 
状態Aと状態Bの存在確率をそれぞれ$f_{A}$, $f_{B}$とすると, 
2状態間のギブスの自由エネルギー差 $\Delta G$は次のように計算される:
\begin{equation}
    \Delta G = G_{B} - G_{A} =
    RT \log \left( \frac{f_{A}}{f_{B}} \right)
    \label{Eq:DeltaG}
\end{equation}
ここで$R$は気体定数であり, $R=8.3145$ J/(mol K). 
求めた$\Delta G$から, 
部分モルエンタルピー差$\Delta H$, 定圧比熱差$\Delta C_{P}$, 部分モル体積差$\Delta V$
はそれぞれ
\begin{alignat}{2}
    &\Delta H &&=
    \left[\frac{\partial \left(\Delta G / T\right)}{\partial \left( 1/T \right)}\right]_{P} =
    R \left[\frac{\partial \log (f_{A}/f_{B})} {\partial(1/T)} \right]_{P}
    \label{Eq:DeltaH} \\
    &\Delta C_{p} &&=
    - T \left(\frac{\partial^{2} \Delta G}{\partial T^{2}}\right)_{P}
    \label{Eq:DeltaCp} \\
    &\Delta V &&=
    \left[\frac{\partial \Delta G}{\partial P}\right]_{T} =
    RT \left[\frac{\partial \log (f_{A}/f_{B})}{\partial P} \right]_{T}
  \label{Eq:DeltaV}
\end{alignat}
と計算される. 
さらに, 内部エネルギー差$\Delta U$とエンタルピー差$\Delta S$はそれぞれ
\begin{align}
    \Delta U &= \Delta H - P \Delta V
    \label{Eq:DeltaU} \\
    \Delta S &= \frac{\Delta H - \Delta G}{T}
    \label{Eq:DeltaS}
\end{align}
と計算される. 

\section{Hawleyの方程式}
二状態間のギブスの自由エネルギー差$\Delta G$の温度・圧力依存性はHawley~\cite{Hawley1971}によって提案された式
\begin{align}
    \Delta G(T,P) &=
    \Delta G_{0} - \Delta S_{0} (T - T_{0}) - \Delta C_{P} \left[T \left\{ \ln \left(\frac{T}{T_{0}} \right) -1 \right\} + T_{0} \right]
    \notag \\ &+
    \Delta V_{0}(P-P_{0}) + \frac{\Delta \beta}{2}(P-P_{0})^{2} + \Delta \alpha (P-P_{0})(T-T_{0})
    \label{Eq:Hawley-DeltaG}
\end{align}
を用いてフィッティングすることができる. 
ここで, $T_{0}$と$P_{0}$はそれぞれ参照温度, 参照圧力である. 
$\Delta G_{0}$, $\Delta S_{0}$, $\Delta V_{0}$はそれぞれ
温度$T_{0}$, 圧力$P_{0}$におけるギブスの自由エネルギー差, エントロピー差, 部分モル体積差である. 
$\Delta C_{p}$は定圧比熱差である. 
また, $\Delta \alpha$は熱膨張係数$\Delta \beta$は熱圧縮率である. 
通常, $T_{0}$, $P_{0}$における$\Delta G_{0}$は基準点として事前に算出しておく. 
したがって, $\Delta S_{0}$, $\Delta V_{0}$, $\Delta C_{p}$
$\Delta \beta$, $\Delta \alpha$の5つがフィティングパラメータとなる. 

式~(\ref{Eq:Hawley-DeltaG})に基づくと, 圧力$P_{0}$における$\Delta G$の温度依存性と, 
温度$T_{0}$における$\Delta G$の圧力依存性は次の式でフィッティングできる. 
\begin{align}
 \Delta G(T) =
 & \Delta G_{0} - \Delta S_{0} (T - T_{0}) - \Delta C_{P} \left[T \left\{ \ln \left(\frac{T}{T_{0}} \right) -1 \right\} + T_{0} \right]
 \label{Eq:Hawley-DeltaG-T}
 \\
 \Delta G(P) =
 & \Delta G_{0} + \Delta V_{0}(P-P_{0}) + \frac{\Delta \beta}{2}(P-P_{0})^{2}
 \label{Eq:Hawley-DeltaG-P}
\end{align}

\subsection{Hawleyの式の導出}
ギブスの自由エネルギー
\begin{equation}
    \Delta G = \Delta U + P\Delta V - T \Delta S
\end{equation}
の全微分は
\begin{align}
    d(\Delta G) &=
    \frac{\partial \Delta G}{\partial T} dT +
    \frac{\partial \Delta G}{\partial P} dP
    \\ &=
    -\Delta S dT + \Delta V dP
    \label{Eq:d-DeltaG}
\end{align}
である. 
定圧比熱差$\Delta C_{p}$, 熱膨張係数$\Delta \alpha$, 熱圧縮率$\Delta \beta$は, 
\begin{align}
    \Delta \alpha &=
    \left. \frac{\partial}{\partial T} \right|_{P}
    \left. \frac{\partial \Delta G}{\partial P} \right|_{T} \\
    \Delta \beta &=
    \frac{\partial^{2} \Delta G}{\partial^{2}P} =
    \left. \frac{\partial V}{\partial P} \right|_{T} \\
    \Delta C_{p} &=
    \left. -T \frac{\partial^{2} \Delta G}{\partial^{2}T}\right|_{P} =
    \left. T \frac{\partial \Delta S}{\partial T}\right|_{P}
\end{align}
である. 
以下の導出において, 定圧比熱差$\Delta C_{p}$, 熱膨張係数$\Delta \alpha$, 熱圧縮率$\Delta \beta$は, 
温度・圧力に依存しない定数と仮定する. 
Hawlay式を得るために, 式(\ref{Eq:d-DeltaG})中の, $\Delta S$と$\Delta V$を具体的に計算する必要がある. 
$\Delta S$と$\Delta V$の独立変数は, 温度と圧力であるから, その全微分は
\begin{align}
    d(\Delta S) &=
    \frac{\partial \Delta S}{\partial T} dT + \frac{\partial \Delta S}{\partial P} dP \\ &=
    \frac{\partial \Delta S}{\partial T} dT - \frac{\partial \Delta V}{\partial T} dP \\ &=
    \frac{\Delta C_{p}}{T}dT - \Delta \alpha dP
\end{align}
である. 第一行目から第二行目の変形においてMaxwellの関係式
\begin{equation}
    \frac{\partial \Delta S}{\partial P} = - \frac{\partial \Delta V}{\partial T}
\end{equation}
を使用した. 
同様に, $\Delta V$の全微分は
\begin{align}
    d(\Delta V) &=
    \frac{\partial \Delta V}{\partial T} dT + \frac{\partial \Delta V}{\partial P} dP \\ &=
    \Delta \alpha dT + \Delta \beta dP
\end{align}
である. これらの式を積分すると, 
\begin{align}
    \Delta S &=
    \Delta S_{0} + \int_{T_{0}}^{T} \frac{\Delta C_{p}}{T}dT - \int_{P_{0}}^{P} \Delta \alpha dP \notag \\ &=
    \Delta S_{0} + \Delta C_{p} \ln(T/T_{0}) + \Delta \alpha (P - P_{0})
\end{align}
\begin{align}
    \Delta V &=
    \Delta V_{0} + \Delta \alpha (T - T_{0}) + \Delta \beta (P - P_{0})
\end{align}
を得る. 式(\ref{Eq:d-DeltaG})に代入すると, 
\begin{align}
    d(\Delta G) =
    &-
    \left[
            \Delta S_{0} + \Delta C_{p} \ln(T/T_{0}) + \Delta \alpha (P - P_{0})
    \right] dT \notag \\ &+
    \left[
            \Delta V_{0} + \Delta \alpha (T - T_{0}) + \Delta \beta (P - P_{0})
    \right] dP
\end{align}
となる. これを積分するとHawleyの式, 
\begin{align}
    \Delta G(T,P) &=
    \Delta G_{0} - \Delta S_{0} (T - T_{0}) -
    \Delta C_{P} \left[T \ln \left(\frac{T}{T_{0}} \right) - (T - T_{0}) \right]
    \notag \\ &+
    \Delta V_{0}(P-P_{0}) + \frac{\Delta \beta}{2}(P-P_{0})^{2} + \Delta \alpha (P-P_{0})(T-T_{0})
\end{align}
が得られる. 

\subsection{部分モルエンタルピーと部分モル体積}

式~(\ref{Eq:Hawley-DeltaG})を式~(\ref{Eq:DeltaH})と式~(\ref{Eq:DeltaV})に代入する. 
部分モルエンタルピー差$\Delta H$は
\begin{align}
 \Delta H (T,P)
 &=  \Delta G_{0} + T_{0} \Delta S_{0} + (T - T_{0})\Delta C_{p}
 \notag
 \\ &~~~~
  + \Delta V_{0} (P - P_{0}) + \frac{\Delta \beta}{2} (P - P_{0})^{2}
  - \Delta \alpha (P - P_{0})T_{0}
 \label{Eq:deltaHTP_Hawley}
\end{align}
と計算される. 導出の途中で, $1/T=x$と置いて, 
\begin{align}
    \frac{\partial}{\partial(1/T)} \ln \frac{T_{0}}{T}=
    \frac{dx}{d(1/T)}\frac{d}{dx} \ln(T_{0}x) =
    T_{0} \frac{1}{T_{0}x} =
    \frac{1}{x} =
    T
\end{align}
の関係式を用いた. 
また, 部分モル体積差$\Delta V$は
\begin{align}
 \Delta V (T,P)
 &= \Delta V_{0} + \Delta \beta (P - P_{0}) + \Delta \alpha (T - T_{0})
 \label{Eq:deltaVTP_Hawley}
\end{align}
と計算される. 

\bibliographystyle{junsrt}
\bibliography{thermodynamics-quantities}
\input{../include/end}
