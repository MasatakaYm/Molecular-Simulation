\documentclass[a4paper, 10.5pt, oneside, openany, uplatex]{jsarticle}

\author{山内 仁喬}
% 余白の設定.
% 参考文献:Latex2e 美文書作成入門, 14.3ページレイアウトの変更

% 行長の変更
\setlength{\textwidth}{40zw}           %全角40文字分

% 行間を制御するコマンド
\renewcommand{\baselinestretch}{0.9}

% 左マージンを変更
\setlength{\oddsidemargin}{25truemm}   % 左余白
\addtolength{\oddsidemargin}{-1truein} % 左位置デフォルトから-1inch

% 上マージンを変更
\setlength{\topmargin}{15truemm}       % 上余白
\addtolength{\topmargin}{-1truein}     % 上位置デフォルトから-1inch

% 本文領域の縦横の長さ変更
\setlength{\textheight}{242truemm}     % テキスト高さ: 297-(25+30)=242mm
\setlength{\textwidth}{160truemm}      % テキスト幅:  210-(25+25)=160mm
\setlength{\fullwidth}{\textwidth}     % ページ全体の幅


% 図・表の個数などの設定.
%% 図・表を入りやすさを制御するパラメーター
\setcounter{topnumber}{4}
\setcounter{bottomnumber}{4}
\setcounter{totalnumber}{4}
\setcounter{dbltopnumber}{3}
\setcounter{tocdepth}{1} % 項レベルまで目次に反映させるコマンド.
\renewcommand{\topfraction}{.95}
\renewcommand{\bottomfraction}{.90}
\renewcommand{\textfraction}{.05}
\renewcommand{\floatpagefraction}{.95}

% 使用するパッケージを記述.
\usepackage{amsmath} % 複雑な数式を使うときに便利
\usepackage{dcolumn}
\usepackage{color}
\usepackage{tabularx, dcolumn}
\usepackage{bm} % 数式環境内で太字を使うときに便利.
\usepackage{subcaption}  % 関連した複数の図を並べる時に使う
\usepackage[dvipdfmx]{graphicx} % 画像を挿入したり,テキストや図の拡大縮小・回転を行う.
\usepackage{verbatim} % 入力どおりの出力を行う.
\usepackage{makeidx} % 索引を作成できる.
\usepackage{dcolumn} % 表の数値を小数点で桁を揃える.
\usepackage{lscape} % 図表を90度横に倒して配置する.
\usepackage{setspace} % 行間調整.

\def\mbf#1{\mbox{\boldmath ${#1}$}}

% \newcolumntype{d}{D{+}{\,\pm\,}{4,5}}
% \newcolumntype{i}{D{+}{\,\pm\,}{-1}}
% \newcolumntype{.}{D{.}{.}{6,3}}

\input{../include/begin}

\title{タンパク質の2次構造判定:Dictionary of Protein Secondary Structure (DSSP)}
\maketitle

本章では, タンパク質の2次構造を判定するのによく使われるアルゴリズムである
Dictionary of Protein Secondary Structure (DSSP)\cite{Kabsch1983}を解説する.
DSSPは主鎖のカルボニル基(-CO)とイミノ基(-NH)間の静電相互作用エネルギーから水素結合を判定し,
水素結合のパターンでヘリックスやシートなどの2次構造を判定するアルゴリズムである.


\section{水素結合による構造}

\subsection{水素結合の定義}
DSSP判定ではタンパク質の主鎖のC, O原子上に$(+q_{1},~ -q_{1})$, N, H原子上に$(-q_{2},~ +q_{2})$
の部分電荷を割り振り, それらの静電相互作用によって水素結合を判定する:

\begin{equation}
 E = q_{1} q_{2}
     \left\{
              \frac{1}{r_{\mathrm{ON}}} + \frac{1}{r_{\mathrm{CH}}} 
            - \frac{1}{r_{\mathrm{CH}}} - \frac{1}{r_{\mathrm{CN}}} 
     \right\} * f .
\end{equation}
ここで$q_{1} = 0.42e$, $q_{2} = 0.20e$, $e$は電荷素量, $r_{\mathrm{AB}}$は原子A, B間
の距離(\AA)である. 
また, $f = 332$は無次元の係数であり, $E$の単位はkcal/molである.

静電相互作用$E$が$-0.5 \mathrm{kcal/mol}$よりも小さいとき, $i$番目の残基のC=Oと$j$番目の残基のN-H
間で水素結合が形成されていると定義する:

\begin{equation}
 \mathrm{H bond}[i,~ j]
 \equiv
 [E < -0.5 ~\mathrm{kcal/mol}] .
\end{equation}


\subsection{基本構造: nターン}
\paragraph{ターン構造の定義}
$i$番目のアミノ酸主鎖のCO基と$i+n$番目のアミノ酸主鎖のNH基が水素結合を形成している時, 
$i$番目の残基に$n$ターン構造を割り当てる.
\begin{equation}
 \mathrm{nTurn}[i]
 \equiv
 \mathrm{H bond}[i ,~ i  + n],
 ~~~
 n = 3,~ 4,~ 5.
\end{equation}

% \paragraph{ターン構造の表記}
% ターン構造のパターンが見つかったとき, 水素結合の端を''$>$'', ``$<$''で表し, 括弧の間は''n''で表記する:


\subsection{基本構造:平行/反平行ブリッジ}
平行/反平行ブリッジを判定する際には, 重なりの無い2つの領域
$[i-1,~ i,~ i+1]$, $[j-1,~ j,~ j+1]$を考える.
つまり, $i$番目と$j$番目のアミノ酸は3残基以上離れている必要がある.

\paragraph{平行ブリッジの定義}
i番目とj番目のアミノ酸残基間の平行ブリッジを以下の2つの水素結合パターンで定義する:
\begin{equation}
 \mathrm{Parallel}\mathrm{Bridge}~[i,~j]
 \equiv
 \begin{cases}
  \mathrm{H bond}[i  - 1,~ j]~ \mathrm{and}~ \mathrm{H bond}[j ,~ i  + 1], \mathrm{or}\\
  \mathrm{H bond}[j  - 1,~ i]~ \mathrm{and}~ \mathrm{H bond}[i ,~ j  + 1].
 \end{cases}
\end{equation}

\paragraph{反平行ブリッジの定義}
i番目とj番目のアミノ酸残基間の反平行ブリッジを以下の2つの水素結合パターンで定義する:
\begin{equation}
 \mathrm{Antiparallel}\mathrm{Bridge}~[i,~ j]
 \equiv
 \begin{cases}
  \mathrm{H bond}[i,~ j]~         \mathrm{and}~ \mathrm{H bond}[j,~ i], \mathrm{or}\\
  \mathrm{H bond}[i - 1,~ j + 1]~ \mathrm{and}~ \mathrm{H bond}[j - 1,~ i + 1].
 \end{cases}
\end{equation}


\subsection{協同的構造:ヘリックス}
\paragraph{ヘリックスの定義}
ヘリックスは2つの連続したターンで定義される.
\begin{align}
 \mathrm{3Helix}[i,~i+2]
&\equiv
 [\mathrm{3Turn}[i-1]~ \mathrm{and}~ \mathrm{3Turn}[i]], \\
 \mathrm{4Helix}[i,~i+3]
&\equiv
 [\mathrm{4Turn}[i-1]~ \mathrm{and}~ \mathrm{4Turn}[i]], \\
 \mathrm{5Helix}[i,~i+4]
&\equiv
 [\mathrm{5Turn}[i-1]~ \mathrm{and}~ \mathrm{5Turn}[i]].
\end{align}
例えば4Helixの場合, $\mathrm{Hbond}[i-1,~ i+3]$と$\mathrm{Hbond}[i,~ i+4]$の形成が
4Helixの判定に必要であると言い換えられる. つまり, $i+1$, $i+2$番目の残基の水素結合は必要
としないことに注意されたい.
3Helix, 4Helix, 5Helixはそれぞれ, 一般的に$3_{10}\mathrm{Helix}$, 
$\alpha\mathrm{Helix}$, $\pi\mathrm{Helix}$と呼ばれる.

\subsection{協同的構造:$\beta$ラダー, $\beta$シート構造}
\paragraph{ラダー構造の定義}
ラダー構造はブリッジ構造を基に判定される:
\begin{align}
 \mathrm{ladder}
&\equiv
 \mathrm{''連続した同じ種類のブリッジの集合''} %\\
%\mathrm{''set~of~one~or~more~consecutive~bridges~of~identical~type''} \\
%  \mathrm{sheet}
% &\equiv
%  \mathrm{''残基を共有して結合しているラダーの集合''}
%\mathrm{''set of one or more ladders connected by shared residues''}
\end{align}

\paragraph{シート構造の定義}
シート構造はラダー構造を基に判定される:
\begin{align}
%  \mathrm{ladder}
% &\equiv
%  \mathrm{''連続した同じ種類のブリッジの集合''} \\
%\mathrm{''set~of~one~or~more~consecutive~bridges~of~identical~type''} \\
 \mathrm{sheet}
&\equiv
 \mathrm{''残基を共有して結合しているラダーの集合''}
%\mathrm{''set of one or more ladders connected by shared residues''}
\end{align}

\subsection{2次構造に関連する量: TCO}
\paragraph{TCOの定義}
$i$番目の残基のC=Oと$i-1$番目の残基のC=Oが成す角度を$\theta$とする. 
TCOは, $\cos \theta$で定義される:
\begin{align}
 \mathrm{TCO} (i)
&\equiv 
 \cos \theta \\
&=
 \frac{ \overrightarrow{\mathrm{CO}}_{[i]} \cdot \overrightarrow{\mathrm{CO}}_{[i-1]}}
      {|\overrightarrow{\mathrm{CO}}_{[i]}|     |\overrightarrow{\mathrm{CO}}_{[i-1]}|}.
\end{align}

$\alpha$ヘリックス構造の場合, $\mathrm{TCO} \approx 1$ (つまり, $\theta \approx 0$)となる.
一方, $\beta$シート構造の場合, $\mathrm{TCO} \approx -1$ (つまり, $\theta \approx \pi$)となる.


\section{幾何構造}
\subsection{幾何構造に関連する量: Kappa}
\paragraph{Kaapaの定義}
$i$番目の残基に対するKappaは, $i-2$, $i$, $i+2$番目の残基の3つの$\mathrm{C}_{\alpha}$原子
の結合角で定義される:
\begin{align}
 \mathrm{Kappa}(i)
&\equiv
 \mathrm{Angle}
 \left[
        \left\{ \mathbf{C}_{\alpha}(i)   - \mathbf{C}_{\alpha}(i-2) \right\},~
        \left\{ \mathbf{C}_{\alpha}(i+2) - \mathbf{C}_{\alpha}(i)   \right\}
 \right].
\end{align}
あるいは, 
\begin{align}
 \bm{r}_{1} &= \mathbf{C}_{\alpha}(i-2) - \mathbf{C}_{\alpha}(i), \\
 \bm{r}_{2} &= \mathbf{C}_{\alpha}(i+2) - \mathbf{C}_{\alpha}(i),  \\
 \theta     &= \arccos
                  \left(
                         \frac{\bm{r}_{1} \cdot \bm{r}_{2}}{r_{1} r_{2}}
                  \right),
\end{align}
とすれば,
\begin{align}
 \mathrm{Kappa} = 180.0 - \theta
\end{align}
と計算される.

\subsection{幾何構造に関連する量: Alpha}
\paragraph{Alphaの定義}
$i$番目の残基に対するAlphaは, $i-1$, $i$, $i+1$, $i+2$番目の残基の4つの$\mathrm{C}_{\alpha}$
原子の二面角で定義される:

\begin{align}
 \mathrm{Alpha}(i)
 \equiv
 \mathrm{Dihedral}
 \left\{
         \mathbf{C}_{\alpha}(i-1),~ \mathbf{C}_{\alpha}(i),~
         \mathbf{C}_{\alpha}(i+1),~ \mathbf{C}_{\alpha}(i+2)  
 \right\}.
\end{align}

\subsection{幾何構造に関連する量: 主鎖の二面角 $\phi$}
\paragraph{主鎖の二面角 $\phi$の定義}
$i$番目の残基の主鎖に対する$\phi$は以下のように定義される:

\begin{align}
 \phi (i)
 \equiv
 \mathrm{Dihedral}
 \left\{
         \mathbf{C}(i-1),~        \mathbf{N}(i),~
         \mathbf{C}_{\alpha}(i),~ \mathbf{C}(i)  
 \right\}.
\end{align}

\subsection{幾何構造に関連する量: 主鎖の二面角 $\phi$}
\paragraph{主鎖の二面角 $\psi$の定義}
$i$番目の残基の主鎖に対する$\psi$は以下のように定義される:

\begin{align}
 \psi (i)
 \equiv
 \mathrm{Dihedral}
 \left\{
         \mathbf{N}(i),~ \mathbf{C}_{\alpha}(i),~
         \mathbf{C}(i),~ \mathbf{N}(i+1)  
 \right\}.
\end{align}


\subsection{曲がった構造 (Bend)}
\paragraph{Bend構造の定義}
$\mathrm{Kappa}(i)$が$70^{\circ}$以上の時, $i$番目の残基をBend構造と判定する.
\begin{align}
 \mathrm{Bend}(i)
 \equiv
 [
  \mathrm{Kappa} (i) > 70^{\circ}
 ]
\end{align}

\subsection{キラリティー (Chirality)}
\paragraph{Chiralityの定義}
$i$番目の残基のキラリティーは, $\mathrm{Alpha}(i)$の値を元にして判定する:
\begin{align}
 \mathrm{Chiraliry} (i)
 =
 \left\{
 \begin{array}{ccrl}
  -, & ~~~\textrm{if~} &-180^{\circ}& < \mathrm{Alpha}(i)~ < 0^{\circ} \\
  +, & ~~~\textrm{if~} &   0^{\circ}& < \mathrm{Alpha}(i)~ < 180^{\circ}
 \end{array} 
 \right.
\end{align}

\bibliographystyle{junsrt}
\bibliography{dssp}
\input{../include/end}
