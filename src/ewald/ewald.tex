\documentclass[a4paper, 10.5pt, oneside, openany, uplatex]{jsarticle}

\author{山内 仁喬}
% 余白の設定.
% 参考文献:Latex2e 美文書作成入門, 14.3ページレイアウトの変更

% 行長の変更
\setlength{\textwidth}{40zw}           %全角40文字分

% 行間を制御するコマンド
\renewcommand{\baselinestretch}{0.9}

% 左マージンを変更
\setlength{\oddsidemargin}{25truemm}   % 左余白
\addtolength{\oddsidemargin}{-1truein} % 左位置デフォルトから-1inch

% 上マージンを変更
\setlength{\topmargin}{15truemm}       % 上余白
\addtolength{\topmargin}{-1truein}     % 上位置デフォルトから-1inch

% 本文領域の縦横の長さ変更
\setlength{\textheight}{242truemm}     % テキスト高さ: 297-(25+30)=242mm
\setlength{\textwidth}{160truemm}      % テキスト幅:  210-(25+25)=160mm
\setlength{\fullwidth}{\textwidth}     % ページ全体の幅


% 図・表の個数などの設定.
%% 図・表を入りやすさを制御するパラメーター
\setcounter{topnumber}{4}
\setcounter{bottomnumber}{4}
\setcounter{totalnumber}{4}
\setcounter{dbltopnumber}{3}
\setcounter{tocdepth}{1} % 項レベルまで目次に反映させるコマンド.
\renewcommand{\topfraction}{.95}
\renewcommand{\bottomfraction}{.90}
\renewcommand{\textfraction}{.05}
\renewcommand{\floatpagefraction}{.95}

% 使用するパッケージを記述.
\usepackage{amsmath} % 複雑な数式を使うときに便利
\usepackage{dcolumn}
\usepackage{color}
\usepackage{tabularx, dcolumn}
\usepackage{bm} % 数式環境内で太字を使うときに便利.
\usepackage{subcaption}  % 関連した複数の図を並べる時に使う
\usepackage[dvipdfmx]{graphicx} % 画像を挿入したり,テキストや図の拡大縮小・回転を行う.
\usepackage{verbatim} % 入力どおりの出力を行う.
\usepackage{makeidx} % 索引を作成できる.
\usepackage{dcolumn} % 表の数値を小数点で桁を揃える.
\usepackage{lscape} % 図表を90度横に倒して配置する.
\usepackage{setspace} % 行間調整.

\def\mbf#1{\mbox{\boldmath ${#1}$}}

% \newcolumntype{d}{D{+}{\,\pm\,}{4,5}}
% \newcolumntype{i}{D{+}{\,\pm\,}{-1}}
% \newcolumntype{.}{D{.}{.}{6,3}}

\begin{document}


\title{静電相互作用の計算方法: Ewaldの方法}
\maketitle

静電相互エネルギーは$1/r$に比例し減衰が遅い相互作用であるので, その効果は長距離に及ぶ.
そのため, LJ相互作用のときのようにカットオフ半径を導入すると誤差が大きくなってしまう.
Ewaldの方法\cite{Okazaki2000, Ueda2003}では相互作用の項を実空間と逆空間に分割する.
遠方からの寄与を逆空間上で計算することで, 長距離に起因する相互作用を切断することなく取り入れることができる.
このようにして, 精度よく静電相互作用を計算できる.
しかし, Ewaldの方法では計算量は依然として$\mathcal{O}(N^{2})$であり, 計算コストがかかる.
Particle Mesh Ewald法\cite{Darden1993, Essmann1995}では, 逆空間上の電荷を格子点上に内挿し,
高速フーリエ変換を使用することで, 逆空間における静電相互作用の計算量を$\mathcal{O}(N \log N)$に減少
させる.

\section{Ewaldの方法}
ここでは, $N$個の粒子から構成される系を考える.
各粒子$i$は座標$\bm{r}_{i}$上に部分電荷$q_{i}$を持っているとする.
3つの基本並進ベクトル
$\bm{a}_{1} = (a_{1x}, a_{1y}, a_{1z})^{t}$,
$\bm{a}_{2} = (a_{2x}, a_{2y}, a_{2z})^{t}$,
$\bm{a}_{3} = (a_{3x}, a_{3y}, a_{3z})^{t}$
で張られる平方六面体を基本セルとして, これに応じた周期境界条件を用いることを考える.
このとき, 基本セル中の粒子$j$のイメージセルにおける位置ベクトル$\bm{r}_{j}^{\prime}$は
$\bm{L}=(\bm{a}_{1}~\bm{a}_{2}~\bm{a}_{3})$と,
ある整数の組のベクトル$\bm{n}=(n_{1},n_{2},n_{3})^{\mathrm{t}}$を用いて
\begin{align}
   \bm{r}_{j}^{\prime}
 = \bm{r}_{j} - \bm{Ln}
 = \bm{r}_{j}
  -(n_{1} \bm{a}_{1} +  n_{2}  \bm{a}_{1} + n_{3} \bm{a}_{3})
 \label{eq:Ewald1}
\end{align}
とかけるので, 全静電相互作用のエネルギーは
\begin{align}
    U_{\mathrm{elec}} (\bm{r}_{1}, \bm{r}_{2}, \cdots, \bm{r}_{N})
 &= \frac{1}{2} \sum_{\bm{n}} \sum_{i=1}^{N} \left.\sum_{j=1}^{N}\right.^{\prime}
    \frac{q_{i}q_{j}}{4 \pi \epsilon_{0}}
    \psi( |\bm{r}_{i} - \bm{r}_{j}^{\prime}| )
    \\
 &= \frac{1}{2} \sum_{\bm{n}} \sum_{i=1}^{N} \left.\sum_{j=1}^{N}\right.^{\prime}
    \frac{q_{i}q_{j}}{4 \pi \epsilon_{0}}
    \psi( |\bm{r}_{i} - \bm{r}_{j} + \bm{Ln}| )
 \label{eq:Ewald2.1}
 \\
 \psi(r) &= \frac{1}{r}
 \label{eq:Ewald2.2}
\end{align}
となる.
ここで$\epsilon_{0}$は真空中の誘電率, $\sum^{\prime}$は$\bm{n}=0$の時に
和から$i=j$の場合を除くことを意味する.
以降, 便利のため全原子の座標をまとめて$\bm{r}^{N}$と書く.

$\psi (r)$は減衰の遅い関数であるので,
$\psi (r)$と同程度に減衰の遅い関数$\psi_{0}(r)$を導入することで,
式(\ref{eq:Ewald2.1})を次のように書き換える.
\begin{align}
    U_{\mathrm{elec}} (\bm{r}^{N})
 &= U_{\mathrm{elec}}^{(1)} + U_{\mathrm{elec}}^{(2)} + U_{\mathrm{elec}}^{(3)}
 \label{eq:Ewald3.1}
 \\
    U_{\mathrm{elec}}^{(1)}
 &=
    \frac{1}{2} \sum_{\bm{n}} \sum_{i=1}^{N} \left.\sum_{j=1}^{N}\right.^{\prime}
    \frac{q_{i} q_{j}}{4 \pi \epsilon_{0}}
    \left\{
            \psi( |\bm{r}_{i} - \bm{r}_{j} + \bm{Ln}| )
           -
            \psi_{0}( |\bm{r}_{i} - \bm{r}_{j} + \bm{Ln}| )
    \right\}
 \label{eq:Ewald3.2}
 \\
    U_{\mathrm{elec}}^{(2)}
 &=
    \frac{1}{2} \sum_{\bm{n}} \sum_{i=1}^{N} \sum_{j=1}^{N}
    \frac{q_{i} q_{j}}{4 \pi \epsilon_{0}}
    \psi_{0}( |\bm{r}_{i} - \bm{r}_{j} + \bm{Ln}| )
 \label{eq:Ewald3.3}
 \\
    U_{\mathrm{elec}}^{(3)}
 &=
    -\frac{1}{2} \sum_{i=1}^{N} \frac{q_{i}^2}{4 \pi \epsilon_{0}} \psi_{0}(0)
 \label{eq:Ewald3.4}
\end{align}
ここで, $U_{\mathrm{elec}}^{(1)}$は十分早く0に収束する関数の和となっている.
よってポテンシャルカット$r_{\mathrm{c}}$まで考慮すれば満足できる精度の計算が可能である.
$U_{\mathrm{elec}}^{(1)}$に$\psi_{0}(r)$を導入した
差分が$U_{\mathrm{elec}}^{(2)}$, $U_{\mathrm{elec}}^{(3)}$の項として現れている.
$U_{\mathrm{elec}}^{(2)}$においては, $\bm{n}=0$, $i=j$の場合を考慮する.
その差分が$U_{\mathrm{elec}}^{(3)}$となっている.
$U_{\mathrm{elec}}^{(2)}$は周期的に並んでいる, 同じ構造を持つ粒子系についての和である.
そこで, 実空間で収束の遅い関数を逆空間で計算しようという発想のもとで$U_{\mathrm{elec}}^{(2)}$を
フーリエ級数展開すると,
\begin{equation}
    U_{\mathrm{elec}}^{(2)}
  =
    \frac{1}{2}
    \sum_{\bm{n}} \sum_{i=1}^{N} \sum_{j=1}^{N}
    \frac{1}{(2 \pi)^{3}}
    \frac{q_{i}q_{j}}{4 \pi \epsilon_{0}}
    \int d \bm{k} ~ \hat{\psi}_{0}(\bm{k})
    e^{i \bm{k} \cdot (\bm{r}_{i} - \bm{r}_{j} + \bm{Ln})}
 \label{eq:Ewald4}
\end{equation}
となる. ただし, $\hat{\psi}_{0}(\bm{k})$は$\psi_{0}(\bm{r})$のフーリエ変換
\begin{equation}
    \hat{\psi}_{0}(\bm{k})
  = \int d \bm{r} \psi_{0}(\bm{r}) e^{-i \bm{k} \cdot \bm{r}}
 \label{eq:Ewald5}
\end{equation}
である. ここでポアンカレの和の公式
\begin{equation}
    \sum_{\bm{n}} e^{i \bm{k} \cdot \bm{Ln}}
  =
    \frac{(2\pi)^{3}}{\bm{a}_{1} \cdot \bm{a}_{2} \times \bm{a}_{3}}
    \sum_{\bm{m}} \delta(\bm{k} - 2\pi \bm{m})
 \label{eq:Ewald6}
\end{equation}
を用いると,
\begin{align}
    U_{\mathrm{elec}}^{(2)}
 &=
    \frac{1}{2V} \sum_{i=1}^{N} \sum_{j=1}^{N}
    \frac{q_{i}q_{j}}{4 \pi \epsilon_{0}} \sum_{\bm{m}}
    \int d \bm{k} \hat{\psi}_{0}(\bm{k})
    \delta (\bm{k} - 2 \pi \bm{m})
    e^{i \bm{k} \cdot (\bm{r}_{i} - \bm{r}_{j})}
 \\
 &=
    \frac{1}{2V}
    \sum_{\bm{m}} \hat{\psi}_{0}(2 \pi \bm{\bm{m}})
    \sum_{i=1}^{N} \sum_{j=1}^{N}
    \frac{q_{i}q_{j}}{4 \pi \epsilon_{0}}
    e^{2 \pi i \bm{m} \cdot (\bm{r}_{i} - \bm{r}_{j})}
 \label{eq:Ewald7}
\end{align}
を得る. ただし$\bm{a}_{1} \cdot \bm{a}_{2} \times \bm{a}_{3}=V$と置き直した. 
逆格子ベクトル$\bm{m}$は, ある整数の組$(m_{1},m_{2},m_{3})$を用いて
\begin{align}
   \bm{m}
 =
    m_{1} \bm{a}_{1}^{*} + m_{2} \bm{a}_{2}^{*} + m_{3} \bm{a}_{3}^{*}
 \label{eq:Ewald8}
\end{align}
\begin{align}
   \bm{a}_{1}^{*}
 =
   \frac{\bm{a}_{2} \times \bm{a}_{3}}
        {\bm{a}_{1} \cdot \bm{a}_{2} \times \bm{a}_{3}}
 , ~~
   \bm{a}_{2}^{*}
 =
   \frac{\bm{a}_{3} \times \bm{a}_{1}}
        {\bm{a}_{1} \cdot \bm{a}_{2} \times \bm{a}_{3}}
 , ~~
   \bm{a}_{3}^{*}
 =
   \frac{\bm{a}_{1} \times \bm{a}_{3}}
        {\bm{a}_{1} \cdot \bm{a}_{2} \times \bm{a}_{3}}
 \label{eq:Ewald9}
\end{align}
で与えられる. ここで, $\bm{a}_{\alpha} \cdot \bm{a}_{\beta}^{*}=\delta_{\alpha \beta}$
の関係がある.

以上をまとめると,
\begin{align}
    V_{N} (\bm{r}^{N})
 &=
    U_{\mathrm{elec}}^{(1)} + U_{\mathrm{elec}}^{(2)} + U_{\mathrm{elec}}^{(3)}
 \label{eq:Ewald10.1}
 \\
    U_{\mathrm{elec}}^{(1)}
 &=
    \frac{1}{2} \sum_{\bm{n}} \sum_{i=1}^{N} \left.\sum_{j=1}^{N}\right.^{\prime}
    \frac{q_{i} q_{j}}{4 \pi \epsilon_{0}}
    \left\{
            \psi( |\bm{r}_{i} - \bm{r}_{j} + \bm{Ln}| )
           -
            \psi_{0}( |\bm{r}_{i} - \bm{r}_{j} + \bm{Ln}| )
    \right\}
 \label{eq:Ewald10.2}
 \\
    U_{\mathrm{elec}}^{(2)}
 &=
    \frac{1}{2V} \sum_{\bm{m}} \hat{\psi}_{0}(2 \pi \bm{m})
    \sum_{i=1}^{N} \sum_{j=1}^{N}
    \frac{q_{i}q_{j}}{4 \pi \epsilon_{0}}
    e^{2 \pi i \bm{m} \cdot (\bm{r}_{i} - \bm{r}_{j})}
 \label{eq:Ewald10.3}
 \\
    U_{\mathrm{elec}}^{(3)}
 &=
   -\frac{1}{2} \sum_{i=1}^{N} \frac{q_{i}^2}{4 \pi \epsilon_{0}} \psi_{0}(0)
 \label{eq:Ewald10.4}
\end{align}
となる.

\subsection{$\psi_{0}(r)$の導出}
続いて, $\psi_{0}(r)$の具体的な形を求める.
Ewaldの方法では, 差し引くポテンシャルエネルギーとして, ガウス関数型の電荷分布
\begin{equation}
    \rho_{0}(r)
  =
    q^{\prime} \left(\frac{\alpha^{2}}{\pi}\right)^{\frac{3}{2}} e^{-\alpha^{2} r^{2}}
  \label{eq:Ewald11}
\end{equation}
が作り出す電位$\phi_{0}$の中に電荷$q$が置かれた時のポテンシャルエネルギーを使用する.
そこでポアソン方程式
\begin{equation}
 \nabla^{2} \phi_{0} = - \frac{\rho_{0}}{\epsilon_{0}}
 \label{eq:Ewald12}
\end{equation}
を用いて, 電位$\phi_{0}$を計算する.
$\rho_{0}(r)$が動径方向のみに依存することから, 極座標表示の動径方向のみを考えると, ポアソン方程式は
\begin{equation}
    \frac{\partial^{2} (r \phi_{0})}{\partial r^{2}}
  =
   -\frac{q^{\prime}}{\epsilon_{0}}
    \left( \frac{\alpha^{2}}{\pi} \right)^{\frac{3}{2}} r e^{-\alpha^{2} r^{2}}
\label{eq:Ewald13}
\end{equation}
となる. この式を1回積分し, 適当な定数$a$と積分定数$C_{1}$を導入すると,
\begin{equation}
    \frac{\partial (r \phi_{0})}{\partial r}
  =
   -\frac{q^{\prime}}{\epsilon_{0}} \left( \frac{\alpha^{2}}{\pi} \right)^{\frac{3}{2}}
    \int_{a}^{r} r e^{-\alpha^{2} r^{2}} dr + C_{1}
\label{eq:Ewald14}
\end{equation}
となる. ガウス分布した電荷の作る電位についての境界条件
\begin{equation}
    \left.\frac{\partial (r \phi_{0})}{\partial r} \right|_{r=\infty}
  =
    \phi_{0}(\infty) + r \left.\frac{\partial \phi_{0}}{\partial r} \right|_{r=\infty}
  =
    0
\label{eq:Ewald15}
\end{equation}
であることから, $r \to \infty$とした時に,
\begin{equation}
    C_{1}
  =
    -\frac{q^{\prime}}{\epsilon_{0}} \left(\frac{\alpha^{2}}{\pi}\right)^{\frac{3}{2}}
     \int_{\infty}^{a} r e^{- \alpha^{2} r^{2}} dr1
  \label{eq:Ewald16}
\end{equation}
を得る. したがって
\begin{align}
     \frac{\partial (r \phi_{0})}{\partial r}
 &=
    -\frac{q^{\prime}}{\epsilon_{0}} \left( \frac{\alpha^{2}}{\pi} \right)^{\frac{3}{2}}
     \int_{\infty}^{r} r e^{-\alpha ^{2} r^{2}} dr
 \notag
 \\
 &=
   -\frac{q^{\prime}}{\epsilon_{0}} \left( \frac{\alpha^{2}}{\pi} \right)^{\frac{3}{2}}
    \left[
          -\frac{1}{2\alpha^{2}}e^{-\alpha^2 r^{2}}
    \right]^{r}_{\infty}
 \notag
 \\
 &=
    \frac{q^{\prime}}{\epsilon_{0}} \frac{\alpha}{2 \pi^{\frac{3}{2}}} e^{- \alpha^{2} r^{2}}
 \label{eq:Ewald17}
\end{align}
と計算される. さらにもう1度この式を積分する. 適当な定数$b$と積分定数$C_{2}$を用いると,
\begin{align}
   r \phi_{0}
 =
   \frac{q^{\prime}}{\epsilon_{0}} \frac{\alpha}{2 \pi^{\frac{3}{2}}}
   \int_{b}^{r} e^{- \alpha^{2} r^{2}} dr + C_{2}
 \label{eq:Ewald18}
\end{align}
となる. 境界条件
\begin{equation}
 r \phi_{0}(r) |_{r=0} = 0
 \label{eq:Ewald19}
\end{equation}
より, 積分定数$C_{2}$は
\begin{equation}
    C_{2}
  =
    -\frac{q^{\prime}}{\epsilon_{0}} \frac{\alpha}{2 \pi^{\frac{3}{2}}}
     \int_{b}^{0} e^{- \alpha^{2} r^{2}}
 \label{eq:Ewald20}
\end{equation}
であるので,
\begin{align}
    r \phi_{0}(r)
 &=
    \frac{q^{\prime}}{\epsilon_{0}} \frac{\alpha}{2 \pi^{\frac{3}{2}}}
    \left\{
            \int_{b}^{r} e^{- \alpha^{2} r^{2}} dr
           +\int_{0}^{b} e^{- \alpha^{2} r^{2}} dr
    \right\}
 \notag
 \\
 &=
    \frac{q^{\prime}}{\epsilon_{0}} \frac{\alpha}{2 \pi^{\frac{3}{2}}}
    \int_{0}^{r} e^{- \alpha^{2} r^{2}} dr
 \label{eq:Ewald21}
\end{align}
と計算される. さらに, 積分変数に関して$\alpha r = r^{'}$と変数変換を行うと,
\begin{align}
    r \phi_{0}(r)
 &=
    \frac{q^{\prime}}{4 \pi \epsilon_{0}}
    \frac{2}{\sqrt{\pi}} \int_{0}^{\alpha r} e^{- {r^{\prime}}^{2}} ~dr^{\prime}
 \notag
 \\
 &=
    \frac{q^{\prime}}{4 \pi \epsilon_{0}} \mathrm{erf} (\alpha r)
 \label{eq:Ewald22}
\end{align}
となる. ここで, 関数$\mathrm{erf}(\alpha r)$は誤差関数と呼ばれ,
\begin{equation}
 \mathrm{efr}(x) = \frac{2}{\sqrt{\pi}} \int_{0}^{x} e^{- t^{2}} dt
 \label{eq:Ewald23}
\end{equation}
で定義される. 以上より電位は
\begin{equation}
 \phi_{0}(r) = \frac{q^{\prime}}{4 \pi \epsilon_{0}} \frac{\mathrm{erf(\alpha r)}}{r}
 \label{eq:Ewald24}
\end{equation}
と求まる. ゆえに差し引くポテンシャルエネルギー$\psi_{0}(r)$は
\begin{equation}
 \psi_{0}(r) = \frac{\mathrm{erf}( \alpha r)}{r}
 \label{eq:Ewald25}
\end{equation}
で与えられる.

\subsection{$U_{\mathrm{elec}}^{(1)}$の具体的な形と$\bm{F}_{i}^{(1)}$の導出}
\paragraph{$U_{\mathrm{elec}}^{(1)}$の具体的な形} \

式(\ref{eq:Ewald25})を用いると, $U_{\mathrm{elec}}^{(1)}$は
\begin{align}
    U_{\mathrm{elec}}^{(1)}
 &=
    \frac{1}{2} \sum_{\bm{n}} \sum_{i=1}^{N} \left.\sum_{j=1}^{N}\right.^{\prime}
    \frac{q_{i}q_{j}}{4 \pi \epsilon_{0}}
    \left\{
            \psi( |\bm{r}_{i} - \bm{r}_{j} + \bm{Ln}| )
           -\psi_{0}( |\bm{r}_{i} - \bm{r}_{j} + \bm{Ln}| )
    \right\}
 \notag
 \\
 &=
    \frac{1}{2} \sum_{\bm{n}} \sum_{i=1}^{N} \left.\sum_{j=1}^{N}\right.^{\prime}
    \frac{q_{i}q_{j}}{4 \pi \epsilon_{0}}
    \left\{
            \frac{1}{|\bm{r}_{i} - \bm{r}_{j} + \bm{Ln}|}
           -\frac{\mathrm{erf}(\alpha |\bm{r}_{i} - \bm{r}_{j} + \bm{Ln}| )}
                 {|\bm{r}_{i} - \bm{r}_{j} + \bm{Ln}|}
    \right\}
 \notag
 \\
 &=
    \frac{1}{2} \sum_{\bm{n}} \sum_{i=1}^{N} \left.\sum_{j=1}^{N}\right.^{\prime}
    \frac{q_{i}q_{j}}{4 \pi \epsilon_{0}}
    \frac{\mathrm{erfc}(\alpha |\bm{r}_{i} - \bm{r}_{j} + \bm{Ln}| )}
         {|\bm{r}_{i} - \bm{r}_{j} + \bm{Ln}|}
 \label{eq:Ewald26}
\end{align}
とかける. ここで, 補誤差関数
\begin{equation}
    \mathrm{erfc}(x)
  =
    1 - \mathrm{erf(x)}
  =
  \frac{2}{\sqrt{\pi}} \int^{\infty}_{0} e^{-t^{2}} ~dt
 \label{eq:Ewald27}
\end{equation}
を導入した.

\paragraph{$\bm{F}_{i}^{(1)}$の導出} \

ポテンシャルの位置微分から力$\bm{F}_{i}^{(1)}$が求まる.
$\sum_{i}\sum_{j}^{'}$の和について, $i$に関する微分が2回出てくることに注意して計算すると,
\begin{align}
   \bm{F}_{i}^{(1)}
 =&
   -\frac{d U_{\mathrm{elec}}^{(1)}}{d \bm{r}_{i}}
 \notag
 \\
 =&
   -\sum_{\bm{n}} \left.\sum_{j=1}^{N}\right.^{\prime}
 \frac{q_{i} q_{j}}{4 \pi \epsilon_{0}}
 \Bigg[
        \mathrm{erfc}(\alpha |\bm{r}_{i} - \bm{r}_{j} + \bm{Ln}|)
        \frac{d}{d \bm{r}_{i}}
        \left\{
                \frac{1}{|\bm{r}_{i} - \bm{r}_{j} + \bm{Ln}|}
        \right\}
 \notag
 \\
 &~~~~~~~~~~~~~~~~~~~~~~+
        \frac{1}{|\bm{r}_{i} - \bm{r}_{j} + \bm{Ln}|}
        \frac{d}{d \bm{r}_{i}}
        \mathrm{erfc}(\alpha |\bm{r}_{i} - \bm{r}_{j} + \bm{Ln}|)
  \Bigg]
 \label{eq:Ewald27_1}
\end{align}
となる. 式(\ref{eq:Ewald27_1})の右辺第1項目の微分は
\begin{align}
   \frac{d}{d \bm{r}_{i}}
   \left\{
          \frac{1}{|\bm{r}_{i} - \bm{r}_{j} + \bm{Ln}|}
   \right\}
 =
   \left\{
          \left(\bm{r}_{i} - \bm{r}_{j} + \bm{Ln}\right)^{2}
   \right\}^{-\frac{1}{2}}
 =
   -\frac{\left(\bm{r}_{i} - \bm{r}_{j} + \bm{Ln}\right)}
         {|\bm{r}_{i} - \bm{r}_{j} + \bm{Ln}|^{3}}
\end{align}
と計算される. また, 式(\ref{eq:Ewald27_1})の右辺第2項目の微分は,
\begin{align}
  \frac{d}{d \bm{r}_{i}}
  \mathrm{erfc}(\alpha |\bm{r}_{i} - \bm{r}_{j} + \bm{Ln}|)
 =
  \frac{d}{d \bm{r}_{i}}
  \left\{
         1 - \frac{2}{\sqrt{\pi}}
         \int_{0}^{\alpha |\bm{r}_{i} - \bm{r}_{j} + \bm{Ln}|}
         e^{-t^{2}} dt
 \right\}
\end{align}
である. $x=\alpha |\bm{r}_{i} - \bm{r}_{j} + \bm{Ln}|$と変数変換をすると
\begin{equation}
 \frac{d x}{d \bm{r}_{i}}
  = \frac{\alpha \left(\bm{r}_{i} - \bm{r}_{j} + \bm{Ln}\right)}
         {|\bm{r}_{i} - \bm{r}_{j} + \bm{Ln}|}
  \notag
\end{equation}
であるので,
\begin{align}
    \frac{d}{d \bm{r}_{i}}
    \left\{
           1 - \frac{2}{\sqrt{\pi}}
           \int_{0}^{\alpha |\bm{r}_{i} - \bm{r}_{j} + \bm{Ln}|}
           e^{-t^{2}} dt
 \right\}
 =&
   -\frac{d}{d x}
    \left\{
           \frac{2}{\sqrt{\pi}}
           \int_{0}^{x} e^{-t^{2}} dt
   \right\}
   \frac{d x}{d \bm{r}_{i}}
 \notag
 \\
 =&
   \frac{2}{\sqrt{\pi}} e^{-x^{2}}
   \frac{\alpha \left(\bm{r}_{i} - \bm{r}_{j} + \bm{Ln}\right)}
        {|\bm{r}_{i} - \bm{r}_{j} + \bm{Ln}|}
 \notag
 \\
 =&
   \frac{2\alpha}{\sqrt{\pi}}
   e^{-\alpha^{2} |\bm{r}_{i} - \bm{r}_{j} + \bm{Ln}|^{2}}
   \frac{\bm{r}_{i} - \bm{r}_{j} + \bm{Ln}}
        {|\bm{r}_{i} - \bm{r}_{j} + \bm{Ln}|}
 \notag
\end{align}
と計算される. 以上をまとめると, 力$\bm{F}_{i}^{(1)}$は
\begin{align}
    \bm{F}_{i}^{(1)}
 &=
    \sum_{\bm{n}} \left.\sum_{j=1}^{N}\right.^{\prime}
    \frac{q_{i}q_{j}}{4 \pi \epsilon_{0}}
 \notag
 \\
 &~~\times
    \left[
           \frac{\mathrm{erfc} \{ \alpha |\bm{r}_{i} - \bm{r}_{j} + \bm{Ln}| \} }
                {|\bm{r}_{i} - \bm{r}_{j} + \bm{Ln}|^{2}}
          +\frac{2 \alpha}{\sqrt{\pi}}
           \frac{\exp \{-\alpha^{2} |\bm{r}_{i} - \bm{r}_{j} + \bm{Ln}|^{2} \}}
                {|\bm{r}_{i} - \bm{r}_{j} + \bm{Ln}|}
    \right]
 \frac{ \bm{r}_{i} - \bm{r}_{j} + \bm{Ln} }
      {|\bm{r}_{i} - \bm{r}_{j} + \bm{Ln}|}
\end{align}
から求められる.

\subsection{$U_{\mathrm{elec}}^{(2)}$の具体的な形と$\bm{F}_{i}^{(2)}$の導出}
\paragraph{$U_{\mathrm{elec}}^{(2)}$の具体的な形} \

式(\ref{eq:Ewald10.3})中の$\hat{\psi}_{0}(2 \pi \bm{m})$をポアソン方程式から直接求める.
微分に対するフーリエ変換の公式
\begin{equation}
    \int \nabla^{2} \phi_{0} (\bm{r}) e^{- i \bm{G} \cdot \bm{r}} d \bm{r}
  =
    -\bm{G}^{2} \hat{\phi}_{0}(\bm{G})
 \label{eq:Ewald28}
\end{equation}
にポアソン方程式(\ref{eq:Ewald13})を代入すると,
\begin{equation}
    \hat{\phi}_{0}(2 \pi \bm{m})
  =
    \frac{q^{\prime}}{\epsilon_{0} (2 \pi \bm{m})^{2}}
    \left( \frac{\alpha^{2}}{\pi} \right)^{\frac{3}{2}}
    \int e^{- \alpha^{2} r^{2}} e^{-2 \pi i \bm{m} \cdot \bm{r}} ~d\bm{r}
 \label{eq:Ewald29}
\end{equation}
となる. この積分は次のように3次元のガウス積分を実行することで
\begin{align}
    \hat{\phi}_{0}(2 \pi \bm{m})
 &=
    \frac{q^{\prime}}{4 \pi^{2} \epsilon_{0} \bm{m}^{2}}
    \left( \frac{\alpha^{2}}{\pi} \right)^{\frac{3}{2}}
    \int \exp
    \left[
          -\alpha^{2} \left\{ \bm{r} - \frac{i \pi \bm{m}}{\alpha^{2}} \right\}^{2}
          - \frac{\pi^{2}\bm{m}^{2}}{\alpha^{2}}
    \right] ~d \bm{r}
 \notag
 \\
 &=
    \frac{q^{\prime}}{4 \pi^{2} \epsilon_{0} \bm{m}^{2}}
    \left( \frac{\alpha^{2}}{\pi} \right)^{\frac{3}{2}}
    \exp\left( -\frac{\pi^{2}\bm{m}^{2}}{\alpha^{2}} \right)
    \int \exp
    \left[
          -\alpha^{2} \left\{ \bm{r} - \frac{i \pi \bm{m}}{\alpha^{2}} \right\}^{2}
    \right] ~d \bm{r}
 \notag
 \\
 &=
    \frac{q^{\prime}}{4 \pi ^{2} \epsilon_{0} \bm{m}^{2}}
    \exp\left( -\frac{\pi^{2}\bm{m}^{2}}{\alpha^{2}} \right)
 \label{eq:Ewald30}
\end{align}
と求まる. ゆえに,
\begin{align}
   \hat{\psi}_{0} ( \bm{r} )
 =
   \frac{4 \pi \epsilon_{0}}{q^{\prime}}  \hat{\phi}_{0} ( 2 \pi \bm{m} )
 =
   \frac{1}{\pi \bm{m}^{2}}
   \exp\left(-\frac{\pi^{2}\bm{m}^{2}}{\alpha^{2}}\right)
 \label{eq:Ewald31}
\end{align}
である. 以上から$U_{\mathrm{elec}}^{(2)}$は以下の表式で書くことができる.
\begin{equation}
    U_{\mathrm{elec}}^{(2)}
  =
    \frac{1}{2 \pi V } \sum_{\bm{m}}
    \frac{\exp( -\frac{\pi^{2} |\bm{m}|^{2}}{\alpha^{2}})}{|\bm{m}|^{2}}
    \sum_{i=1}^{N} \sum_{j=1}^{N}
    \frac{q_{i}q_{j} }{4 \pi \epsilon_{0}}
    e^{2 \pi i \bm{m} \cdot (\bm{r}_{i} - \bm{r}_{j})}
 \label{eq:Ewald32}
\end{equation}
ここで注意しなければならないのは, $\bm{m}=0$のとき$U_{\mathrm{elec}}^{(2)}$は$1/|\bm{m}|^{2}$
が原因で発散してしまうことである. そのため, $\bm{m}=0$は無視をしなければならない.
もし$\sum_{i}q_{i}=0$, つまり系が中性であるならば$\bm{m}=0$の項を消すことができる.
Ewald法を使用する際には, 余分な電荷を付け加えるなどをして系を中性化することに気をつけなければならない.

数値計算を行う際には, $U_{\mathrm{elec}}^{(2)}$を三角関数を用いて書き下した形を使う方が便利である.
オイラーの公式を用いて指数関数を
\begin{align}
   e^{2 \pi i \bm{m} \cdot (\bm{r}_{i} - \bm{r}_{j})}
 =
     \cos\left\{ 2 \pi \bm{m} \cdot (\bm{r}_{i} - \bm{r}_{j}) \right\}
 + i \sin\left\{ 2 \pi \bm{m} \cdot (\bm{r}_{i} - \bm{r}_{j}) \right\}
\end{align}
と展開する. $\sum_{\bm{m}}$において$\bm{m} = -\bm{m^{\prime}}$のような
符号の異なる逆格子ベクトルのペアについては$\sin\left\{ 2 \pi \bm{m} \cdot (\bm{r}_{i} - \bm{r}_{j}) \right\}$の項が打ち消しあうため,

\begin{equation}
    U_{\mathrm{elec}}^{(2)}
  =
    \frac{1}{2 \pi V } \sum_{\bm{m}}
    \frac{\exp( -\frac{\pi^{2} |\bm{m}|^{2}}{\alpha^{2}})}{|\bm{m}|^{2}}
    \sum_{i=1}^{N} \sum_{j=1}^{N}
    \frac{q_{i}q_{j} }{4 \pi \epsilon_{0}}
    \cos\left\{ 2 \pi \bm{m} \cdot (\bm{r}_{i} - \bm{r}_{j}) \right\}
 \label{eq:Ewald32_1}
\end{equation}
と表すことができる. 更に, 三角関数の加法定理を使うと

\begin{align}
 &\sum_{i=1}^{N} \sum_{j=1}^{N}
   q_{i} q_{j}
   \cos\left\{ 2 \pi \bm{m} \cdot (\bm{r}_{i} - \bm{r}_{j}) \right\}
 \notag
 \\
 &=
   \sum_{i=1}^{N} \sum_{j=1}^{N}
   q_{i} q_{j}
   \left\{
           \cos\left(2 \pi \bm{m} \cdot \bm{r}_{i}\right)
           \cos\left(2 \pi \bm{m} \cdot \bm{r}_{j}\right)
         +
           \sin\left(2 \pi \bm{m} \cdot \bm{r}_{i}\right)
           \sin\left(2 \pi \bm{m} \cdot \bm{r}_{j}\right)
 \right\}
 \notag
 \\
 &=
          \sum_{i=1}^{N} q_{i} \cos\left(2 \pi \bm{m} \cdot \bm{r}_{i}\right)
          \sum_{j=1}^{N} q_{j} \cos\left(2 \pi \bm{m} \cdot \bm{r}_{j}\right)
        +
          \sum_{i=1}^{N} q_{i} \sin\left(2 \pi \bm{m} \cdot \bm{r}_{i}\right)
          \sum_{j=1}^{N} q_{j} \sin\left(2 \pi \bm{m} \cdot \bm{r}_{j}\right)
 \notag
 \\
 &=
  \left\{
          \sum_{i=1}^{N} q_{i} \cos\left(2 \pi \bm{m} \cdot \bm{r}_{i}\right)
  \right\}^{2}
 +
  \left\{
          \sum_{i=1}^{N} q_{i} \sin\left(2 \pi \bm{m} \cdot \bm{r}_{i}\right)
  \right\}^{2}
\end{align}
であるので,

\begin{align}
     U_{\mathrm{elec}}^{(2)}
  =&
    \frac{1}{2 \pi V (4 \pi \epsilon_{0})}
    \sum_{\bm{m}}
    \frac{\exp( -\frac{\pi^{2} |\bm{m}|^{2}}{\alpha^{2}})}{|\bm{m}|^{2}}
 \notag
 \\
 &\times
    \left[
    \left\{
           \sum_{i=1}^{N} q_{i} \cos\left(2 \pi \bm{m} \cdot \bm{r}_{i}\right)
    \right\}^{2}
  +
    \left\{
           \sum_{i=1}^{N} q_{i} \sin\left(2 \pi \bm{m} \cdot \bm{r}_{i}\right)
    \right\}^{2}
    \right]
 \label{eq:Ewald32_2}
\end{align}
と書き下せる. この形式では, あらかじめ
$\sum q_{i} \cos(2\pi\bm{m}\cdot\bm{r}_{i})$と
$\sum q_{i} \sin(2\pi\bm{m}\cdot\bm{r}_{i})$
を計算しておくことができるため, 計算コストを削減することができる.

\paragraph{$\bm{F}_{i}^{(2)}$の導出} \

式(\ref{eq:Ewald32})を座標で微分する.
$\sum_{i}\sum_{j}^{\prime}$に$i$に関する微分が2回出てくることに注意すると,
\begin{align}
 \bm{F}_{i}^{(2)}
 =&
   \frac{d U_{\mathrm{elec}}^{(2)}}{d \bm{r}_{i}}
 \notag
 \\
 =&
   -\frac{2}{2 \pi V} \sum_{\bm{m}}
    \frac{\exp( -\frac{\pi^{2} |\bm{m}|^{2}}{\alpha^{2}})}{|\bm{m}|^{2}}
    \sum_{j=1}^{N}
    \frac{q_{i}q_{j} }{4 \pi \epsilon_{0}}
    (2 \pi i \bm{m})
    e^{2 \pi i \bm{m} \cdot (\bm{r}_{i} - \bm{r}_{j})}
 \notag
 \\
  =&
    \frac{2}{V}
    \sum_{\bm{m}} \frac{i \bm{m}}{|\bm{m}|^{2}}
    \exp\left(-\frac{\pi^{2} |\bm{m}|^{2}}{\alpha^{2}}\right)
    \sum_{j=1}^{N}
    \frac{q_{i}q_{j} }{4 \pi \epsilon_{0}}
    e^{2 \pi i \bm{m} \cdot (\bm{r}_{i} - \bm{r}_{j})}
\end{align}
と求められる. 更に, オイラーの公式を用いて指数関数を展開すると,
\begin{equation}
  \bm{F}_{i}^{(2)}
 =
  \frac{2}{V}
  \sum_{\bm{m}} \frac{i \bm{m}}{|\bm{m}|^{2}}
  \exp\left(-\frac{\pi^{2} |\bm{m}|^{2}}{\alpha^{2}}\right)
  \sum_{j=1}^{N}
  \frac{q_{i}q_{j} }{4 \pi \epsilon_{0}}
  \sin \left\{2 \pi i \bm{m} \cdot (\bm{r}_{i} - \bm{r}_{j}) \right\}
\end{equation}
と変形できる. これは式(\ref{eq:Ewald32_1})を微分した形と同じである.
ここで, 三角関数の加法定理
\begin{align}
  \sum_{j=1}^{N} q_{i} q_{j}
 &
  \sin\left\{ 2 \pi \bm{m} \cdot (\bm{r}_{i} - \bm{r}_{j}) \right\}
 \notag
 \\
=&
   q_{i} \sin(2 \pi \bm{m} \cdot \bm{r}_{i})
   \sum_{j=1}^{N} q_{j} \cos(2 \pi \bm{m} \cdot \bm{r}_{j})
  -
   q_{i} \cos(2 \pi \bm{m} \cdot \bm{r}_{i})
   \sum_{j=1}^{N} q_{j} \sin(2 \pi \bm{m} \cdot \bm{r}_{j})
\end{align}
を用いることで,
\begin{align}
  \bm{F}_{i}^{(2)}
=&
  \frac{2}{V(4 \pi \epsilon_{0})}
  \sum_{\bm{m}} \frac{i \bm{m}}{|\bm{m}|^{2}}
  \exp\left(-\frac{\pi^{2} |\bm{m}|^{2}}{\alpha^{2}}\right)
 \notag
 \\
 \times&
 \left[
   q_{i} \sin(2 \pi \bm{m} \cdot \bm{r}_{i})
   \sum_{j=1}^{N} q_{j} \cos(2 \pi \bm{m} \cdot \bm{r}_{j})
  -
   q_{i} \cos(2 \pi \bm{m} \cdot \bm{r}_{i})
   \sum_{j=1}^{N} q_{j} \sin(2 \pi \bm{m} \cdot \bm{r}_{j})
 \right]
\end{align}
と変形することができる.

\subsection{$U_{\mathrm{elec}}^{(3)}$の具体的な形と$\bm{F}_{i}^{(3)}$の導出}
\paragraph{$U_{\mathrm{elec}}^{(3)}$の具体的な形} \

誤差関数のべき級数展開は
\begin{equation}
    \mathrm{erf}(x)
  =
    \frac{2}{\sqrt{\pi}} \sum_{n=0}^{\infty}
    \frac{(-1)^{n} x^{2n+1}}{n! (2n+1)}
  =
    \frac{2}{\sqrt{\pi}}\left( x - \frac{x^{3}}{3} + \cdots\right)
 \label{eq:Ewald33}
\end{equation}
であるので,
\begin{align}
   \psi_{0}(0)
 =
   \left.\frac{\mathrm{erf(x)}}{r} \right|_{r=0}
 =
   \left.\frac{2}{\sqrt{\pi}} \frac{\alpha r - \frac{1}{3}(\alpha r)^{3} + \cdots}{r} \right|_{r=0}
 =
   \frac{2 \alpha}{\sqrt{\pi}}
 \label{eq:Ewald34}
\end{align}
と計算される. これを式(\ref{eq:Ewald10.4})に代入することで, $U_{\mathrm{elec}}^{(3)}$の表式
\begin{equation}
    U_{\mathrm{elec}}^{(3)}
  =
   -\frac{\alpha}{\sqrt{\pi}} \sum_{i=1}^{N} \frac{q_{i}^{2}}{4 \pi \epsilon_{0}}
 \label{eq:Ewald35}
\end{equation}
が求まる.

\paragraph{$\bm{F}_{i}^{(3)}$の導出} \

$U_{\mathrm{elec}}^{(3)}$が座標に依存しないことから
\begin{equation}
 \bm{F}_{i}^{(3)} = 0
\end{equation}
となる.


% \subsection{力の計算}
% ポテンシャルの位置微分から力$\bm{F}_{i}$が求まる.
% $\sum_{i}\sum_{j}^{'}$の和について, $i$に関する微分が2回出てくることに注意して計算すると,
% \begin{equation}
%     \bm{F}_{i}(\bm{r}^{N})
%   =
%     \bm{F}_{i}^{(1)} + \bm{F}_{i}^{(2)} + \bm{F}_{i}^{(3)}
%  \label{eq:Ewald36}
% \end{equation}
% \begin{align}
%     \bm{F}_{i}^{(1)}
%  &=
%     \sum_{\bm{n}} \sum_{j}^{\prime}
%     \frac{q_{i}q_{j}}{4 \pi \epsilon_{0}}
%  \notag
%  \\
%  &~~\times
%     \left[
%            \frac{\mathrm{erfc} \{ \alpha |\bm{r}_{i} - \bm{r}_{j} + \bm{Ln}| \} }
%                 {|\bm{r}_{i} - \bm{r}_{j} + \bm{Ln}|^{2}}
%           +\frac{2 \alpha}{\sqrt{\pi}}
%            \frac{-\exp \{ \alpha^{2} |\bm{r}_{i} - \bm{r}_{j} + \bm{Ln}|^{2} \}}
%                 {|\bm{r}_{i} - \bm{r}_{j} + \bm{Ln}|}
%     \right]
%  \frac{ \bm{r}_{i} - \bm{r}_{j} + \bm{Ln} }
%       {|\bm{r}_{i} - \bm{r}_{j} + \bm{Ln}|}
%  \\
%     \bm{F}_{i}^{(2)}
%  &=
%    -\frac{2}{4 \pi \epsilon_{0} V}
%     \sum_{\bm{m}} \frac{i \bm{m}}{|\bm{m}|^{2}}
%     \exp \left( -\frac{\pi^{2}|\bm{m}|^{2} }{ \alpha^{2} } \right)
%     \sum_{j} q_{i}q_{j} e^{2 \pi i \bm{m} \cdot (\bm{r}_{i} - \bm{r}_{j})}
%  \\
%     \bm{F}_{i}^{(3)}
%  &=
%     0
%  \label{eq:Ewald37}
% \end{align}
% と求まる.

\subsection{Ewald法による静電ポテンシャル項と力の計算のまとめ}
以上で導出してきた結果をまとめる.
\begin{align}
    U_{\mathrm{elec}}^{(1)}
 =&
    \frac{1}{2} \sum_{\bm{n}} \sum_{i=1}^{N} \left.\sum_{j=1}^{N}\right.^{\prime}
    \frac{q_{i}q_{j}}{4 \pi \epsilon_{0}}
    \frac{\mathrm{erfc}(\alpha |\bm{r}_{i} - \bm{r}_{j} + \bm{Ln}| )}
         {|\bm{r}_{i} - \bm{r}_{j} + \bm{Ln}|}
 %\label{eq:Ewald38.1}
 \notag
 \\
 \notag
 \\
    U_{\mathrm{elec}}^{(2)}
 =&
    \frac{1}{2 \pi V} \sum_{\bm{m}}
    \frac{\exp(\frac{- \pi^{2}|\bm{m}|^{2}}{\alpha^{2}})}{|\bm{m}|^{2}}
    \sum_{i=1}^{N} \sum_{j=1}^{N}
    \frac{q_{i}q_{j}}{4 \pi \epsilon_{0}}
     e^{2 \pi i \bm{m} \cdot (\bm{r}_{i} - \bm{r}_{j})}
 %\label{eq:Ewald38.2}
 \notag
 \\
 =&
    \frac{1}{2 \pi V } \sum_{\bm{m}}
    \frac{\exp( -\frac{\pi^{2} |\bm{m}|^{2}}{\alpha^{2}})}{|\bm{m}|^{2}}
    \sum_{i=1}^{N} \sum_{j=1}^{N}
    \frac{q_{i}q_{j} }{4 \pi \epsilon_{0}}
    \cos\left\{ 2 \pi \bm{m} \cdot (\bm{r}_{i} - \bm{r}_{j}) \right\}
 \notag
 \\
 =&\frac{1}{2 \pi V (4 \pi \epsilon_{0})}
    \sum_{\bm{m}}
    \frac{\exp( -\frac{\pi^{2} |\bm{m}|^{2}}{\alpha^{2}})}{|\bm{m}|^{2}}
    \left[
    \left\{
           \sum_{i=1}^{N} q_{i} \cos\left(2 \pi \bm{m} \cdot \bm{r}_{i}\right)
    \right\}^{2}
  +
    \left\{
           \sum_{i=1}^{N} q_{i} \sin\left(2 \pi \bm{m} \cdot \bm{r}_{i}\right)
    \right\}^{2}
    \right]
 \notag
 \\
 \notag
 \\
    U_{\mathrm{elec}}^{(3)}
 =&
   -\frac{\alpha}{\sqrt{\pi}} \sum_{i=1}^{N}\frac{q_{i}^{2}}{4 \pi \epsilon_{0}}
 %\label{eq:Ewald38.3}
 \notag
 \\
 \notag
 \\
    \bm{F}_{i}^{(1)}
 =&
    \sum_{\bm{n}} \left. \sum_{j=1}^{N} \right.^{\prime}
    \frac{q_{i}q_{j}}{4 \pi \epsilon_{0}}
 \notag
 \\
 &~~\times
    \left[
           \frac{\mathrm{erfc} \{ \alpha |\bm{r}_{i} - \bm{r}_{j} + \bm{Ln}| \} }
                {|\bm{r}_{i} - \bm{r}_{j} + \bm{Ln}|^{2}}
          +\frac{2 \alpha}{\sqrt{\pi}}
           \frac{\exp \{ -\alpha^{2} |\bm{r}_{i} - \bm{r}_{j} + \bm{Ln}|^{2} \}}
                {|\bm{r}_{i} - \bm{r}_{j} + \bm{Ln}|}
    \right]
    \frac{ \bm{r}_{i} - \bm{r}_{j} + \bm{Ln} }
         {|\bm{r}_{i} - \bm{r}_{j} + \bm{Ln}|}
 %\label{eq:Ewald38.4}
 \notag
 \\
 \notag
 \\
    \bm{F}_{i}^{(2)}
 =&
   -\frac{2}{ V}
    \sum_{\bm{m}} \frac{i \bm{m}}{|\bm{m}|^{2}}
    \exp \left( -\frac{ \pi^{2}|\bm{m}|^{2} }{ \alpha^{2} } \right)
    \sum_{j=1}^{N}
    \frac{q_{i}q_{j}}{4 \pi \epsilon_{0}}
    e^{2 \pi i \bm{m} \cdot (\bm{r}_{i} - \bm{r}_{j})}
 %\label{eq:Ewald38.5}
 \notag
 \\
 \notag
 \\
    \bm{F}_{i}^{(3)}
 =&
    0
 %\label{eq:Ewald38.6}
 \notag
\end{align}

\clearpage
\section{Particle Mesh Ewald(PME)法}
\subsection{PMEの数学的基礎:離散フーリエ変換}
$K_{1}$,$K_{2}$,$K_{3}$を正の整数とする.
$0 \le k_{\alpha} < K_{\alpha}$について複素数の値を持つ配列$A(k_{1}, k_{2}, k_{3})$を考える.
この時, 離散フーリエ変換$\mathcal{F}$と逆離散フーリエ変換$\mathcal{F}^{-1}$は
\begin{alignat}{2}
 &   \mathcal{F}(A)(m_{1},m_{2},m_{3})
 &&=
     \sum_{k_{1}=0}^{K_{1}-1} \sum_{k_{2}=0}^{K_{2}-1} \sum_{k_{3}=0}^{K_{3}-1}
     A(k_{1}, k_{2}, k_{3})
 \notag
 \\
 & &&
 ~~~~ \times
 \exp
 \left[ 2\pi i
        \left(
               \frac{m_{1} k_{1}}{K_{1}} + \frac{m_{2} k_{2}}{K_{2}} + \frac{m_{3} k_{3}}{K_{3}}
        \right)
 \right]
 \label{eq:PME1}
 \\
 &   \mathcal{F}^{-1} (A)(m_{1},m_{2},m_{3})
 &&=
     \frac{1}{K_{1} K_{2} K_{3}}
     \sum_{l_{1}=0}^{K_{1}-1} \sum_{l_{2}=0}^{K_{2}-1} \sum_{l_{3}=0}^{K_{3}-1}
     A(l_{1}, l_{2}, l_{3})
 \notag
 \\
 & &&
 ~~~~ \times
 \exp
 \left[ -2\pi i
        \left(
               \frac{m_{1} l_{1}}{K_{1}} + \frac{m_{2} l_{2}}{K_{2}} + \frac{m_{3} l_{3}}{K_{3}}
        \right)
 \right]
 \label{eq:PME2}
\end{alignat}
とかかれる. 定義より,
\begin{equation}
    \mathcal{F}^{-1} \left[ \mathcal{F} (A) \right]
  =
    \mathcal{F} \left[ \mathcal{F}^{-1} (A) \right]
  =
   A
 \label{eq:PME3}
\end{equation}
が成立する.
数値計算の便利のため, 以下の表記を導入する.
\begin{equation}
    \hat{\mathcal{F}}^{-1}(A)
  =
    K_{1} K_{2} K_{3} \mathcal{F}^{-1} (A)(m_{1},m_{2},m_{3})
 \label{eq:PME4}
\end{equation}
フーリエ変換について以下の恒等式が成立する.
\begin{align}
   A * B
 = \mathcal{F} \left[ \mathcal{F}^{-1} (A*B) \right]
 = K_{1} K_{2} K_{3} \mathcal{F} \left[ \mathcal{F}^{-1} (A) \cdot \mathcal{F}^{-1} (B) \right]
 \label{eq:PME5}
\end{align}
ここで, $A*B$は畳み込みであり,
\begin{align}
   A*B(j_{1}, j_{2}, j_{3})
 =
   \sum_{k_{1}=0}^{K_{1}-1} \sum_{k_{2}=0}^{K_{2}-1} \sum_{k_{3}=0}^{K_{3}-1}
   A(j_{1}-k_{1}, j_{2}-k_{2}, j_{3} - k_{3})
   \cdot
   B(k_{1}, k_{2}, k_{3})
 \label{eq:PME6}
\end{align}
と計算される.

\subsection{PMEの数学的基礎:B-spline関数}
任意の実数$u$に対して, 0次のB-spline関数を
\begin{align}
  M_{1}^{l} (u)
=
  \begin{cases}
   1,~  & u \in (l,~ l+1] \\
   0,~  & u \notin (l,~ l+1] \\
  \end{cases}
  \label{eq:PME7}
\end{align}
と定義する. ここで, $l$は0以上の整数とする.
2よりも大きい自然数$n$に対して, $n-1$次のB-spline関数$M_{n}^{l}(u)$は次の漸化式で定義される.
\begin{align}
  M_{n}^{l}(u)
&=
  \frac{u-l}{n-1} M_{n-1}^{l} (u) + \frac{-u+n+l}{n-1} M_{n-1}^{l+1} (u)
  \label{eq:PME8}
  \\
&=
  \frac{u-l}{n-1} M_{n-1}^{l} (u) + \frac{-u+n+l}{n-1} M_{n-1}^{l} (u-1)
\end{align}
B-spline関数は以下の性質を持つ.
\begin{enumerate}
 \item $l \le u \le l+n$で$M_{n}^{l}(u)>0$である. $u<l,~ l+n<u$では$M_{n}^{l}(u)=0$となる.
 \item $M_{n}^{l}(u)=M_{n}^{l}(l+n-u)$である.
 \item $\sum_{j=-\infty}^{\infty} M_{n}^{l}(u-j)=1$である.
\end{enumerate}
また
$n>2$に対するB-spline関数の微分は,
\begin{align}
   \frac{d}{du}M_{n}^{l}(u)
 &=
    M_{n-1}^{l}(u) - M_{n-1}^{l+1}(u)
 \label{eq:PME9}
 \\
 &=
    M_{n-1}^{l}(u) - M_{n-1}^{l}(u-1)
\end{align}
のように, 解析的に求めることができる.
この解析微分の式は式(\ref{eq:PME8})を使えば, 数学的帰納法により示すことができる.

\subsection{PME法}
PME法では$U_{\mathrm{elec}}^{(2)}$を以下のように書き直す.
\begin{align}
    U_{\mathrm{elec}}^{(2)}
 &=
    \frac{1}{2 \pi V}
    \sum_{\bm{m}} \frac {\exp( -\pi^{2}|\bm{m}|^{2} / \alpha^{2} ) }{|\bm{m}|^{2}}
    \sum_{i=1}^{N}\sum_{j=1}^{N}
    \frac{q_{i} q_{j}}{4 \pi \epsilon_{0}}
    e^{2 \pi i \bm{m} \cdot (\bm{r}_{i} - \bm{r}_{j})}
 \notag
 \\
 &=
    \frac{1}{2 \pi V (4 \pi \epsilon_{0})}
    \sum_{\bm{m}}
    \frac {\exp( -\pi^{2}|\bm{m}|^{2} / \alpha^{2} ) }{|\bm{m}|^{2}}
    S(\bm{m}) S(-\bm{m})
 \label{eq:PME10}
\end{align}
ここで,
\begin{alignat}{2}
     S(\bm{m})
 &\equiv
     S(m_{1},m_{2},m_{3})
 &&=
     \sum_{i=1}^{N} q_{i} e^{2 \pi \bm{m} \cdot \bm{r}_{i}}
 \label{eq:PME12}
\end{alignat}
を定義した. $S(\bm{m})$は構造因子と呼ばれ,
電荷密度$\rho(\bm{r})=\sum_{j=1}^{N}q_{j}\delta(\bm{r}-\bm{r}_{j})$の
フーリエ変換に相当する.
さらに, 座標$\bm{r}_{j}$に位置する点電荷$q_{j}$について, 正の整数$K_{1},K_{2},K_{3}$を用いて逆格子空間における座標
\begin{align}
 &u_{\alpha j} = K_{\alpha} \bm{a}_{\alpha}^{*} \cdot \bm{r}_{j} &\text{for $\alpha=1,2,3$}
 \label{eq:PME13}
\end{align}
を定義する. 周期境界条件のため$u_{\alpha j}$の定義域は$0 \le u_{\alpha j} < K_{\alpha}$となる.
この逆格子空間上における座標を用いると, 構造因子は
\begin{align}
    S(\bm{m})
 &=
    \sum_{j=1}^{N} q_{j} e^{2 \pi i \bm{m} \cdot \bm{r}_{j}}
 \notag
 \\
 &=
    \sum_{j=1}^{N} q_{j}
    e^{2 \pi i (m_{1}\bm{a}_{1}^{*} + m_{2}\bm{a}_{2}^{*} + m_{3}\bm{a}_{3}^{*})
       \cdot \bm{r}_{j}}
 \notag
 \\
 &=
    \sum_{j=1}^{N} q_{j}
    \exp\left(2 \pi i \frac{m_{1} u_{1j}}{K_{1}}\right)
    \exp\left(2 \pi i \frac{m_{2} u_{2j}}{K_{2}}\right)
    \exp\left(2 \pi i \frac{m_{3} u_{3j}}{K_{3}}\right)
 \label{eq:PME14}
\end{align}
と書き直せる.
$n$が偶数であれば, B-spline関数$M_{n}^{0}(u)$を用いて
\begin{align}
 &\exp \left(2 \pi i \frac{m_{\alpha}}{K_{\alpha}} u_{\alpha} \right)
 \approx
     b_{\alpha} (m_{\alpha}) \sum_{k=-\infty}^{\infty} M_{n}^{0}(u_{\alpha} - k )
     \exp \left(2 \pi i \frac{m_{\alpha}}{K_{\alpha}} k \right)
 \label{eq:PME15}
 \\
 &b_{\alpha} (m_{\alpha})
 =
   \exp \left\{2 \pi i (n-1) \frac{m_{\alpha}}{K_{\alpha}} \right\}
   \left[
         \sum_{k=0}^{n-2} M_{n}^{0}(k+1) \exp \left(2 \pi i \frac{m_{\alpha}}{K_{\alpha}} k \right)
    \right]^{-1}
 \label{eq:PME16}
\end{align}
と展開することができる. この展開を用いると構造因子$S(\bm{m})$は
\begin{align}
    S(\bm{m})
 &=
    \sum_{j=1}^{N} q_{j} b_{1}(m_{1}) b_{2}(m_{2}) b_{3}(m_{3})
 \notag
 \\
 &~~ \times
    \sum_{k_{1}=-\infty}^{\infty} \sum_{k_{2}=-\infty}^{\infty} \sum_{k_{3}=-\infty}^{\infty}
    M_{n}^{0}(u_{1j}- k_{1}) M_{n}^{0}(u_{2j} - k_{2}) M_{n}^{0}(u_{3j} - k_{3})
 \notag
 \\
 &~~ \times
    \exp\left( 2 \pi i \frac{m_{1}}{K_{1}} k_{1} \right)
    \exp\left( 2 \pi i \frac{m_{2}}{K_{2}} k_{2} \right)
    \exp\left( 2 \pi i \frac{m_{3}}{K_{3}} k_{3} \right)
 \notag
 \\
 &=
    \sum_{j=1}^{N} q_{j} b_{1}(m_{1}) b_{2}(m_{2}) b_{3}(m_{3})
    \sum_{k_{1}=0}^{K_{1}-1} \sum_{k_{2}=0}^{K_{2}-1} \sum_{k_{3}=0}^{K_{3}-1} \sum_{n_{1},n_{2},n_{3}}
 \notag
 \\
 &~~~~ \times
    M_{n}^{0}(u_{1j} - k_{1} - n_{1}K_{1})
    M_{n}^{0}(u_{2j} - k_{2} - n_{2}K_{2})
    M_{n}^{0}(u_{3j} - k_{3} - n_{3}K_{3})
 \notag
 \\
 &~~~~ \times
    \exp\left( 2 \pi i \frac{m_{1}}{K_{1}} k_{1} \right)
    \exp\left( 2 \pi i \frac{m_{2}}{K_{2}} k_{2} \right)
    \exp\left( 2 \pi i \frac{m_{3}}{K_{3}} k_{3} \right)
 \label{eq:PME17}
\end{align}
と変形できる. 逆空間に内挿された点電荷を
\begin{align}
    Q(\bm{k})
 \equiv
    Q(k_{1}, k_{2}, k_{3})
 &=
    \sum_{j=1}^{N} q_{j} \sum_{n_{1}, n_{2}, n_{3}}
    M_{n}^{0}(u_{1j}- k_{1} - n_{1}K_{1})
 \notag
 \\ &~~~~ \times
    M_{n}^{0}(u_{2j} - k_{2} - n_{2}K_{2}) M_{n}^{0}(u_{3j} - k_{3} - n_{3}K_{3})
 \label{eq:PME18}
 \end{align}
と定義すると,
\begin{align}
    S(\bm{m})
 &=
    b_{1}(m_{1}) b_{2}(m_{2}) b_{3}(m_{3})
    \sum_{k_{1}=0}^{K_{1}-1}  \sum_{k_{2}=0}^{K_{2}-1}  \sum_{k_{3}=0}^{K_{3}-1}
 \notag
 \\
 &~~~~ \times
    Q(\bm{k})
    \exp\left( 2 \pi i \frac{m_{1}}{K_{1}} k_{1} \right)
    \exp\left( 2 \pi i \frac{m_{2}}{K_{2}} k_{2} \right)
    \exp\left( 2 \pi i \frac{m_{3}}{K_{3}} k_{3} \right)
 \notag
 \\
 &=
    b_{1}(m_{1}) b_{2}(m_{2}) b_{3}(m_{3})
    \mathcal{F}(Q) (\bm{m})
 \label{eq:PME19}
\end{align}
のように離散フーリエ変換$\mathcal{F}$を用いて構造因子$S(\bm{m})$を表すことができる.
さらに, $b_{\alpha}(m_{\alpha})$と$b_{\alpha}(-m_{\alpha})$が複素共役であることを用いて, 
\begin{align}
   B(\bm{m})
 \equiv
   B(m_{1}, m_{2}, m_{3})
 =
   |b_{1}(m_{1})|^{2} |b_{2}(m_{2})|^{2} |b_{3}(m_{3})|^{2}
 \label{eq:PME21}
\end{align}
を定義すると,
\begin{align}
   S(\bm{m})S(-\bm{m})
&=
   B(\bm{m}) \mathcal{F}(Q)(\bm{m}) \mathcal{F}(Q)(-\bm{m})
 \notag \\
&=
   B(\bm{m}) |\mathcal{F}(Q)(\bm{m})|^{2}
 \notag \\
&=
   B(\bm{m}) |\hat{\mathcal{F}}^{-1}(Q)(\bm{m})|^{2}
 \notag
\end{align}
である.
第1式から第2式の変形おいて, $F(Q)(\bm{m})$と$F(Q)(-\bm{m})$
が複素共役であることを使用した.
すると, 式(\ref{eq:PME10})は
\begin{align}
   U_{\mathrm{elec}}^{(2)}
 &=
   \frac{1}{2 \pi V (4 \pi \epsilon_{0})} \sum_{\bm{m}}
   \frac {\exp( - \pi^{2}|\bm{m}|^{2} / \alpha^{2} ) }{|\bm{m}|^{2}}
   B(\bm{m}) S(\bm{m})S(-\bm{m})
 % \notag
 % \\
 % &\equiv
 %   \frac{1}{2 \pi V (4 \pi \epsilon_{0})}
 %   \sum_{m_{1}=0}^{K_{1}-1} \sum_{m_{2}=0}^{K_{2}-1} \sum_{m_{3}=0}^{K_{3}-1}
 %   C(\bm{m}) B(\bm{m})
 %   \mathcal{F}(Q) (\bm{m}) \mathcal{F}(Q) (\bm{-m})
 % \notag
 % \\
 % &=
 %    \frac{1}{2 \pi V (4 \pi \epsilon_{0})}
 %    \sum_{m_{1}=0}^{K_{1}-1} \sum_{m_{2}=0}^{K_{2}-1} \sum_{m_{3}=0}^{K_{3}-1}
 %    C(\bm{m}) B(\bm{m})
 %    | \mathcal{F}(Q) (\bm{m}) |^{2}
 \notag
 \\
 &=
    \frac{1}{2 \pi V (4 \pi \epsilon_{0})}
    \sum_{m_{1}=0}^{K_{1}-1} \sum_{m_{2}=0}^{K_{2}-1} \sum_{m_{3}=0}^{K_{3}-1}
    C(\bm{m}) B(\bm{m})
    | \hat{\mathcal{F}}^{-1}(Q) (\bm{m}) |^{2}
 \label{eq:PME22}
\end{align}
と計算される. ただし,
\begin{align}
  &C(\bm{m})
 = C(m_{1}, m_{2}, m_{3})
= \frac {\exp( - \pi^{2}|\bm{m}|^{2} / \alpha^{2} ) }{|\bm{m}|^{2}}
 ~~~~\textrm{for $\bm{m} \neq 0$}, ~~C(0,0,0) = 0
 \label{eq:PME11}
\end{align}
を定義した. ただし,
$\bm{m} \equiv m_{1}^{\prime}\bm{a}_{1}^{*} + m_{2}^{\prime} \bm{a}_{2}^{*}+ m_{3}^{\prime}\bm{a}_{3}^{*}$であり, $m_{i}^{\prime}$は$0 \le m_{i} \le K_{i}/2$の範囲で$m_{i}^{\prime} = m_{i}$, その他の範囲では$m_{i}^{\prime} = m_{k} - K_{i}$となる.

続いて力の表式を求める. 座標$\bm{r}_{j}$で微分すると,
\begin{align}
    \bm{F}_{j}^{(2)}
 &=
    -\frac{1}{2 \pi V (4 \pi \epsilon_{0})} \sum_{\bm{m}} C(\bm{m}) B(\bm{m})
 \notag
 \\
 &~~~~~~~~~~~~~~~~~~~~~~~~~\times
 \left\{
           \frac{\partial \mathcal{F}(Q)(\bm{m})}{\partial \bm{r}_{j}}
           \mathcal{F}(Q) (- \bm{m})
          +\mathcal{F}(Q) (\bm{m})
           \frac{\partial \mathcal{F}(Q)(- \bm{m})}{\partial \bm{r}_{j}}
   \right\}
 \notag
 \\
 &=
   -\frac{1}{\pi V (4 \pi \epsilon_{0})} \sum_{\bm{m}} C(\bm{m}) B(\bm{m})
    \left\{
            \sum_{\bm{k}}
            \frac{\partial Q(\bm{k})}{\partial \bm{r}_{j}}
            \sum_{\bm{l}}
            Q(\bm{l})
            e^{2 \pi i \frac{\bm{m}}{\bm{K}} \cdot (\bm{k} - \bm{l})}
    \right\}
 \notag
 \\
 &=
   -\frac{1}{\pi V (4 \pi \epsilon_{0})}
    \sum_{\bm{k}}
    \frac{\partial Q(\bm{k})}{\partial \bm{r}_{j}}
    \sum_{\bm{l}}
     Q(\bm{l})
    \sum_{\bm{m}} C(\bm{m}) B(\bm{m})
    e^{2 \pi i \frac{\bm{m}}{\bm{K}} \cdot (\bm{k} - \bm{l})}
 \notag
 \\
 &=
   -\frac{1}{\pi V (4 \pi \epsilon_{0})}
    \sum_{\bm{k}}
    \frac{\partial Q(\bm{k})}{\partial \bm{r}_{j}}
    \sum_{\bm{l}}
     Q(\bm{l})
    \mathcal{F}(B \cdot C) (\bm{k} - \bm{l})
 \notag
 \\
 &=
   -\frac{1}{\pi V (4 \pi \epsilon_{0})}
    \sum_{k_{1}=0}^{K_{1}-1}
    \sum_{k_{2}=0}^{K_{2}-1}
    \sum_{k_{3}=0}^{K_{3}-1}
    \frac{\partial Q (\bm{k})}{\partial \bm{r}_{j}}
    \left\{
            \mathcal{F}(B \cdot C) * Q
    \right\} (k_{1}, k_{2}, k_{3})
 \label{eq:PME23}
\end{align}
と変形される. 第3式から第4式で$\bm{m}$に関するフーリエ変換の定義式, 続く第4式から第5式において畳み込みの定義を用いた.
さらに, フーリエ変換と畳み込みの関係式(\ref{eq:PME5})を用いると,
\begin{align}
    \mathcal{F} (B \cdot C) * Q
 &=
    K_{1} K_{2} K_{3}
    \mathcal{F}
    \left[
           \mathcal{F}^{-1} \mathcal{F} (B \cdot C) \cdot \mathcal{F}^{-1}(Q)
    \right]
 \notag
 \\
 &=
    K_{1} K_{2} K_{3}
    \mathcal{F}
    \left[
           B \cdot C \cdot \mathcal{F}^{-1} (Q)
    \right]
 \notag
 \\
 &=
 \mathcal{F}
 \left[
       B \cdot C \cdot \hat{\mathcal{F}}^{-1} (Q)
 \right]
\end{align}
であるので,
\begin{align}
    \bm{F}_{j}^{(2)}
 &=
   -\frac{1}{\pi V (4 \pi \epsilon_{0})}
    \sum_{k_{1}=0}^{K_{1}-1}
    \sum_{k_{2}=0}^{K_{2}-1}
    \sum_{k_{3}=0}^{K_{3}-1}
    \frac{\partial Q (\bm{k})}{\partial \bm{r}_{j}}
    \mathcal{F} \left[ B \cdot C \cdot \hat{\mathcal{F}}^{-1}(Q) \right]
 \notag
 \\
 &=
   -\frac{1}{\pi V (4 \pi \epsilon_{0})}
    \sum_{k_{1}=0}^{K_{1}-1}
    \sum_{k_{2}=0}^{K_{2}-1}
    \sum_{k_{3}=0}^{K_{3}-1}
    \left\{
            \sum_{\alpha = 1}^{3}
            \frac{\partial Q (\bm{k})}{\partial u_{\alpha j}}
            \frac{\partial u_{\alpha j}}{\partial \bm{r}_{j}}
    \right\}
    \mathcal{F}\left[ B \cdot C \cdot \hat{\mathcal{F}}^{-1}(Q) \right]
 \label{eq:PME24}
\end{align}
と計算される. ここで, B-spline関数の解析的微分の公式(\ref{eq:PME9})を用いると,
\begin{align}
    \left.
    \frac{\partial Q(\bm{k})}{\partial u_{\alpha j}}
    \right|_{\alpha = 1}
 &=
    \sum_{j=1}^{N}
    \sum_{n_{1}, n_{2}, n_{3}} q_{j}
    \frac{\partial M_{n}^{0}(u_{1j} - k_{1} - n_{1}K_{1})}{\partial u_{1j}}
 \notag
 \\
 &~~~~~~~~~~~~~~~~~~~~\times
    M_{n}^{0}(u_{2j} - k_{2} - n_{2}K_{2}) M_{n}^{0}(u_{3j} - k_{3} - n_{3}K_{3})
 \notag
 \\
 &=
    \sum_{j=1}^{N} \sum_{n_{1}, n_{2}, n_{3}} q_{j}
    \left\{
            M_{n-1}^{0}(u_{1j} - k_{1} - n_{1}K_{1}) - M_{n-1}^{0}(u_{1j} - k_{1} - n_{1}K_{1} - 1)
    \right\}
 \notag
 \\
 &~~~~~~~~~~~~~~~~~~~~\times
    M_{n}^{0}(u_{2j} - k_{2} - n_{2}K_{2}) M_{n}^{0}(u_{3j} - k_{3} - n_{3}K_{3})
 \label{eq:PME25}
\end{align}
と計算される. また, 式(\ref{eq:PME13})より
\begin{align}
 u_{\alpha j}
 = K_{\alpha} \bm{a}_{\alpha}^{*} \cdot \bm{r}_{j}
 = K_{\alpha} (a_{\alpha 1}^{*} r_{j1} + a_{\alpha 2}^{*} r_{j2} + a_{\alpha 3}^{*} r_{j3})
\end{align}
であるので,
\begin{align}
   \frac{\partial u_{\alpha j}}{\partial \bm{r}_{j}}
 = \left[\frac{\partial u_{\alpha j}}{\partial r_{j1}},~
         \frac{\partial u_{\alpha j}}{\partial r_{j2}},~
         \frac{\partial u_{\alpha j}}{\partial r_{j3}} \right]^{t}
 = \left[ K_{\alpha} a_{\alpha 1}^{*},~
          K_{\alpha} a_{\alpha 2}^{*},~
          K_{\alpha} a_{\alpha 3}^{*}
         \right]^{t}
 =
   K_{\alpha}\bm{a}_{\alpha}^{*}
 \label{eq:PME26}
\end{align}
と求まる. これより,
\begin{align}
  \frac{\partial Q(\bm{k})}{\partial \bm{r}_{j}}
 =
    \sum_{j=1}^{N} q_{j}
    \Bigl[
           &K_{1} \bm{a}_{1}^{*}
            \sum_{n_{1},n_{2},n_{3}}
            \left\{
              M_{n-1}^{0}(u_{1j} - k_{1} - n_{1}K_{1}) - M_{n-1}^{0}(u_{1j} - k_{1} - n_{1}K_{1} - 1)
            \right\}
            \notag \\
            &~~~~~~~~~~~~~~~\times
            M_{n}^{0}(u_{2j} - k_{2} - n_{2}K_{2}) M_{n}^{0}(u_{3j} - k_{3} - n_{3}K_{3})
    \notag \\
          +~&K_{2} \bm{a}_{2}^{*}
            \sum_{n_{1},n_{2},n_{3}}
            \left\{
              M_{n-1}^{0}(u_{2j} - k_{2} - n_{2}K_{2}) - M_{n-1}^{0}(u_{2j} - k_{2} - n_{2}K_{2} - 1)
            \right\}
            \notag \\
            &~~~~~~~~~~~~~~~\times
            M_{n}^{0}(u_{1j} - k_{1} - n_{1}K_{1}) M_{n}^{0}(u_{3j} - k_{3} - n_{3}K_{3})
    \notag \\
          +~&K_{3} \bm{a}_{3}^{*}
            \sum_{n_{1},n_{2},n_{3}}
            \left\{
              M_{n-1}^{0}(u_{3j} - k_{3} - n_{3}K_{3}) - M_{n-1}^{0}(u_{3j} - k_{3} - n_{3}K_{3} - 1)
            \right\}
            \notag \\
            &~~~~~~~~~~~~~~~\times
            M_{n}^{0}(u_{1j} - k_{1} - n_{1}K_{1}) M_{n}^{0}(u_{2j} - k_{2} - n_{2}K_{2})
    \Bigr]
 \label{eq:PME27}
\end{align}
と計算される.
\\


\subsection{PME法を用いた計算手順}
\begin{enumerate}
 \setlength{\leftskip}{0.4cm}
 \item[Step 0]
	      シミュレーションを始める前にあらかじめ, 式(\ref{eq:PME16}),(\ref{eq:PME11})から
	      $C(m_{1}, m_{2}, m_{3})$と$|b_{\alpha}(m_{\alpha})|^{2}$を計算する. \\
	      これらを用いて, $B(m_{1}, m_{2}, m_{3}) \cdot C(m_{1}, m_{2}, m_{3})$を計算する.
 \item[Step 1]
	      式(\ref{eq:PME18})を用いて, 配列$Q(k_{1}, k_{2}, k_{3})$を計算する.
	      B-spline関数 $M_{n}(u)$は$0<u<n$のみで値を持つ関数である. そのため,
	      粒子jの持つ電荷$q_{j}$を各グリッド上に配分する際には,
	      $j=1 \cdots N$, $\alpha=1,2,3$, $l=0,\cdots,n$
	      に対して$M_{n}(u_{\alpha j} - l)$を計算すれば十分である.
 \item[Step 2]
	      $Q(k_{1}, k_{2}, k_{3})$の配列を逆フーリエ変換することで$\hat{\mathcal{F}}^{-1} (Q)(m_{1}, m_{2}, m_{3})$を求める.
 \item[Step 3]
	      Step 2で求めた$\hat{\mathcal{F}}^{-1}(Q)(m_{1}, m_{2}, m_{3})$と Step 0で求めた
	      $B(m_{1}, m_{2}, m_{3}) \cdot C(m_{1}, m_{2}, m_{3})$を用いることで, 式(\ref{eq:PME22})から
	      逆空間からの静電相互作用ポテンシャルエネルギーを計算することができる.
 \item[Step 4]
	      $B(m_{1}, m_{2}, m_{3}) \cdot C(m_{1}, m_{2}, m_{3}) \cdot \hat{\mathcal{F}}^{-1}(Q)$を離散フーリエ変換する.
	      これを, 式(\ref{eq:PME24})に代入することで逆空間に由来する静電相互作用力を計算することができる.
\end{enumerate}
\clearpage

\section{Particle Mesh Ewald法を実装をした時のメモ}
\subsection{平行六面体セルの数学的基礎}

\subsection{分子系に対するEwald法の表式}
実際の分子系に対してLJ相互作用や静電相互作用の計算では,
2つ隣までの原子との相互作用(いわゆる1-2, 1-3相互作用)を取り除くことがある.
1-2, 1-3相互作用の原子ペア$(i,j)$の集合(Masked pairlist)を$M$と書く.
式(\ref{eq:Ewald2.1})には, このような相互作用も含まれているので,
$\sum_{(i,j) \in M} q_{i}q_{j} / |\bm{r}_{i} - \bm{r}_{j} |$
を実空間, 逆空間ポテンシャルから計算されるエネルギーから差し引かなくてはいけない.
さらに, 1-4相互作用に対する静電相互作用はスケール倍されることがある. このスケールファクターを$s$とする.
このような場合, Ewald法による静電ポテンシャル項と力は以下のように修正される.

 \begin{align}
     U_{\mathrm{elec}}^{(1)}
  =&
     \frac{1}{2} \sum_{\bm{n}} \sum_{i=1}^{N} \left.\sum_{j=1}^{N}\right.^{*}
     \frac{q_{i}q_{j}}{4 \pi \epsilon_{0}}
     \frac{s - \mathrm{erf}(\alpha |\bm{r}_{i} - \bm{r}_{j} + \bm{Ln}| )}
          {|\bm{r}_{i} - \bm{r}_{j} + \bm{Ln}|}
  \label{eq:Ewald38.1}
  \\
  \notag
  \\
     U_{\mathrm{elec}}^{(2)}
  =&
      \frac{1}{2 \pi V} \sum_{\bm{m}}
      \frac{\exp(\frac{- \pi^{2}|\bm{m}|^{2}}{\alpha^{2}})}{|\bm{m}|^{2}}
      \sum_{i=1}^{N} \sum_{j=1}^{N}
      \frac{q_{i}q_{j}}{4 \pi \epsilon_{0}}
       e^{2 \pi i \bm{m} \cdot (\bm{r}_{i} - \bm{r}_{j})}
  \label{eq:Ewald38.2}
  \\
      U_{\mathrm{elec}}^{(3)}
  =&
    -\frac{1}{2} \sum_{(i,j) \in M}
     \frac{q_{i}q_{j}}{4 \pi \epsilon_{0}}
     \frac{\mathrm{erf}(\alpha |\bm{r}_{i} - \bm{r}_{j}|)}
          {|\bm{r}_{i} - \bm{r}_{j}|}
    -\frac{\alpha}{\sqrt{\pi}} \sum_{i=1}^{N}\frac{q_{i}^{2}}{4 \pi \epsilon_{0}}
  \label{eq:Ewald38.3}
  \\
  \notag
  \\
     \bm{F}_{i}^{(1)}
  =&
     \sum_{\bm{n}} \left. \sum_{j=1}^{N} \right.^{*}
     \frac{q_{i}q_{j}}{4 \pi \epsilon_{0}}
  \notag
  \\
  &~~\times
     \left[
            \frac{s-\mathrm{erf} \{ \alpha |\bm{r}_{i} - \bm{r}_{j} + \bm{Ln}| \} }
                 {|\bm{r}_{i} - \bm{r}_{j} + \bm{Ln}|^{2}}
           +\frac{2 \alpha}{\sqrt{\pi}}
            \frac{\exp \{ -\alpha^{2} |\bm{r}_{i} - \bm{r}_{j} + \bm{Ln}|^{2} \}}
                 {|\bm{r}_{i} - \bm{r}_{j} + \bm{Ln}|}
     \right]
  \notag
  \\
  &~~\times
     \frac{ \bm{r}_{i} - \bm{r}_{j} + \bm{Ln} }
          {|\bm{r}_{i} - \bm{r}_{j} + \bm{Ln}|}
  \label{eq:Ewald38.4}
  \\
  \notag
  \\
     \bm{F}_{i}^{(2)}
  =&
    -\frac{2}{ V}
     \sum_{\bm{m}} \frac{i \bm{m}}{|\bm{m}|^{2}}
     \exp \left( -\frac{ \pi^{2}|\bm{m}|^{2} }{ \alpha^{2} } \right)
     \sum_{j=1}^{N}
     \frac{q_{i}q_{j}}{4 \pi \epsilon_{0}}
     e^{2 \pi i \bm{m} \cdot (\bm{r}_{i} - \bm{r}_{j})}
  \label{eq:Ewald38.5}
  \\
  \notag
  \\
     \bm{F}_{i}^{(3)}
  =&
     \sum_{(i,j) \in M}
     \frac{q_{i}q_{j}}{4 \pi \epsilon_{0}}
  % \notag
  % \\
  % &~~\times
     \left[
           -\frac{\mathrm{erf} \{ \alpha |\bm{r}_{i} - \bm{r}_{j}| \} }
                 {|\bm{r}_{i} - \bm{r}_{j}|^{2}}
           +\frac{2 \alpha}{\sqrt{\pi}}
            \frac{\exp \{ -\alpha^{2} |\bm{r}_{i} - \bm{r}_{j}|^{2} \}}
                 {|\bm{r}_{i} - \bm{r}_{j}|}
     \right]
     \frac{\bm{r}_{i} - \bm{r}_{j}}
          {|\bm{r}_{i} - \bm{r}_{j}|}
\end{align}
式(\ref{eq:Ewald38.1})の$*$は$\bm{n} = 0$における$i=j$の場合, あるいは$(i,j) \in M$の場合
の項を取り除くことを意味する.

\subsection{実空間の計算について}
\paragraph{実空間の表式} \

実空間に由来する静電相互作用は, 多くの場合単位セル内で収束してしまう.
このように実空間の相互作用を単位セル内のみで計算される時は
\begin{align}
    U_{\mathrm{elec}}^{(1)}
 &=
    \frac{1}{2} \sum_{\bm{n}} \sum_{i=1}^{N} \left.\sum_{j=1}^{N}\right.^{*}
    \frac{q_{i}q_{j}}{4 \pi \epsilon_{0}}
    \frac{s-\mathrm{erf}(\alpha |\bm{r}_{i} - \bm{r}_{j} + \bm{Ln}| )}
          {|\bm{r}_{i} - \bm{r}_{j} + \bm{Ln}|}
 \notag
 \\
 &=
    \sum_{\mathrm{nonbond}}
    \frac{q_{i}q_{j}}{4 \pi \epsilon_{0}}
    \frac{s-\mathrm{erf}(\alpha |\bm{r}_{i} - \bm{r}_{j}| )}
         {|\bm{r}_{i} - \bm{r}_{j}|}
 \notag
\end{align}
と書き直すことができる.

$|\bm{r}_{i} - \bm{r}_{j}|$を無次元化した座標を用いて計算することを考える.
無次元化した座標$\Tilde{\bm{r}}_{i}$を以下のように定める.
\begin{align}
   \bm{r}_{i}
 =
   \left(
          \begin{array}{c}
            x_{i} \\ y_{i} \\ z_{i}
          \end{array}
   \right)
 =
  (\bm{a}_{1}, \bm{a}_{2}, \bm{a}_{3})
   \left(
          \begin{array}{c}
            \xi_{i} \\ \eta_{i} \\ \zeta_{i}
          \end{array}
   \right)
 \equiv
   \bm{L}\Tilde{\bm{r}}_{i}
\end{align}
ここで計量テンソル
\begin{align}
   \bm{L}^{\mathrm{t}} \bm{L}
 =
    \left(
          \begin{array}{ccc}
           \bm{a}_{1} \cdot \bm{a}_{1} & \bm{a}_{1} \cdot \bm{a}_{2} & \bm{a}_{1} \cdot \bm{a}_{3} \\
           \bm{a}_{2} \cdot \bm{a}_{1} & \bm{a}_{2} \cdot \bm{a}_{2} & \bm{a}_{2} \cdot \bm{a}_{3} \\
           \bm{a}_{3} \cdot \bm{a}_{1} & \bm{a}_{3} \cdot \bm{a}_{2} & \bm{a}_{3} \cdot \bm{a}_{3} \\
          \end{array}
   \right)
  \equiv
 \bm{G}
\end{align}
を定めると,
\begin{align}
     |\bm{r}_{i} - \bm{r}_{j}|
 =&~
     |\bm{L} (\Tilde{\bm{r}}_{i} - \Tilde{\bm{r}}_{j})|
 \notag
 \\
 =&~
   \left\{ (\Tilde{\bm{r}}_{i} - \Tilde{\bm{r}}_{j})^{\mathrm{t}} \bm{L}^{\mathrm{t}}
            \bm{L} (\Tilde{\bm{r}}_{i} - \Tilde{\bm{r}}_{j})
   \right\}^{\frac{1}{2}}
 \notag
 \\
 =&~
   \left\{
      G_{11} \xi_{ji}^{2} + G_{22} \eta_{ji}^{2} + G_{33} \zeta_{ji}^{2}
  + 2(G_{12} \xi_{ji} \eta_{ji} + G_{13} \xi_{ji} \zeta_{ji} + G_{23} \eta_{ji} \zeta_{ji})
   \right\}^{\frac{1}{2}}
\end{align}
と原子間の距離を計算することができる.
\paragraph{単位セルが立方体の時} \

単位セルが1辺の長さが$L$の立方体の時, 計量テンソルは
\begin{align}
   \bm{G}
 =
 \left(
        \begin{array}{ccc}
         L^{2} & 0     & 0     \\
         0     & L^{2} & 0     \\
         0     & 0     & L^{2} \\
        \end{array}
 \right)
 \notag
\end{align}
であるので,
\begin{align}
   \frac{\mathrm{erfc} \left(\alpha |\bm{r}_{i} - \bm{r}_{j}|\right)}{|\bm{r}_{i} - \bm{r}_{j}|}
&=
   \frac{\mathrm{erfc} \left\{\alpha L \left(\xi_{ji}^{2} + \eta_{ji}^{2} + \zeta_{ji}^{2} \right)^\frac{1}{2}\right\}}
        {\alpha L \left( \xi_{ji}^{2} + \eta_{ji}^{2} + \zeta_{ji}^{2} \right)^\frac{1}{2}}
 \notag
 \\
&=
   \frac{\mathrm{erfc} \left\{\kappa \left(\xi_{ji}^{2} + \eta_{ji}^{2} + \zeta_{ji}^{2} \right)^\frac{1}{2}\right\}}
        {\kappa \left( \xi_{ji}^{2} + \eta_{ji}^{2} + \zeta_{ji}^{2} \right)^\frac{1}{2}}
 \notag
\end{align}
とかける.
\subsection{PMEを用いた逆空間の静電相互作用の計算アルゴリズム}
PMEを用いて逆空間の静電ポテンシャルと力を計算する手順を示す.

\paragraph{Step~0-1 $B(m_{1}, m_{2}, m_{3})$の計算} \
\begin{align}
   B(m_{1}, m_{2}, m_{3})
 =
   |b_{1}(m_{1})|^{2} |b_{2}(m_{2})|^{2} |b_{3}(m_{3})|^{2}
 \notag
\end{align}
ここで, $0 \le m_{\alpha} \le K_{\alpha}-1$であり,
\begin{align}
  b_{\alpha} (m_{\alpha})
 =
   \exp \left\{2 \pi i (n-1) \frac{m_{\alpha}}{K_{\alpha}} \right\}
   \left[
         \sum_{k=0}^{n-2} M_{n}^{0}(k+1) \exp \left(2 \pi i \frac{m_{\alpha}}{K_{\alpha}} k \right)
   \right]^{-1}
 \notag
\end{align}
で求められる. オイラーの公式を用いて指数関数を三角関数で書き直すと,
\begin{align}
   b_{\alpha} (m_{\alpha})
 &=
   \frac{\cos\left\{2 \pi \frac{m_{\alpha}}{K_{\alpha}}(n-1)\right\} + i \sin\left\{2 \pi \frac{m_{\alpha}}{K_{\alpha}}(n-1)\right\}}
        {   \sum_{k=0}^{n-2} M_{n}^{0}(k+1) \cos \left(2 \pi \frac{m_{\alpha}}{K_{\alpha}} k \right)
         +i \sum_{k=0}^{n-2} M_{n}^{0}(k+1) \sin \left(2 \pi \frac{m_{\alpha}}{K_{\alpha}} k \right)
         }
 \notag
\end{align}
であるので,
\begin{align}
   |b_{\alpha}(m_{\alpha})|^{2}
 = \left[
          \left\{ \sum_{k=0}^{n-2} M_{n}^{0}(k+1) \cos \left(2 \pi \frac{m_{\alpha}}{K_{\alpha}} k \right) \right\}^{2}
        + \left\{ \sum_{k=0}^{n-2} M_{n}^{0}(k+1) \sin \left(2 \pi \frac{m_{\alpha}}{K_{\alpha}} k \right) \right\}^{2}
   \right]^{-1}
 \notag
\end{align}
と計算される.

\paragraph{Step~0-2 $C(m_{1}, m_{2}, m_{3})$の計算} \
\begin{align}
  &C(\bm{m})
 = C(m_{1}, m_{2}, m_{3})
= \frac {\exp( - \pi^{2}|\bm{m}|^{2} / \alpha^{2} ) }{|\bm{m}|^{2}}
 ~~~~\textrm{for $\bm{m} \neq 0$}, ~~C(0,0,0) = 0
 \notag
\end{align}
$\bm{m} \equiv m_{1}^{\prime}\bm{a}_{1}^{*} + m_{2}^{\prime} \bm{a}_{2}^{*}+ m_{3}^{\prime}\bm{a}_{3}^{*}$であり,
$m_{i}^{\prime}$は$0 \le m_{i} \le K_{i}/2$の範囲で$m_{i}^{\prime} = m_{i}$, その他の範囲では$m_{i}^{\prime} = m_{k} - K_{i}$である.

したがって$\bm{m}^{2}$は
\begin{align}
   \bm{m}^{2}
=&~
   (m_{1}^{\prime}\bm{a}_{1}^{*} + m_{2}^{\prime} \bm{a}_{2}^{*}+ m_{3}^{\prime}\bm{a}_{3}^{*})
   (m_{1}^{\prime}\bm{a}_{1}^{*} + m_{2}^{\prime} \bm{a}_{2}^{*}+ m_{3}^{\prime}\bm{a}_{3}^{*})
 \notag
 \\
 =&~
     m_{1}^{\prime 2} \bm{a}_{1}^{*} \cdot \bm{a}_{1}^{*}
   + m_{2}^{\prime 2} \bm{a}_{2}^{*} \cdot \bm{a}_{2}^{*}
   + m_{3}^{\prime 2} \bm{a}_{3}^{*} \cdot \bm{a}_{3}^{*}
 \notag
 \\
 &+ 2 (  m_{1}^{\prime 2} m_{2}^{\prime 2} \bm{a}_{1}^{*} \cdot \bm{a}_{2}^{*}
       + m_{1}^{\prime 2} m_{3}^{\prime 2} \bm{a}_{1}^{*} \cdot \bm{a}_{3}^{*}
       + m_{2}^{\prime 2} m_{3}^{\prime 2} \bm{a}_{2}^{*} \cdot \bm{a}_{3}^{*} )
 \notag
\end{align}
と計算される.
\\

\subparagraph{単位ユニットが1辺$L$の立方体の時} \

逆格子ベクトルが
\begin{align}
 \bm{a}_{\alpha}^{*} = \frac{1}{L} (1, 0, 0)^{\mathrm{t}}
 \notag
\end{align}
であるので
\begin{align}
   \bm{m}^{2}
 =
   \frac{1}{L^2} (m_{1}^{\prime 2} + m_{2}^{\prime 2} + m_{3}^{\prime 2})^{2}
 \notag
\end{align}
と求められる. ゆえに$C(\bm{m})$は
\begin{align}
   \frac{1}{\pi V} C(m_{1}, m_{2}, m_{3})
 &=
   \frac{1}{\pi V}
   \frac{\exp\left\{\pi^{2} (m_{1}^{\prime 2} + m_{2}^{\prime 2} + m_{3}^{\prime 2})^{2}/(\alpha L)^2\right\}}
        {(m_{1}^{\prime 2} + m_{2}^{\prime 2} + m_{3}^{\prime 2})/L^2}
 \notag
 \\
  &=
   \frac{1}{\pi L}
   \frac{\exp\left\{\pi^{2} (m_{1}^{\prime 2} + m_{2}^{\prime 2} + m_{3}^{\prime 2})^{2}/\kappa^2\right\}}
        {m_{1}^{\prime 2} + m_{2}^{\prime 2} + m_{3}^{\prime 2}}
 \notag
\end{align}
と計算される.
このような表式を使用すると, $L$に依存しない形になる(つまり無次元化されている).
したがってAndersenの方法のような立方体のシミュレーションボックスの圧力を制御するときに, 体積変化のたびに$C(m_1, m_2, m_3)$を更新する必要がなくなる.


\paragraph{Step~0-3 $B \cdot C(m_{1}, m_{2}, m_{3})$の計算} \

Step~0-1, Step~0-2で計算した$B(m_{1}, m_{2}, m_{3})$と$C(m_{1}, m_{2}, m_{3})$から,
$B \cdot C (m_{1}, m_{2}, m_{3})$を計算する.
\\

\paragraph{Step~1 $u_{\alpha j}$の計算} \
\begin{align}
    u_{\alpha j }
  =
    K_{\alpha} \bm{a}_{\alpha}^{*} \cdot \bm{r}_{j}
  =
    K_{\alpha} \bm{a}_{\alpha}^{*} \cdot \bm{L} \Tilde{\bm{r}}_{j}
  =
    K_{\alpha} \bm{a}_{\alpha}^{*}
    \cdot
    (\xi_{j} \bm{a}_{1} + \eta_{j} \bm{a}_{2} + \zeta_{j}\bm{a}_{3})
 \notag
\end{align}
であるので, スケールされた座標系$\Tilde{\bm{r}}_{j}=(\xi_{j},~ \eta_{j},~ \zeta_{j})$を用いて
\begin{align}
 u_{1j} &= K_{1} \xi_{j}   \notag \\
 u_{2j} &= K_{2} \eta_{j}  \notag \\
 u_{3j} &= K_{3} \zeta_{j} \notag
\end{align}
と計算できる.

\paragraph{Step 2 $Q(k_{1}, k_{2}, k_{3})$の計算} \
\begin{align}
    Q(\bm{k})
 \equiv
    Q(k_{1}, k_{2}, k_{3})
 &=
    \sum_{j=1}^{N} q_{j} \sum_{n_{1}, n_{2}, n_{3}}
    M_{n}^{0}(u_{1j}- k_{1} - n_{1}K_{1})
 \notag
 \\ &~~~~ \times
    M_{n}^{0}(u_{2j} - k_{2} - n_{2}K_{2}) M_{n}^{0}(u_{3j} - k_{3} - n_{3}K_{3})
 \notag
\end{align}
ここで$M_{n}^{0}$はB-Spline関数である.

B-Spline関数$M_{n}^{0}(u)$の定義域は$0 \le u \le n$であるので,
$M_{n}^{0}(u_{\alpha j} - k_{\alpha} - n_{\alpha}K_{\alpha})$の定義域は
$0 \le u_{\alpha j} - k_{\alpha} - n_{\alpha}K_{\alpha} \le n$となる.
したがって
\begin{alignat}{5}
  &-n                &\le &~ k_{\alpha} + n_{\alpha}K_{\alpha} - u_{\alpha} &\le &~0
 \notag
 \\
  &-n + u_{\alpha j} &\le &~ k_{\alpha} + n_{\alpha}K_{\alpha}              &\le &~u_{\alpha j}
\end{alignat}
と変化される.
$u_{\alpha j}$の定義域は$0 \le u_{\alpha j} \le K_{\alpha}$であることから,
\begin{align}
 -n < k_{\alpha} + n_{\alpha} K_{\alpha} \le K_{\alpha}
\end{align}
という不等式を得る.
これより$n_{\alpha} = -1,~ 0$を考えれば十分であることが分かる.
逆格子空間における座標$u_{\alpha j}$を整数部分$u_{\alpha j}^{\mathrm{int}}$
と小数部分$u_{\alpha j}^{\mathrm{frac}}$に分割する.
B-spline関数の定義域を考えれば, 粒子$j$の電荷は
$k_{\alpha} = u_{\alpha j}^{\mathrm{int}},~ u_{\alpha j}^{\mathrm{int}}-1,~ \cdots,~ u_{\alpha j}^{\mathrm{int}} - (n - 1)$
のみに内挿されることが分かる.

\paragraph{Step 3 $Q(k_{1}, k_{2}, k_{3})$のフーリエ変換を実行} \
\begin{align}
   \hat{\mathcal{F}}^{-1} (Q) (m_{1}, m_{2}, m_{3})
 =
   \sum_{k_{1}=0}^{K_{1}-1} \sum_{k_{2}=0}^{K_{2}-1} \sum_{k_{3}=0}^{K_{3}-1}
   Q(k_{1}, k_{2}, k_{3})
   e^{-2 \pi i \left(\frac{m_{1}}{K_{1}}k_{1} + \frac{m_{2}}{K_{2}}k_{2} +\frac{m_{3}}{K_{3}}k_{3} \right)}
 \notag
\end{align}
を計算する
\paragraph{Step 4 逆空間におけるポテンシャル$U_{\mathrm{elec}}^{(2)}$の計算} \

$\hat{\mathcal{F}}^{-1}(Q)(\bm{m})$とStep~0で求めた$B \cdot C(\bm{m})$を用いることで,
逆空間からの静電ポテンシャルを
\begin{align}
    U_{\mathrm{elec}}{(2)}
 =
    \frac{1}{2 \pi V (4 \pi \epsilon_{0})}
    \sum_{m_{1}=0}^{K_{1}-1} \sum_{m_{2}=0}^{K_{2}-1} \sum_{m_{3}=0}^{K_{3}-1}
    B \cdot C(m_{1}, m_{2}, m_{3})
    | \hat{\mathcal{F}}^{-1}(Q) (m_{1}, m_{2}, m_{3}) |^{2}
 \notag
\end{align}
と計算することができる.
\paragraph{Step 4 逆空間に由来する力$F_{j}^{(2)}$の計算} \

$B \cdot C \cdot \hat{\mathcal{F}}^{-1}(Q)$をフーリエ変換した
$\mathcal{F} \left\{B \cdot C \cdot \hat{\mathcal{F}}^{-1}(Q)\right\}$を計算することで,
逆空間の静電相互ポテンシャルに由来する力を計算することができる.
\begin{align}
    \bm{F}_{j}^{(2)}
 &=
   -\frac{1}{\pi V (4 \pi \epsilon_{0})}
    \sum_{k_{1}=0}^{K_{1}-1}
    \sum_{k_{2}=0}^{K_{2}-1}
    \sum_{k_{3}=0}^{K_{3}-1}
    \frac{\partial Q (\bm{k})}{\partial \bm{r}_{j}}
    \mathcal{F} \left[ B \cdot C \cdot \hat{\mathcal{F}}^{-1}(Q) \right]
\end{align}
ここで,
\begin{align}
  \frac{\partial Q(\bm{k})}{\partial \bm{r}_{j}}
 =
    \sum_{j=1}^{N} q_{j}
    \Bigl[
           &K_{1} \bm{a}_{1}^{*}
            \sum_{n_{1},n_{2},n_{3}}
            \left\{
              M_{n-1}^{0}(u_{1j} - k_{1} - n_{1}K_{1}) - M_{n-1}^{0}(u_{1j} - k_{1} - n_{1}K_{1} - 1)
            \right\}
            \notag \\
            &~~~~~~~~~~~~~~~\times
            M_{n}^{0}(u_{2j} - k_{2} - n_{2}K_{2}) M_{n}^{0}(u_{3j} - k_{3} - n_{3}K_{3}
    \notag \\
          +~&K_{2} \bm{a}_{2}^{*}
            \sum_{n_{1},n_{2},n_{3}}
            \left\{
              M_{n-1}^{0}(u_{2j} - k_{2} - n_{2}K_{2}) - M_{n-1}^{0}(u_{2j} - k_{2} - n_{2}K_{2} - 1)
            \right\}
            \notag \\
            &~~~~~~~~~~~~~~~\times
            M_{n}^{0}(u_{1j} - k_{1} - n_{1}K_{1}) M_{n}^{0}(u_{3j} - k_{3} - n_{3}K_{3})
    \notag \\
          +~&K_{3} \bm{a}_{3}^{*}
            \sum_{n_{1},n_{2},n_{3}}
            \left\{
              M_{n-1}^{0}(u_{3j} - k_{3} - n_{3}K_{3}) - M_{n-1}^{0}(u_{3j} - k_{3} - n_{3}K_{3} - 1)
            \right\}
            \notag \\
            &~~~~~~~~~~~~~~~\times
            M_{n}^{0}(u_{1j} - k_{1} - n_{1}K_{1}) M_{n}^{0}(u_{2j} - k_{2} - n_{2}K_{2})
    \Bigr]
\end{align}
と計算される.
\\

\subparagraph{単位セルが1辺の長さ$L$の立方体の時} \
\begin{align}
 \frac{\partial u_{\alpha j}}{\partial \bm{r}_{j}} = K_{\alpha}\bm{a}_{\alpha}^{*}
 \notag
\end{align}
であるので,
\begin{align}
 \frac{\partial u_{1 j}}{\partial \bm{r}_{j}}
 = \frac{K_{1}}{L}
    \left(
          \begin{array}{c}
            1 \\ 0 \\ 0
          \end{array}
    \right)
 ,~~~
  \frac{\partial u_{2 j}}{\partial \bm{r}_{j}}
 = \frac{K_{2}}{L}
    \left(
          \begin{array}{c}
            0 \\ 1 \\ 0
          \end{array}
 \right)
 ,~~~
  \frac{\partial u_{3 j}}{\partial \bm{r}_{j}}
 = \frac{K_{3}}{L}
    \left(
          \begin{array}{c}
            0 \\ 0 \\ 1
          \end{array}
    \right)
 \notag
\end{align}
と計算される.
\clearpage
\section{付録}
\subsection{B-spline関数の具体例}
\subsubsection{1次のB-spline関数}
\begin{align}
  M_{2}^{l}(u)
=
  \begin{cases}
   u-l,~      & u \in    (l,~   l+1] \\
   2-(u-l),~  & u \in    (l+1,~ l+2] \\
   0,~        & u \notin (l,~   l+2] \\
  \end{cases}
\end{align}

\subsubsection{2次のB-spline関数}
\begin{align}
  M_{3}^{l}(u)
=
  \begin{cases}
   \frac{1}{2}(u-l)^{2},~               & u \in    (l,~   l+1] \\
   -(u-l)^{2} + 3(u-l) - \frac{3}{2},~  & u \in    (l+1,~ l+2] \\
   \frac{1}{2}\{(u-l)-3\}^{2},~         & u \in    (l+2,~ l+3] \\
   0,~                                  & u \notin (l,  ~ l+3]
  \end{cases}
\end{align}

\subsubsection{3次のB-spline関数}
\begin{align}
  M_{4}^{l}(u)
=
  \begin{cases}
   \frac{1}{6}(u-l)^{3},~                                 & u \in    (l,~   l+1] \\
   \frac{1}{6}\{-3(u-l)^3 + 12(u-l)^2 - 12(u-l) +  4\},~  & u \in    (l+1,~ l+2] \\
   \frac{1}{6}\{ 3(u-l)^3 - 24(u-l)^2 - 60(u-l) - 44\},~  & u \in    (l+2,~ l+3] \\
  -\frac{1}{6}\{(u-l) - 4\}^{2},~                         & u \in    (l+3,~ l+4] \\
   0,~                                                    & u \notin (l,~   l+4]
  \end{cases}
\end{align}

\clearpage

\subsection{B-spline関数の解析微分の証明}
数学的帰納法によって証明できる.
まず$n=3$の時,
\begin{align}
\frac{d}{du} M_{3}^{l}(u)
=
  \begin{cases}
    u-l,~        & u \in    (l,~   l+1] \\
    -2(u-l)+3,~  & u \in    (l+1,~ l+2] \\
    (u-l)-3,~    & u \in    (l+2,~ l+3] \\
    0,~          & u \notin (l,  ~ l+3]
   \end{cases}
\end{align}
と計算できる.
\begin{align}
  M_{2}^{l}(u)
=
  \begin{cases}
   u-l,~      & u \in    (l,~   l+1] \\
   2-(u-l),~  & u \in    (l+1,~ l+2] \\
   0,~        & u \notin (l,~   l+2] \\
  \end{cases}
\end{align}
かつ,
\begin{align}
  M_{2}^{l+1}(u)
=
  \begin{cases}
   u-(l+1),~        & u \in    (l+1,~ l+2] \\
   2-\{u-(l+1)\},~  & u \in    (l+2,~ l+3] \\
   0,~              & u \notin (l+1,~ l+3] \\
  \end{cases}
\end{align}
であるので, $n=3$の時
\begin{align}
 \frac{d}{du} M_{3}^{l}(u) = M_{2}^{l}(u) - M_{2}^{l+1}(u)
\end{align}
が成立. 一般の$n$について式(\ref{eq:PME9})が成り立つと仮定する.
漸化式(\ref{eq:PME8})より,
\begin{align}
  M_{n+1}^{l}(u)
&=
  \frac{u-l}{n} M_{n}^{l} (u) + \frac{-u+n+l+1}{n} M_{n}^{l+1} (u)
\end{align}
となる. $u$について微分すると,
\begin{align}
   \frac{d}{du} M_{n+1}^{l}(u)
=&~
   \frac{d}{du}
   \left\{
           \frac{u-l}{n} M_{n}^{l}(u) + \frac{-u+n+l+1}{n} M_{n}^{l+1}(u)
   \right\}
 \notag \\
 \notag \\
=&~
    \frac{1}{n} M_{n}^{l}(u)
  + \frac{u-l}{n} \frac{d M_{n}^{l}(u)}{du}
  - \frac{1}{n}M_{n}^{l+1}(u)
  + \frac{-u+n+l+1}{n} \frac{d M_{n}^{l+1}(u)}{du}
 \notag \\
 \notag \\
=&~
    \frac{1}{n} M_{n}^{l}(u) - \frac{1}{n}M_{n}^{l+1}(u)
  \notag
  \\
 &+ \frac{u-l}{n} \{M_{n-1}^{l} - M_{n-1}^{l+1}(u)\}
  + \frac{-u+n+l+1}{n} \{M_{n-1}^{l+1} - M_{n-1}^{l+2}(u)\}
 \notag \\
 \notag \\
=&~
    \frac{1}{n}M_{n}^{l}(u) - \frac{1}{n}M_{n}^{l+1}(u)
  \notag
  \\
 &+ \frac{1}{n}
    \left\{
            (u-l) M_{n-1}^{l}(u) + (-u+n+l) M_{n-1}^{l+1}(u) - n M_{n-1}^{l+1}(u)
    \right\}
  \notag
  \\
 &+ \frac{1}{n}
    \left\{
    -(u-n-l-1) M_{n-1}^{l+1}(u) - (-u+n+l+1) M_{n-1}^{l+2}(u)
    \right\}
 \notag \\
 \notag \\
=&~
    \frac{1}{n} M_{n}^{l}(u) - \frac{1}{n} M_{n}^{l+1}(u)
  \notag
  \\
 &+ \frac{n-1}{n}
    \left\{
            \frac{u-l}{n-1} M_{n-1}^{l}(u) + \frac{-u+n+l}{n-1} M_{n-1}^{l+1}(u)
    \right\}
  \notag
  \\
 &- \frac{n-1}{n}
    \left\{
    \frac{-u-(l+1)}{n-1} M_{n-1}^{l+1}(u) - \frac{-u+n+(l+1)}{n-1} M_{n-1}^{l+2}(u)
    \right\}
 \notag \\
 \notag \\
=&~
    \frac{1}{n}M_{n}^{l}(u) - \frac{1}{n}M_{n}^{l+1}(u)
  + \frac{n-1}{n} M_{n}^{l}(u) - \frac{n-1}{n} M_{n}^{l+1}
 \notag \\
 \notag \\
=&~
    M_{n}^{l} (u) - M_{n}^{l+1}(u)
 \notag
\end{align}
\\
以上より,
\begin{equation}
  \frac{d}{du} M_{n}^{l}(u) = M_{n-1}^{l}(u) - M_{n-1}^{l+1}(u)
  \notag
\end{equation}
が成立.

\clearpage

\bibliographystyle{junsrt}
\bibliography{ewald}
\end{document}

