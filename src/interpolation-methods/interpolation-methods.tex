\documentclass[a4paper, 10.5pt, oneside, openany, uplatex]{jsarticle}

\author{山内 仁喬}
% 余白の設定.
% 参考文献:Latex2e 美文書作成入門, 14.3ページレイアウトの変更

% 行長の変更
\setlength{\textwidth}{40zw}           %全角40文字分

% 行間を制御するコマンド
\renewcommand{\baselinestretch}{0.9}

% 左マージンを変更
\setlength{\oddsidemargin}{25truemm}   % 左余白
\addtolength{\oddsidemargin}{-1truein} % 左位置デフォルトから-1inch

% 上マージンを変更
\setlength{\topmargin}{15truemm}       % 上余白
\addtolength{\topmargin}{-1truein}     % 上位置デフォルトから-1inch

% 本文領域の縦横の長さ変更
\setlength{\textheight}{242truemm}     % テキスト高さ: 297-(25+30)=242mm
\setlength{\textwidth}{160truemm}      % テキスト幅:  210-(25+25)=160mm
\setlength{\fullwidth}{\textwidth}     % ページ全体の幅


% 図・表の個数などの設定.
%% 図・表を入りやすさを制御するパラメーター
\setcounter{topnumber}{4}
\setcounter{bottomnumber}{4}
\setcounter{totalnumber}{4}
\setcounter{dbltopnumber}{3}
\setcounter{tocdepth}{1} % 項レベルまで目次に反映させるコマンド.
\renewcommand{\topfraction}{.95}
\renewcommand{\bottomfraction}{.90}
\renewcommand{\textfraction}{.05}
\renewcommand{\floatpagefraction}{.95}

% 使用するパッケージを記述.
\usepackage{amsmath} % 複雑な数式を使うときに便利
\usepackage{dcolumn}
\usepackage{color}
\usepackage{tabularx, dcolumn}
\usepackage{bm} % 数式環境内で太字を使うときに便利.
\usepackage{subcaption}  % 関連した複数の図を並べる時に使う
\usepackage[dvipdfmx]{graphicx} % 画像を挿入したり,テキストや図の拡大縮小・回転を行う.
\usepackage{verbatim} % 入力どおりの出力を行う.
\usepackage{makeidx} % 索引を作成できる.
\usepackage{dcolumn} % 表の数値を小数点で桁を揃える.
\usepackage{lscape} % 図表を90度横に倒して配置する.
\usepackage{setspace} % 行間調整.

\def\mbf#1{\mbox{\boldmath ${#1}$}}

% \newcolumntype{d}{D{+}{\,\pm\,}{4,5}}
% \newcolumntype{i}{D{+}{\,\pm\,}{-1}}
% \newcolumntype{.}{D{.}{.}{6,3}}

\begin{document}


\title{関数の近似と補間法}
\maketitle

\section{線形最小二乗法}
次のような$n$組みのデータのあてはめ問題を考える:
\begin{align}
    \mathrm{観測点:}~ & t_{1},~ t_{2},~\ldots,~ t_{n}\\
    \mathrm{測定値:}~ & f_{1},~ f_{2},~\ldots,~ f_{n}\\
    \mathrm{測定値の分散:}~ & \sigma_{1},~ \sigma_{2},~\ldots,~ \sigma_{n}
\end{align}
測定の誤差(分散)$\sigma_{i}^{2}$は, 測定値$f_{i}$の信頼性を表す尺度として考えることができる. 
このデータを, ある一次独立な関数系
\begin{equation}
    \phi_{1}(t),~ \phi_{2}(t),~\ldots,~ \phi_{m}(t)
\end{equation}
の一次結合
\begin{equation}
    f(t) = \sum_{j}^{m} x_{j} \phi_{j}(t)
\end{equation}
によってあてはめる. 
この形の$f(t)$は$\{\phi_{j}(t)\}$に関して線形であるから, 線形モデルと呼ばれる. 
一次独立な関数系として, 単項式
\begin{equation}
    \phi_{j}(t) = t^{j-1}
\end{equation}
が最も広く採用されている. 

最小二乗法は, 
\begin{align}
    S(x_{1},~ x_{2},~\ldots,~ x_{m})
    &=
    \sum_{i=1}^{n} \frac{[f_{i} - f(t_{i})]^{2}}{\sigma_{i}^{2}}
    \\ &=
    \sum_{i=1}^{n}
    \frac{[f_{i} - \sum_{j=1}^{n} x_{j} \phi_{j}(t_{i})]^{2}}
    {\sigma_{i}^{2}}
\end{align}
を最小にすることによって, 未知係数$x_{1},~\ldots,~x_{m}$の組みを決定する方法である. 
$S$を最小にする条件は, 
\begin{equation}
    \frac{\partial S}{\partial x_{k}} = 0,
    ~~~~
    k = 1,2,\ldots,m
\end{equation}
によって与えられるので, 
\begin{equation}
    \sum_{i=1}^{n}
    \frac{[f_{i} - \sum_{j=1}^{n} x_{j} \phi_{j}(t_{i})]\phi_{k}(t_{i})}
    {\sigma_{i}^{2}} = 0,
    ~~~~
    k = 1,2,\ldots,m
    \label{Eq:LSM-Minimum-Condition1}
\end{equation}
とかける. ここで$i$, $j$成分が
\begin{equation}
    A_{ij} = \frac{\phi_{j}(t_{j})}{\sigma_{i}},
    ~~~~
    (i = 1,2,\ldots,n,
    ~
    j = 1,2,\ldots,m)
\end{equation}
で定義される$n \times m$行列を導入する. 
この$A_{ij}$はヤコビアン行列あるいは計画行列と呼ばれている. 
このとき, 式(\ref{Eq:LSM-Minimum-Condition1})は
\begin{align}
    &
    \sum_{i=1}^{n} \sum_{j=1}^{m}
    \frac{\phi_{j}(t_{i})}{\sigma_{i}}
    \frac{\phi_{k}(t_{i})}{\sigma_{i}} =
    \sum_{i=1}^{n}
    \frac{\phi_{k}(t_{i})}{\sigma_{i}}
    \frac{f_{i}}{\sigma_{i}}
    \\
    \to &
    \sum_{j=1}^{m}
    \left(\sum_{i=1}^{n} A_{ik} A_{ij}\right) x_{j}
    =
    \sum_{i=1}^{n}
    A_{ik} \frac{f_{i}}{\sigma_{i}},
    ~~~~
    (k = 1,2,\ldots,m)
    \label{Eq:LSM-Minimum-Condition2}
\end{align}
となる. さらに
\begin{equation}
    b_{i} = \frac{f_{i}}{\sigma_{i}},
    ~~~~
    (i=1,2,\ldots,n)
\end{equation}
を第$i$成分にもつベクトル$\bm{b}$, 
$x_{j}$を第$j$成分にもつベクトルを$\bm{x}$とおくと, 式(\ref{Eq:LSM-Minimum-Condition2})
は次のようにかける:
\begin{equation}
    A^{t}A \bm{x} = A^{t} \bm{b}
\end{equation}
ここで, $A^{t}$は行列$A$の転置である. 
これを正規方程式という. 
この方程式を解けば, 未知係数$x_{i}$を求めることができる. 

\subsection{単純な多項式$\phi_{j}(t) = t^{j-1}$の場合}
単純な多項式$\phi_{j}(t) = t^{j-1}$を用いた, 最小二乗法を考える. 
\begin{align}
    f(t) &=
    \sum_{j=1}^{m} x_{j} t^{j-1} =
    x_{1} + x_{2} t + \ldots + x_{m}t^{m-1},
    \\
    A_{ij} &=
    \frac{t_{i}^{j-1}}{\sigma_{i}},
    \\
    A_{ik} &=
    \frac{t_{i}^{k-1}}{\sigma_{i}},
    \\
    b_{i} &=
    \frac{f_{i}}{\sigma_{i}}
\end{align}
であるので, 式(\ref{Eq:LSM-Minimum-Condition2})に代入すると, 
\begin{equation}
    \sum_{j=1}^{m}
    \left(
        \sum_{i=1}^{n}
        \frac{t_{i}^{k-1}}{\sigma_{i}}
        \frac{t_{i}^{j-1}}{\sigma_{i}}
    \right) x_{j}
    =
    \sum_{i=1}^{n}
    \frac{t_{i}^{k-1}}{\sigma_{i}}
    \frac{f_{i}}{\sigma_{i}}
\end{equation}
を得る. 行列形式で愚直に書くと, 
\begin{align}
    \begin{bmatrix}
        \sum_{i=1}^{n} \frac{1}{\sigma_{i}^{2}} t_{i}^{0} t_{i}^{0} &
        \sum_{i=1}^{n} \frac{1}{\sigma_{i}^{2}} t_{i}^{0} t_{i}^{1} &
        \ldots &
        \sum_{i=1}^{n} \frac{1}{\sigma_{i}^{2}} t_{i}^{0} t_{i}^{m-1}
        \\
        \sum_{i=1}^{n} \frac{1}{\sigma_{i}^{2}} t_{i}^{1} t_{i}^{0} &
        \sum_{i=1}^{n} \frac{1}{\sigma_{i}^{2}} t_{i}^{1} t_{i}^{1} &
        \ldots &
        \sum_{i=1}^{n} \frac{1}{\sigma_{i}^{2}} t_{i}^{1} t_{i}^{m-1}
        \\
        \vdots & \vdots & \ddots & \vdots
        \\
        \sum_{i=1}^{n} \frac{1}{\sigma_{i}^{2}} t_{i}^{m} t_{i}^{0} &
        \sum_{i=1}^{n} \frac{1}{\sigma_{i}^{2}} t_{i}^{m} t_{i}^{1} &
        \ldots &
        \sum_{i=1}^{n} \frac{1}{\sigma_{i}^{2}} t_{i}^{m} t_{i}^{m-1}
    \end{bmatrix}
    \begin{bmatrix}
        x_{0}  \\
        x_{1}  \\
        \vdots \\
        x_{m}
    \end{bmatrix}
    =
    \begin{bmatrix}
        \sum_{i=1}^{n} \frac{1}{\sigma_{i}^{2}} f_{i} \\
        \sum_{i=1}^{n} \frac{1}{\sigma_{i}^{2}} f_{i} t_{i} \\
        \vdots \\
        \sum_{i=1}^{n} \frac{1}{\sigma_{i}^{2}} f_{i} t_{i}^{m-1}
    \end{bmatrix}
\end{align}
となる. 
左辺の左の行列について, 
\begin{align}
    \bm{D} =
    \begin{bmatrix}
        \sum_{i=1}^{n} \frac{1}{\sigma_{i}^{2}} t_{i}^{0} &
        \sum_{i=1}^{n} \frac{1}{\sigma_{i}^{2}} t_{i}^{1} &
        \ldots &
        \sum_{i=1}^{n} \frac{1}{\sigma_{i}^{2}} t_{i}^{m-1}
        \\
        \sum_{i=1}^{n} \frac{1}{\sigma_{i}^{2}} t_{i}^{1} &
        \sum_{i=1}^{n} \frac{1}{\sigma_{i}^{2}} t_{i}^{2} &
        \ldots &
        \sum_{i=1}^{n} \frac{1}{\sigma_{i}^{2}} t_{i}^{m}
        \\
        \vdots & \vdots & \ddots & \vdots
        \\
        \sum_{i=1}^{n} \frac{1}{\sigma_{i}^{2}} t_{i}^{m-1} &
        \sum_{i=1}^{n} \frac{1}{\sigma_{i}^{2}} t_{i}^{m}   &
        \ldots &
        \sum_{i=1}^{n} \frac{1}{\sigma_{i}^{2}} t_{i}^{2m-1}
    \end{bmatrix}
    \label{Eq:LSM-Def-Matrix-D}
\end{align}
と置くと, 正規方程式は
\begin{align}
    \begin{bmatrix}
        x_{0}  \\
        x_{1}  \\
        \vdots \\
        x_{m}
    \end{bmatrix}
    &=
    \bm{D}^{-1}
    \begin{bmatrix}
        \sum_{i=1}^{n} \frac{1}{\sigma_{i}^{2}} f_{i} \\
        \sum_{i=1}^{n} \frac{1}{\sigma_{i}^{2}} f_{i} t_{i} \\
        \vdots \\
        \sum_{i=1}^{n} \frac{1}{\sigma_{i}^{2}} f_{i} t_{i}^{m-1}
    \end{bmatrix}
    \\
    &=
    \begin{bmatrix}
        D_{11}^{-1} & D_{12}^{-1} & \ldots & D_{1m}^{-1} \\
        D_{21}^{-1} & D_{22}^{-1} & \ldots & D_{2m}^{-1} \\
        \vdots & \vdots & \ddots & \vdots \\
        D_{m1}^{-1} & D_{m2}^{-1} & \ldots & D_{mm}^{-1} \\
    \end{bmatrix}
    \begin{bmatrix}
        \sum_{i=1}^{n} \frac{1}{\sigma_{i}^{2}} f_{i} \\
        \sum_{i=1}^{n} \frac{1}{\sigma_{i}^{2}} f_{i} t_{i} \\
        \vdots \\
        \sum_{i=1}^{n} \frac{1}{\sigma_{i}^{2}} f_{i} t_{i}^{m-1}
    \end{bmatrix}
    \\
    &=
    \begin{bmatrix}
        D_{11}^{-1} \sum_{i=1}^{n} \frac{1}{\sigma_{i}^{2}} f_{i} +
        D_{12}^{-1} \sum_{i=1}^{n} \frac{1}{\sigma_{i}^{2}} f_{i} t_{i} +
        \ldots +
        D_{1m}^{-1} \sum_{i=1}^{n} \frac{1}{\sigma_{i}^{2}} f_{i} t_{i}^{m-1}
        \\
        D_{21}^{-1} \sum_{i=1}^{n} \frac{1}{\sigma_{i}^{2}} f_{i} +
        D_{22}^{-1} \sum_{i=1}^{n} \frac{1}{\sigma_{i}^{2}} f_{i} t_{i} +
        \ldots +
        D_{2m}^{-1} \sum_{i=1}^{n} \frac{1}{\sigma_{i}^{2}} f_{i} t_{i}^{m-1}
        \\
        \vdots
        \\
        D_{m1}^{-1} \sum_{i=1}^{n} \frac{1}{\sigma_{i}^{2}} f_{i} +
        D_{m2}^{-1} \sum_{i=1}^{n} \frac{1}{\sigma_{i}^{2}} f_{i} t_{i} +
        \ldots +
        D_{mm}^{-1} \sum_{i=1}^{n} \frac{1}{\sigma_{i}^{2}} f_{i} t_{i}^{m-1}
    \end{bmatrix}
    \label{Eq:LSM-Coefficients-Solution}
\end{align}
のように解くことができる. 
また, 係数の誤差は, 
\begin{equation}
    \sigma_{xk}
    =
    \sqrt{
        \sum_{i=1}^{n}
        \left(\frac{\partial x_{k}}{\partial f_{i}} \sigma_{i}\right)^{2}
    }
\end{equation}
であるので, 式(\ref{Eq:LSM-Coefficients-Solution})を用いると具体的に
\begin{equation}
    \sigma_{xk}
    =
    \sqrt{
        \sum_{i=1}^{n}
        \left(
            \sum_{j=1}^{m} D_{kj}^{-1}
            \frac{1}{\sigma_{i}} t_{i}^{j-1}
        \right)^{2}
    }
\end{equation}
と計算される. 

\subsection{具体例: 一次関数$f(t) = x_{0} t + x_{1}$で最小二乗法}
一次関数$f(t) = x_{0} t + x_{1}$で最小二乗法を実行するときの, 具体的な未知係数と誤差の表式を見ていく. 
行列$(\ref{Eq:LSM-Def-Matrix-D})$を具体的に計算すると, 
\begin{equation}
    \bm{D}
    =
    \begin{bmatrix}
        \sum_{i=1}^{n} \frac{1}{\sigma_{i}^{2}} &
        \sum_{i=1}^{n} \frac{1}{\sigma_{i}^{2}} t_{i}
        \\
        \sum_{i=1}^{n} \frac{1}{\sigma_{i}^{2}} t_{i} &
        \sum_{i=1}^{n} \frac{1}{\sigma_{i}^{2}} t_{i}^{2}
    \end{bmatrix}
    \equiv
    \begin{bmatrix}
        \sum_{i=1}^{n} w_{i} &
        \sum_{i=1}^{n} w_{i} t_{i}
        \\
        \sum_{i=1}^{n} w_{i} t_{i} &
        \sum_{i=1}^{n} w_{i} t_{i}^{2}
    \end{bmatrix}
\end{equation}
なので, 逆行列は
\begin{align}
    \bm{D}^{-1}
    &=
    \frac{1}{\Delta}
    \begin{bmatrix}
        \sum_{i=1}^{n} w_{i} t_{i}^2&
        \sum_{i=1}^{n} w_{i} t_{i}
        \\
        \sum_{i=1}^{n} w_{i} t_{i} &
        \sum_{i=1}^{n} w_{i} t_{i}^{2}
    \end{bmatrix}
    \\
    \Delta
    &\equiv
    \left(\sum_{i=1}^{n} w_{i}\right)
    \left(\sum_{i=1}^{n} w_{i} t_{i}^{2}\right)
    -
    \left(\sum_{i=1}^{n} w_{i} t_{i}\right)^{2}
\end{align}
と計算される. ここで, $w_{i} = 1/\sigma_{i}^{2}$とおいた. 
したがって, 未知係数は
\begin{align}
    x_{0} &=
    \frac
    {
        \left(\sum_{i=1}^{n} w_{i} t_{i}^{2}\right)
        \left(\sum_{i=1}^{n} w_{i} f_{i}\right)
        -
        \left(\sum_{i=1}^{n} w_{i} t_{i}\right)
        \left(\sum_{i=1}^{n} w_{i} t_{i} f_{i}\right)
    }{\Delta}
    \\
    x_{1} &=
    \frac
    {
        \left(\sum_{i=1}^{n} w_{i} \right)
        \left(\sum_{i=1}^{n} w_{i} t_{i} f_{i}\right)
        -
        \left(\sum_{i=1}^{n} w_{i} t_{i}\right)
        \left(\sum_{i=1}^{n} w_{i} f_{i}\right)
    }{\Delta}
\end{align}
と計算される. 

続いて, 係数の誤差を求めていく. 
$x_{0}$の誤差は, 
\begin{equation}
    \sigma_{x_{0}}
    =
    \sqrt
    {
        \sum_{i=1}^{n}
        \left(\frac{\partial x_{0}}{\partial f_{i}} \sigma_{i}\right)^{2}
    }
\end{equation}
である. 以下, 具体的に計算をしていく:
\begin{align}
    \frac{\partial x_{0}}{\partial f_{i}} \sigma_{i}
    &=
    \frac{1}{\Delta}
    \left[
        \left(\sum_{i}^{n} w_{i} t_{i}^{2}\right) w_{i}
        -
        \left(\sum_{i}^{n} w_{i} t_{i}\right)w_{i} t_{i}
    \right]
    \sigma_{i}
    \\
    \left(\frac{\partial x_{0}}{\partial f_{i}} \sigma_{i}\right)^{2}
    &=
    \frac{1}{\Delta^{2}}
    \left[
        \left(\sum_{i=1}^{n} w_{i} t_{i}^{2}\right)^{2} w_{i}^{2}
        +
        \left(\sum_{i=1}^{n} w_{i} t_{i}\right)^{2} w_{i}^{2} t_{i}^{2}
        -
        2
        \left(\sum_{i=1}^{n} w_{i} t_{i}^{2}\right)
        \left(\sum_{i=1}^{n} w_{i} t_{i}\right) w_{i}^{2} t_{i}
    \right] \sigma_{i}^{2}
    \\ &=
    \frac{1}{\Delta^{2}}
    \left[
        \left(\sum_{i=1}^{n} w_{i} t_{i}^{2}\right)^{2} w_{i}
        +
        \left(\sum_{i=1}^{n} w_{i} t_{i}\right)^{2} w_{i} t_{i}^{2}
        -
        2
        \left(\sum_{i=1}^{n} w_{i} t_{i}^{2}\right)
        \left(\sum_{i=1}^{n} w_{i} t_{i}\right) w_{i} t_{i}
    \right]
\end{align}
最後の式変形には, $w_{i} = 1/\sigma_{i}^{2}$であることを用いた. 
さらに, $i$について和をとると, 
\begin{align}
    &
    \sum_{i=1}^{n}
    \left(\frac{\partial x_{0}}{\partial f_{i}} \sigma_{i}\right)^{2}
    \notag \\
    &=
    \frac{1}{\Delta^{2}}
    \left[
        \left(\sum_{i=1}^{n} w_{i} t_{i}^{2}\right)^{2}
        \left(\sum_{i=1}^{n} w_{i}\right)
        +
        \left(\sum_{i=1}^{n} w_{i} t_{i}\right)^{2}
        \left(\sum_{i=1}^{n} w_{i} t_{i}^{2} \right)
        -
        2
        \left(\sum_{i=1}^{n} w_{i} t_{i}^{2}\right)
        \left(\sum_{i=1}^{n} w_{i} t_{i}\right)^{2}
    \right]
    \notag
    \\ &=
    \frac{1}{\Delta^{2}}
    \left[
        \left(\sum_{i=1}^{n} w_{i} t_{i}^{2}\right)^{2}
        \left(\sum_{i=1}^{n} w_{i}\right)
        -
        \left(\sum_{i=1}^{n} w_{i} t_{i}\right)^{2}
        \left(\sum_{i=1}^{n} w_{i} t_{i}^{2}\right)
    \right]
    \notag
    \\ &=
    \frac{\sum_{i=1}^{n} w_{i} t_{i}^{2}}{\Delta^{2}}
    \left[
        \left(\sum_{i=1}^{n} w_{i} t_{i}^{2}\right)
        \left(\sum_{i=1}^{n} w_{i}\right)
        -
        \left(\sum_{i=1}^{n} w_{i} t_{i}\right)^{2}
    \right]
    \notag
    \\ &=
    \frac{\sum_{i=1}^{n} w_{i} t_{i}^{2}}{\Delta}
\end{align}
したがって, $x_{0}$の誤差は, 
\begin{equation}
    \sigma_{x_{0}} =
    \sqrt{\frac{\sum_{i=1}^{n} w_{i} t_{i}^{2}}{\Delta^{2}}}
\end{equation}
である. 

同様にして, $x_{1}$の誤差, 
\begin{equation}
    \sigma_{x_{1}}
    =
    \sqrt
    {
        \sum_{i=1}^{n}
        \left(\frac{\partial x_{1}}{\partial f_{i}} \sigma_{i}\right)^{2}
    }
\end{equation}
を計算する. 
\begin{align}
    \frac{\partial x_{1}}{\partial f_{i}} \sigma_{i}
    &=
    \frac{1}{\Delta}
    \left[
        \left(\sum_{i=1}^{n} w_{i}\right) w_{i} t_{i}
        -
        \left(\sum_{i=1}^{n} w_{i} t_{i}\right)w_{i}
    \right] \sigma_{i}
    \\
    \left(\frac{\partial x_{1}}{\partial f_{i}} \sigma_{i}\right)^{2}
    &=
    \frac{1}{\Delta^{2}}
    \left[
        \left(\sum_{i=1}^{n} w_{i}\right)^{2} w_{i}^{2} t_{i}^{2}
        +
        \left(\sum_{i=1}^{n} w_{i} t_{i}\right)^{2} w_{i}^{2}
        -
        2
        \left(\sum_{i=1}^{n} w_{i}\right)
        \left(\sum_{i=1}^{n} w_{i} t_{i}\right) w_{i}^{2} t_{i}
    \right] \sigma_{i}^{2}
    \\ &=
    \frac{1}{\Delta^{2}}
    \left[
        \left(\sum_{i=1}^{n} w_{i}\right)^{2} w_{i} t_{i}^{2}
        +
        \left(\sum_{i=1}^{n} w_{i} t_{i}\right)^{2} w_{i}
        -
        2
        \left(\sum_{i=1}^{n} w_{i}\right)
        \left(\sum_{i=1}^{n} w_{i} t_{i}\right) w_{i} t_{i}
    \right]
\end{align}
最後の式変形には, $w_{i} = 1/\sigma_{i}^{2}$であることを用いた. 
$i$について和をとると, 
\begin{align}
    &
    \sum_{i=1}^{n}
    \left(\frac{\partial x_{1}}{\partial f_{i}} \sigma_{i}\right)^{2}
    \notag \\
    &=
    \frac{1}{\Delta^{2}}
    \left[
        \left(\sum_{i=1}^{n} w_{i}\right)^{2}
        \left(\sum_{i=1}^{n} w_{i} t_{i}^{2}\right)
        +
        \left(\sum_{i=1}^{n} w_{i} t_{i}\right)^{2}
        \left(\sum_{i=1}^{n} w_{i}\right)
        -
        2
        \left(\sum_{i=1}^{n} w_{i}\right)
        \left(\sum_{i=1}^{n} w_{i} t_{i}\right)
        \left(\sum_{i=1}^{n} w_{i} t_{i}\right)
    \right]
    \notag
    \\ &=
    \frac{\sum_{i=1}^{n} w_{i}}{\Delta^{2}}
    \left[
        \left(\sum_{i=1}^{2} w_{i}\right)
        \left(\sum_{i=1}^{2} w_{i} t_{i}^{2}\right)
        -
        \left(\sum_{i=1}^{2} w_{i} t_{i}\right)^{2}
    \right]
    \notag
    \\ &=
    \frac{\sum_{i=1}^{n} w_{i}}{\Delta}
\end{align}
したがって, $x_{1}$の誤差は, 
\begin{equation}
    \sigma_{x_{1}}
    =
    \sqrt{\frac{\sum_{i=1}^{n} w_{i}}{\Delta}}
\end{equation}
である. 

\clearpage

\section{スプライン補間}
スプライン補間ではデータ点間の区画を補間する関数を決定する際に, 関数のつなぎ目において, できるだけ高次の微分まで滑らかになるように条件を課す.

\subsection{3次のスプライン補間}
\subsubsection{区分多項式}
データ点$x_{i}$と$x_{i+1}$の区間を次の3次関数で補間することを考える:
\begin{equation}
    S_{i} (x) =
    a_{i} +
    b_{i} (x - x_{i}) +
    c_{i} (x - x_{i})^{2} +
    d_{i} (x - x_{i})^{3}
    ,~~~~
    (i = 0,~1,~2,~\ldots,~N-1)
\end{equation}
この式を区分多項式と呼ぶ. N+1個のデータをつなぐためには, $N$個の区分多項式を使用する.
データを補間するには, 係数$a_{i}$, $b_{i}$, $c_{i}$, $d_{i}$が決める必要がある.
未知係数は全部で$4N$個であるので, $4N$個の方程式が必要である.
そこで, 以下の条件を課すことで未知係数を決定するための方程式を得る.
\begin{enumerate}
    \item 各$S_{i}(x)$に対して両端の値が決まっている. つまり, 全てのデータ点を通る.
    \begin{enumerate}
        \item 各区間の始点はデータ点の値をとる ($N$個の方程式).
        \item 隣合う区分多項式は, 境界点で同じ値をとる ($N$個の方程式).
    \end{enumerate}
    \item 各$S_{i}(x)$について, 境界点の一次微分は連続である ($N-1$個の方程式).
    \item 各$S_{i}(x)$について, 境界点の二次微分は連続である ($N-1$個の方程式).
    \item 自然なスプラインである条件: 最初と最後のデータ点$(j=0,N+1)$において, 二次微分がゼロである ($2$個の方程式).
\end{enumerate}
区分多項式の一次微分と二次微分はそれぞれ
\begin{alignat}{3}
    & S_{i}^{\prime} &&=
      b_{i} +
    2 c_{i} (x - x_{i}) +
    3 d_{i} (x - x_{i})^{2}
    \\
    & S_{i}^{\prime\prime} &&=
    2 c_{i} +
    6 d_{i} (x - x_{i})
\end{alignat}
である.

\subsubsection{未知係数の決定}

\paragraph{条件1-(a): 各区間の始点はデータ点の値をとる}
各区間の始点$S_{i}(x_{i})$でデータ点と同じ値であるとすると, 
\begin{equation}
    S_{i}(x_{i}) = y_{i}
    \notag
\end{equation}
すなわち, 
\begin{equation}
    a_{i} = y_{i}
\end{equation}
を得る.

\paragraph{条件4: 自然なスプラインである条件}
最初と最後のデータ点において, 二次微分がゼロであるので直ちに
\begin{alignat}{3}
    & c_{0}   &&= 0 \\
    & c_{N-1} &&= 0
\end{alignat}
を得る.

\paragraph{条件3: 境界点での二次微分の連続性}
境界点の二次微分は連続性は
\begin{equation}
    S_{i}^{\prime\prime} (x_{i+1})
    =
    S_{i+1}^{\prime\prime} (x_{i+1})
    \notag
\end{equation}
とかける. 具体的に計算すると,
\begin{equation}
    2 c_{i} + 6 d_{i} (x_{i+1} - x_{i})
    =
    2 c_{i+1} + 6 d_{i+1} (x_{i+1} - x_{i+1})
    \notag
\end{equation}
である. これより直ちに
\begin{equation}
    d_{i} = 
    \frac{c_{i+1} - c_{i}}{3(x_{i+1} - x_{i})}
    =
    \frac{c_{i+1} - c_{i}}{3 h_{i}}
\end{equation}
を得る. ここで$h_{i} \equiv x_{i+1} - x_{i}$を定義した.

\paragraph{条件1-(b): 隣合う区分多項式は, 境界点で同じ値をとる}
この条件を書き下すと
\begin{equation}
    S_{i}(x_{i+1}) = S_{i+1}(x_{i+1}) = y_{i+1}
    \notag
\end{equation}
とかける. 具体的に計算すると,
\begin{align}
    &
    a_{i} + b_{i} (x_{i+1} - x_{i}) +
    c_{i} (x_{i+1} - x_{i})^{2} + d_{i} (x_{i+1} - x_{i})^{3}
    \notag \\ =~
    &
    a_{i+1} + b_{i+1} (x_{i+1} - x_{i}) +
    c_{i+1} (x_{i+1} - x_{i})^{2} + d_{i+1} (x_{i+1} - x_{i})^{3}
    \notag \\
    \notag \\
    \to~
    &
    a_{i+1} = a_{i} + b_{i} h_{i} + c_{i} h_{i}^{2} + d_{i} h_{i}^{3}
    \notag
\end{align}
結果を整理すると, 
\begin{align}
    b_{i}
    &=
    \frac{a_{i+1} - a_{i}}{h_{i}} - c_{i} h_{i} - d_{i} h_{i}^{2}
    \notag \\
    &=
    \frac{a_{i+1} - a_{i}}{h_{i}} - c_{i} h_{i} -
    \frac{c_{i+1} - c_{i}}{3 h_{i}} h_{i}^{2}
    \notag \\
    &=
    \frac{a_{i+1} - a_{i}}{h_{i}} -
    \frac{h_{i}(c_{i+1} + 2c_{i})}{3}
\end{align}

\paragraph{条件2: 境界点での一次微分の連続性}
この条件を書き下すと
\begin{equation}
    S_{i}^{\prime} (x_{i+1}) = S_{i+1}^{\prime}(x_{i+1})
\end{equation}
とかける. 具体的に計算すると,
\begin{align}
    &
    b_{i} + 2 c_{i} (x_{i+1} - x_{i}) + 3 d_{i} (x_{i+1} - x_{i})^{2}
    \notag \\ = ~
    &
    b_{i+1} + 2 c_{i+1} (x_{i+1} - x_{i}) + 3 d_{i+1} (x_{i+1} - x_{i})^{2}
    \notag \\
    \notag \\ \to~
    &
    b_{i+1} = b_{i} + 2c_{i} (x_{i+1} - x_{i}) + 3 d_{i} (x_{i+1} - x_{i})^{2}
    \notag \\ \to~
    &
    b_{i+1} = b_{i} + 2c_{i} h_{i} + 3d_{i} h_{i}^{2}
    \notag
\end{align}
を得る. ここまでに得られた$a_{i}$, $b_{i}$, $d_{i}$の表式を代入すると,
\begin{align}
    \left[
        \frac{a_{i+2} - a_{i+1}}{h_{i}} -
        \frac{h_{i}(c_{i+2} + 2c_{i+1})}{3}
    \right]
    =
    \left[
        \frac{a_{i+1} - a_{i}}{h_{i}} -
        \frac{h_{i}(c_{i+1} + 2c_{i})}{3}
    \right]
    +
    2c_{i}h_{i}
    +
    3 \left(\frac{c_{i+1} - c_{i}}{3 h_{i}}\right) h_{i}^{2}
\end{align}
となる. さらに整理すると
\begin{align}
    h_{i+1} c_{i+2} + 2(h_{i+1} + h_{i}) c_{i+1} + c_{i}h_{i}
    &=
    \frac{3}{h_{i-1}}(a_{i+2} - a_{i+1}) -
    \frac{3}{h_{i}}(a_{i+1} - a_{i})
    \notag \\
    =
    \frac{3}{h_{i-1}}(y_{i+2} - y_{i+1}) -
    \frac{3}{h_{i}}(y_{i+1} - y_{i})
\end{align}
を得る. この式を行列形式で書くと
\begin{align}
    &
    \begin{pmatrix}
        1 & 0 &  0 & \cdots & 0 & 0 & 0 & 0\\
        h_{0} & 2(h_{0} - h_{1}) & h_{1} & 0 & 0& \cdots & 0 & 0\\
        0 & h_{1} & 2(h_{1} - h_{2}) & h_{2} & 0 & \cdots &  0 & 0\\
        0 & 0 & h_{2} & 2(h_{2} - h_{3}) & h_{3} & 0 & \cdots & 0 \\
        \vdots & \vdots & \vdots & &  \vdots \\
        0 & 0 &  0 & \cdots & 0 & 0 & 0 & 1
    \end{pmatrix}
    \begin{pmatrix}
        c_{0} \\ c_{1} \\ \vdots \\ \vdots \\ c_{N-2} \\ c_{N-1}
    \end{pmatrix}
    =
    \notag \\
    &
    \begin{pmatrix}
        0 \\
        \frac{3}{h_{1}}(y_{2} - y_{1}) - \frac{3}{h_{0}}(y_{1} - y_{0}) \\
        \vdots \\
        \vdots \\
        \frac{3}{h_{n-1}}(y_{n} - y_{n-1}) - \frac{3}{h_{i}}(y_{n-1} - y_{n-2}) \\
        0
    \end{pmatrix}
    \notag
\end{align}
となる. この連立方程式を解くと係数$c_{i}$を決定できる.

\clearpage
%\bibliographystyle{junsrt}
%\bibliography{}
\end{document}

